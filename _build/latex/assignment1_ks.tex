%% Generated by Sphinx.
\def\sphinxdocclass{jupyterBook}
\documentclass[letterpaper,10pt,english]{jupyterBook}
\ifdefined\pdfpxdimen
   \let\sphinxpxdimen\pdfpxdimen\else\newdimen\sphinxpxdimen
\fi \sphinxpxdimen=.75bp\relax
%% turn off hyperref patch of \index as sphinx.xdy xindy module takes care of
%% suitable \hyperpage mark-up, working around hyperref-xindy incompatibility
\PassOptionsToPackage{hyperindex=false}{hyperref}
%% memoir class requires extra handling
\makeatletter\@ifclassloaded{memoir}
{\ifdefined\memhyperindexfalse\memhyperindexfalse\fi}{}\makeatother

\PassOptionsToPackage{warn}{textcomp}

\catcode`^^^^00a0\active\protected\def^^^^00a0{\leavevmode\nobreak\ }
\usepackage{cmap}
\usepackage{fontspec}
\defaultfontfeatures[\rmfamily,\sffamily,\ttfamily]{}
\usepackage{amsmath,amssymb,amstext}
\usepackage{polyglossia}
\setmainlanguage{english}



\setmainfont{FreeSerif}[
  Extension      = .otf,
  UprightFont    = *,
  ItalicFont     = *Italic,
  BoldFont       = *Bold,
  BoldItalicFont = *BoldItalic
]
\setsansfont{FreeSans}[
  Extension      = .otf,
  UprightFont    = *,
  ItalicFont     = *Oblique,
  BoldFont       = *Bold,
  BoldItalicFont = *BoldOblique,
]
\setmonofont{FreeMono}[
  Extension      = .otf,
  UprightFont    = *,
  ItalicFont     = *Oblique,
  BoldFont       = *Bold,
  BoldItalicFont = *BoldOblique,
]


\usepackage[Bjarne]{fncychap}
\usepackage[,numfigreset=1,mathnumfig]{sphinx}

\fvset{fontsize=\small}
\usepackage{geometry}


% Include hyperref last.
\usepackage{hyperref}
% Fix anchor placement for figures with captions.
\usepackage{hypcap}% it must be loaded after hyperref.
% Set up styles of URL: it should be placed after hyperref.
\urlstyle{same}


\usepackage{sphinxmessages}



        % Start of preamble defined in sphinx-jupyterbook-latex %
         \usepackage[Latin,Greek]{ucharclasses}
        \usepackage{unicode-math}
        % fixing title of the toc
        \addto\captionsenglish{\renewcommand{\contentsname}{Contents}}
        \hypersetup{
            pdfencoding=auto,
            psdextra
        }
        % End of preamble defined in sphinx-jupyterbook-latex %
        

\title{CA4015 Assignment 1}
\date{Oct 12, 2021}
\release{}
\author{Kian Sweeney}
\newcommand{\sphinxlogo}{\vbox{}}
\renewcommand{\releasename}{}
\makeindex
\begin{document}

\pagestyle{empty}
\sphinxmaketitle
\pagestyle{plain}
\sphinxtableofcontents
\pagestyle{normal}
\phantomsection\label{\detokenize{introduction::doc}}


\begin{DUlineblock}{0em}
\item[] \sphinxstylestrong{\Large Introduction}
\end{DUlineblock}

\begin{DUlineblock}{0em}
\item[] \sphinxstylestrong{\large Iowa Gambling Task¶}
\end{DUlineblock}

\sphinxAtStartPar
The \sphinxhref{https://www.youtube.com/watch?v=A6SsQIyJMhs}{Iowa Gambling Task}(IGT) is a pyscological card game thought to portray real world decision making and help understand how and why people make decisions. Typically, the particpiants start with a loan of \$2000 and are typically given a deck of four cards to pick from. The deck either rewards or punishes the particpiant each time and the end goal of the game is to make more money than what they started with. Some of the cards are more favourable than others, providing a steady winning return over the long term whereas others can be typically destructive but sporadically produce high winnings.

\sphinxAtStartPar
\sphinxincludegraphics{{326dcd02bb0538a100762d162750e2a24563b062}.jpg}

\begin{DUlineblock}{0em}
\item[] \sphinxstylestrong{\large Dataset}
\end{DUlineblock}

\sphinxAtStartPar
For the purpose of this assignment there is varying trial lengths for each task. Typically, the task is run in 100 trials but in this instance we have a 95 trial, 100 trial and 150 trial experiments. We were given a variety of data for the trials. There were 617 particpiants across all the studies and all were reported as “healthy”. We had each subjects choice on their respective trial, their winnings on the respective rounds combined with their losses for each round. We also had an “index” file which told us which study the subject was part of. An overview of our data and other information such as the particpiant demographics can be seen \sphinxhref{https://openpsychologydata.metajnl.com/articles/10.5334/jopd.ak/}{here}.

\begin{DUlineblock}{0em}
\item[] \sphinxstylestrong{\large Methodology}
\end{DUlineblock}

\sphinxAtStartPar
Initially, we start by breaking down the data and seeing any interesting trends. We try to see which studies produced the highest winnings/losses and how subjects decision making flowed over time (i.e. did they consistently win small or try change strategy by going for broke and another set of cards). We try to combine this with the background demographic information provided to see can we tell anything from the studies. We then cluster the data accordingly by winnings and studys to detect any trends and try tell a story from the data.


\chapter{1. Data Analysis and Experiments}
\label{\detokenize{data-analysis:data-analysis-and-experiments}}\label{\detokenize{data-analysis::doc}}

\section{Read in data}
\label{\detokenize{data-analysis:read-in-data}}
\begin{sphinxVerbatim}[commandchars=\\\{\}]
\PYG{k+kn}{import} \PYG{n+nn}{pandas} \PYG{k}{as} \PYG{n+nn}{pd}
\PYG{k+kn}{import} \PYG{n+nn}{matplotlib}\PYG{n+nn}{.}\PYG{n+nn}{pyplot} \PYG{k}{as} \PYG{n+nn}{plt}
\PYG{k+kn}{import} \PYG{n+nn}{numpy} \PYG{k}{as} \PYG{n+nn}{np}
\PYG{k+kn}{import} \PYG{n+nn}{seaborn} \PYG{k}{as} \PYG{n+nn}{sns}
\end{sphinxVerbatim}

\begin{sphinxVerbatim}[commandchars=\\\{\}]
\PYG{n}{df} \PYG{o}{=} \PYG{n}{pd}\PYG{o}{.}\PYG{n}{read\PYGZus{}csv}\PYG{p}{(}\PYG{l+s+s1}{\PYGZsq{}}\PYG{l+s+s1}{data/choice\PYGZus{}100.csv}\PYG{l+s+s1}{\PYGZsq{}}\PYG{p}{)}
\PYG{n}{df}\PYG{o}{.}\PYG{n}{head}\PYG{p}{(}\PYG{p}{)}
\end{sphinxVerbatim}

\begin{sphinxVerbatim}[commandchars=\\\{\}]
        Choice\PYGZus{}1  Choice\PYGZus{}2  Choice\PYGZus{}3  Choice\PYGZus{}4  Choice\PYGZus{}5  Choice\PYGZus{}6  Choice\PYGZus{}7  \PYGZbs{}
Subj\PYGZus{}1         1         1         2         4         3         2         1   
Subj\PYGZus{}2         2         1         4         4         3         2         3   
Subj\PYGZus{}3         4         2         3         1         4         2         4   
Subj\PYGZus{}4         4         3         4         2         1         4         3   
Subj\PYGZus{}5         1         2         2         2         2         3         4   

        Choice\PYGZus{}8  Choice\PYGZus{}9  Choice\PYGZus{}10  ...  Choice\PYGZus{}91  Choice\PYGZus{}92  Choice\PYGZus{}93  \PYGZbs{}
Subj\PYGZus{}1         2         4          2  ...          1          1          1   
Subj\PYGZus{}2         2         1          2  ...          4          2          3   
Subj\PYGZus{}3         4         4          3  ...          3          2          1   
Subj\PYGZus{}4         2         2          2  ...          4          2          3   
Subj\PYGZus{}5         1         4          1  ...          2          2          2   

        Choice\PYGZus{}94  Choice\PYGZus{}95  Choice\PYGZus{}96  Choice\PYGZus{}97  Choice\PYGZus{}98  Choice\PYGZus{}99  \PYGZbs{}
Subj\PYGZus{}1          2          2          2          4          2          4   
Subj\PYGZus{}2          4          2          4          4          2          2   
Subj\PYGZus{}3          4          2          2          2          4          2   
Subj\PYGZus{}4          4          3          4          1          4          3   
Subj\PYGZus{}5          2          3          3          3          3          4   

        Choice\PYGZus{}100  
Subj\PYGZus{}1           2  
Subj\PYGZus{}2           4  
Subj\PYGZus{}3           2  
Subj\PYGZus{}4           4  
Subj\PYGZus{}5           4  

[5 rows x 100 columns]
\end{sphinxVerbatim}

\begin{sphinxVerbatim}[commandchars=\\\{\}]
\PYG{n}{index95} \PYG{o}{=} \PYG{n}{pd}\PYG{o}{.}\PYG{n}{read\PYGZus{}csv}\PYG{p}{(}\PYG{l+s+s1}{\PYGZsq{}}\PYG{l+s+s1}{data/index\PYGZus{}95.csv}\PYG{l+s+s1}{\PYGZsq{}}\PYG{p}{)}
\PYG{n}{index100} \PYG{o}{=} \PYG{n}{pd}\PYG{o}{.}\PYG{n}{read\PYGZus{}csv}\PYG{p}{(}\PYG{l+s+s1}{\PYGZsq{}}\PYG{l+s+s1}{data/index\PYGZus{}100.csv}\PYG{l+s+s1}{\PYGZsq{}}\PYG{p}{)}
\PYG{n}{index150} \PYG{o}{=} \PYG{n}{pd}\PYG{o}{.}\PYG{n}{read\PYGZus{}csv}\PYG{p}{(}\PYG{l+s+s1}{\PYGZsq{}}\PYG{l+s+s1}{data/index\PYGZus{}150.csv}\PYG{l+s+s1}{\PYGZsq{}}\PYG{p}{)}
\end{sphinxVerbatim}

\sphinxAtStartPar
Let’s just confirm our python version we are using on our scripts

\begin{sphinxVerbatim}[commandchars=\\\{\}]
\PYG{k+kn}{from} \PYG{n+nn}{platform} \PYG{k+kn}{import} \PYG{n}{python\PYGZus{}version}
\PYG{n+nb}{print}\PYG{p}{(}\PYG{n}{python\PYGZus{}version}\PYG{p}{(}\PYG{p}{)}\PYG{p}{)}
\end{sphinxVerbatim}

\begin{sphinxVerbatim}[commandchars=\\\{\}]
3.8.8
\end{sphinxVerbatim}


\chapter{Initial data exploration \sphinxhyphen{} How many per study made profit?}
\label{\detokenize{data-analysis:initial-data-exploration-how-many-per-study-made-profit}}
\sphinxAtStartPar
Here we check the study that used 95 trials of the experiment and see how many of the 15 subjects made profit.

\begin{sphinxVerbatim}[commandchars=\\\{\}]
\PYG{n}{win95} \PYG{o}{=} \PYG{n}{pd}\PYG{o}{.}\PYG{n}{read\PYGZus{}csv}\PYG{p}{(}\PYG{l+s+s1}{\PYGZsq{}}\PYG{l+s+s1}{data/wi\PYGZus{}95.csv}\PYG{l+s+s1}{\PYGZsq{}}\PYG{p}{)}
\PYG{n}{loss95} \PYG{o}{=} \PYG{n}{pd}\PYG{o}{.}\PYG{n}{read\PYGZus{}csv}\PYG{p}{(}\PYG{l+s+s1}{\PYGZsq{}}\PYG{l+s+s1}{data/lo\PYGZus{}95.csv}\PYG{l+s+s1}{\PYGZsq{}}\PYG{p}{)}
\PYG{n}{totalloss95} \PYG{o}{=} \PYG{n}{loss95}\PYG{o}{.}\PYG{n}{sum}\PYG{p}{(}\PYG{n}{axis}\PYG{o}{=}\PYG{l+m+mi}{1}\PYG{p}{)}
\PYG{n}{totalloss95}\PYG{o}{.}\PYG{n}{head}\PYG{p}{(}\PYG{p}{)}
\end{sphinxVerbatim}

\begin{sphinxVerbatim}[commandchars=\\\{\}]
Subj\PYGZus{}1   \PYGZhy{}4650
Subj\PYGZus{}2   \PYGZhy{}7925
Subj\PYGZus{}3   \PYGZhy{}7850
Subj\PYGZus{}4   \PYGZhy{}7525
Subj\PYGZus{}5   \PYGZhy{}6350
dtype: int64
\end{sphinxVerbatim}

\begin{sphinxVerbatim}[commandchars=\\\{\}]
\PYG{n}{totalwin95} \PYG{o}{=} \PYG{n}{win95}\PYG{o}{.}\PYG{n}{sum}\PYG{p}{(}\PYG{n}{axis}\PYG{o}{=}\PYG{l+m+mi}{1}\PYG{p}{)}
\PYG{n}{totalwin95}\PYG{o}{.}\PYG{n}{head}\PYG{p}{(}\PYG{p}{)}
\end{sphinxVerbatim}

\begin{sphinxVerbatim}[commandchars=\\\{\}]
Subj\PYGZus{}1    5800
Subj\PYGZus{}2    7250
Subj\PYGZus{}3    7100
Subj\PYGZus{}4    7000
Subj\PYGZus{}5    6450
dtype: int64
\end{sphinxVerbatim}

\begin{sphinxVerbatim}[commandchars=\\\{\}]
\PYG{n}{margin95} \PYG{o}{=} \PYG{n}{totalwin95} \PYG{o}{+} \PYG{n}{totalloss95}
\PYG{n}{margin95}\PYG{o}{.}\PYG{n}{head}\PYG{p}{(}\PYG{p}{)}
\end{sphinxVerbatim}

\begin{sphinxVerbatim}[commandchars=\\\{\}]
Subj\PYGZus{}1    1150
Subj\PYGZus{}2    \PYGZhy{}675
Subj\PYGZus{}3    \PYGZhy{}750
Subj\PYGZus{}4    \PYGZhy{}525
Subj\PYGZus{}5     100
dtype: int64
\end{sphinxVerbatim}

\begin{sphinxVerbatim}[commandchars=\\\{\}]
\PYG{n}{columns} \PYG{o}{=} \PYG{p}{[}\PYG{l+s+s1}{\PYGZsq{}}\PYG{l+s+s1}{Margin}\PYG{l+s+s1}{\PYGZsq{}}\PYG{p}{]}

\PYG{n}{margin95df} \PYG{o}{=} \PYG{n}{pd}\PYG{o}{.}\PYG{n}{DataFrame}\PYG{p}{(}\PYG{n}{margin95}\PYG{p}{,} \PYG{n}{columns}\PYG{o}{=}\PYG{n}{columns}\PYG{p}{)}
\PYG{n}{margin95df}
\PYG{n}{ax1} \PYG{o}{=} \PYG{n}{margin95df}\PYG{o}{.}\PYG{n}{plot}\PYG{o}{.}\PYG{n}{kde}\PYG{p}{(}\PYG{n}{title}\PYG{o}{=}\PYG{l+s+s1}{\PYGZsq{}}\PYG{l+s+s1}{Density plot of Profit/Loss margin for 95 people experiments}\PYG{l+s+s1}{\PYGZsq{}}\PYG{p}{,} \PYG{n}{color}\PYG{o}{=}\PYG{l+s+s1}{\PYGZsq{}}\PYG{l+s+s1}{turquoise}\PYG{l+s+s1}{\PYGZsq{}}\PYG{p}{)}
\PYG{n}{ax1}\PYG{o}{.}\PYG{n}{axvline}\PYG{p}{(}\PYG{n}{x}\PYG{o}{=}\PYG{l+m+mi}{0}\PYG{p}{,} \PYG{n}{linestyle}\PYG{o}{=}\PYG{l+s+s1}{\PYGZsq{}}\PYG{l+s+s1}{\PYGZhy{}\PYGZhy{}}\PYG{l+s+s1}{\PYGZsq{}}\PYG{p}{,} \PYG{n}{color}\PYG{o}{=}\PYG{l+s+s1}{\PYGZsq{}}\PYG{l+s+s1}{red}\PYG{l+s+s1}{\PYGZsq{}}\PYG{p}{)}
\end{sphinxVerbatim}

\begin{sphinxVerbatim}[commandchars=\\\{\}]
\PYGZlt{}matplotlib.lines.Line2D at 0x1c611ab4eb0\PYGZgt{}
\end{sphinxVerbatim}

\noindent\sphinxincludegraphics{{data-analysis_11_1}.png}

\begin{sphinxVerbatim}[commandchars=\\\{\}]
\PYG{n+nb}{sum}\PYG{p}{(}\PYG{n}{margin95df}\PYG{o}{.}\PYG{n}{select\PYGZus{}dtypes}\PYG{p}{(}\PYG{n}{np}\PYG{o}{.}\PYG{n}{number}\PYG{p}{)}\PYG{o}{.}\PYG{n}{gt}\PYG{p}{(}\PYG{l+m+mi}{0}\PYG{p}{)}\PYG{o}{.}\PYG{n}{sum}\PYG{p}{(}\PYG{n}{axis}\PYG{o}{=}\PYG{l+m+mi}{1}\PYG{p}{)}\PYG{p}{)}
\end{sphinxVerbatim}

\begin{sphinxVerbatim}[commandchars=\\\{\}]
7
\end{sphinxVerbatim}

\sphinxAtStartPar
Under half (7/15) of the participants made profit in the 95 trial experiment. We now do the same for both the 100 trial and 150 trial experiments

\begin{sphinxVerbatim}[commandchars=\\\{\}]
\PYG{n}{win100} \PYG{o}{=} \PYG{n}{pd}\PYG{o}{.}\PYG{n}{read\PYGZus{}csv}\PYG{p}{(}\PYG{l+s+s1}{\PYGZsq{}}\PYG{l+s+s1}{data/wi\PYGZus{}100.csv}\PYG{l+s+s1}{\PYGZsq{}}\PYG{p}{)}
\PYG{n}{loss100} \PYG{o}{=} \PYG{n}{pd}\PYG{o}{.}\PYG{n}{read\PYGZus{}csv}\PYG{p}{(}\PYG{l+s+s1}{\PYGZsq{}}\PYG{l+s+s1}{data/lo\PYGZus{}100.csv}\PYG{l+s+s1}{\PYGZsq{}}\PYG{p}{)}
\PYG{n}{totalloss100} \PYG{o}{=} \PYG{n}{loss100}\PYG{o}{.}\PYG{n}{sum}\PYG{p}{(}\PYG{n}{axis}\PYG{o}{=}\PYG{l+m+mi}{1}\PYG{p}{)}
\PYG{n}{totalloss100}\PYG{o}{.}\PYG{n}{head}\PYG{p}{(}\PYG{p}{)}
\end{sphinxVerbatim}

\begin{sphinxVerbatim}[commandchars=\\\{\}]
Subj\PYGZus{}1   \PYGZhy{}9950
Subj\PYGZus{}2   \PYGZhy{}8250
Subj\PYGZus{}3   \PYGZhy{}8600
Subj\PYGZus{}4   \PYGZhy{}5650
Subj\PYGZus{}5   \PYGZhy{}8600
dtype: int64
\end{sphinxVerbatim}

\begin{sphinxVerbatim}[commandchars=\\\{\}]
\PYG{n}{totalwin100} \PYG{o}{=} \PYG{n}{win100}\PYG{o}{.}\PYG{n}{sum}\PYG{p}{(}\PYG{n}{axis}\PYG{o}{=}\PYG{l+m+mi}{1}\PYG{p}{)}
\PYG{n}{totalwin100}\PYG{o}{.}\PYG{n}{head}\PYG{p}{(}\PYG{p}{)}
\end{sphinxVerbatim}

\begin{sphinxVerbatim}[commandchars=\\\{\}]
Subj\PYGZus{}1    8150
Subj\PYGZus{}2    7450
Subj\PYGZus{}3    8150
Subj\PYGZus{}4    6850
Subj\PYGZus{}5    7300
dtype: int64
\end{sphinxVerbatim}

\begin{sphinxVerbatim}[commandchars=\\\{\}]
\PYG{n}{margin100} \PYG{o}{=} \PYG{n}{totalwin100} \PYG{o}{+} \PYG{n}{totalloss100}
\PYG{n}{margin100}\PYG{o}{.}\PYG{n}{head}\PYG{p}{(}\PYG{p}{)}
\end{sphinxVerbatim}

\begin{sphinxVerbatim}[commandchars=\\\{\}]
Subj\PYGZus{}1   \PYGZhy{}1800
Subj\PYGZus{}2    \PYGZhy{}800
Subj\PYGZus{}3    \PYGZhy{}450
Subj\PYGZus{}4    1200
Subj\PYGZus{}5   \PYGZhy{}1300
dtype: int64
\end{sphinxVerbatim}

\begin{sphinxVerbatim}[commandchars=\\\{\}]
\PYG{n}{columns} \PYG{o}{=} \PYG{p}{[}\PYG{l+s+s1}{\PYGZsq{}}\PYG{l+s+s1}{Margin}\PYG{l+s+s1}{\PYGZsq{}}\PYG{p}{]}

\PYG{n}{margin100df} \PYG{o}{=} \PYG{n}{pd}\PYG{o}{.}\PYG{n}{DataFrame}\PYG{p}{(}\PYG{n}{margin100}\PYG{p}{,} \PYG{n}{columns}\PYG{o}{=}\PYG{n}{columns}\PYG{p}{)}
\PYG{n}{margin100df}
\PYG{n}{ax2} \PYG{o}{=} \PYG{n}{margin100df}\PYG{o}{.}\PYG{n}{plot}\PYG{o}{.}\PYG{n}{kde}\PYG{p}{(}\PYG{n}{title}\PYG{o}{=}\PYG{l+s+s1}{\PYGZsq{}}\PYG{l+s+s1}{Density plot of Profit/Loss margin for 100 people experiments}\PYG{l+s+s1}{\PYGZsq{}}\PYG{p}{,} \PYG{n}{color}\PYG{o}{=}\PYG{l+s+s1}{\PYGZsq{}}\PYG{l+s+s1}{turquoise}\PYG{l+s+s1}{\PYGZsq{}}\PYG{p}{)}
\PYG{n}{ax2}\PYG{o}{.}\PYG{n}{axvline}\PYG{p}{(}\PYG{n}{x}\PYG{o}{=}\PYG{l+m+mi}{0}\PYG{p}{,} \PYG{n}{linestyle}\PYG{o}{=}\PYG{l+s+s1}{\PYGZsq{}}\PYG{l+s+s1}{\PYGZhy{}\PYGZhy{}}\PYG{l+s+s1}{\PYGZsq{}}\PYG{p}{,} \PYG{n}{color}\PYG{o}{=}\PYG{l+s+s1}{\PYGZsq{}}\PYG{l+s+s1}{red}\PYG{l+s+s1}{\PYGZsq{}}\PYG{p}{)}
\end{sphinxVerbatim}

\begin{sphinxVerbatim}[commandchars=\\\{\}]
\PYGZlt{}matplotlib.lines.Line2D at 0x1c61330ea60\PYGZgt{}
\end{sphinxVerbatim}

\noindent\sphinxincludegraphics{{data-analysis_17_1}.png}

\begin{sphinxVerbatim}[commandchars=\\\{\}]
\PYG{n+nb}{sum}\PYG{p}{(}\PYG{n}{margin100df}\PYG{o}{.}\PYG{n}{select\PYGZus{}dtypes}\PYG{p}{(}\PYG{n}{np}\PYG{o}{.}\PYG{n}{number}\PYG{p}{)}\PYG{o}{.}\PYG{n}{gt}\PYG{p}{(}\PYG{l+m+mi}{0}\PYG{p}{)}\PYG{o}{.}\PYG{n}{sum}\PYG{p}{(}\PYG{n}{axis}\PYG{o}{=}\PYG{l+m+mi}{1}\PYG{p}{)}\PYG{p}{)}
\end{sphinxVerbatim}

\begin{sphinxVerbatim}[commandchars=\\\{\}]
208
\end{sphinxVerbatim}

\sphinxAtStartPar
Only 41\% of participants in the 100 trial experiment made money!

\begin{sphinxVerbatim}[commandchars=\\\{\}]
\PYG{n}{win150} \PYG{o}{=} \PYG{n}{pd}\PYG{o}{.}\PYG{n}{read\PYGZus{}csv}\PYG{p}{(}\PYG{l+s+s1}{\PYGZsq{}}\PYG{l+s+s1}{data/wi\PYGZus{}150.csv}\PYG{l+s+s1}{\PYGZsq{}}\PYG{p}{)}
\PYG{n}{loss150} \PYG{o}{=} \PYG{n}{pd}\PYG{o}{.}\PYG{n}{read\PYGZus{}csv}\PYG{p}{(}\PYG{l+s+s1}{\PYGZsq{}}\PYG{l+s+s1}{data/lo\PYGZus{}150.csv}\PYG{l+s+s1}{\PYGZsq{}}\PYG{p}{)}
\PYG{n}{totalloss150} \PYG{o}{=} \PYG{n}{loss150}\PYG{o}{.}\PYG{n}{sum}\PYG{p}{(}\PYG{n}{axis}\PYG{o}{=}\PYG{l+m+mi}{1}\PYG{p}{)}
\PYG{n}{totalloss150}\PYG{o}{.}\PYG{n}{head}\PYG{p}{(}\PYG{p}{)}
\end{sphinxVerbatim}

\begin{sphinxVerbatim}[commandchars=\\\{\}]
Subj\PYGZus{}1   \PYGZhy{}12200
Subj\PYGZus{}2   \PYGZhy{}13950
Subj\PYGZus{}3    \PYGZhy{}9300
Subj\PYGZus{}4    \PYGZhy{}6750
Subj\PYGZus{}5    \PYGZhy{}6300
dtype: int64
\end{sphinxVerbatim}

\begin{sphinxVerbatim}[commandchars=\\\{\}]
\PYG{n}{totalwin150} \PYG{o}{=} \PYG{n}{win150}\PYG{o}{.}\PYG{n}{sum}\PYG{p}{(}\PYG{n}{axis}\PYG{o}{=}\PYG{l+m+mi}{1}\PYG{p}{)}
\PYG{n}{totalwin150}\PYG{o}{.}\PYG{n}{head}\PYG{p}{(}\PYG{p}{)}
\end{sphinxVerbatim}

\begin{sphinxVerbatim}[commandchars=\\\{\}]
Subj\PYGZus{}1    11650
Subj\PYGZus{}2    12350
Subj\PYGZus{}3    10200
Subj\PYGZus{}4     8950
Subj\PYGZus{}5     8200
dtype: int64
\end{sphinxVerbatim}

\begin{sphinxVerbatim}[commandchars=\\\{\}]
\PYG{n}{margin150} \PYG{o}{=} \PYG{n}{totalwin150} \PYG{o}{+} \PYG{n}{totalloss150}
\PYG{n}{margin150}\PYG{o}{.}\PYG{n}{head}\PYG{p}{(}\PYG{p}{)}
\end{sphinxVerbatim}

\begin{sphinxVerbatim}[commandchars=\\\{\}]
Subj\PYGZus{}1    \PYGZhy{}550
Subj\PYGZus{}2   \PYGZhy{}1600
Subj\PYGZus{}3     900
Subj\PYGZus{}4    2200
Subj\PYGZus{}5    1900
dtype: int64
\end{sphinxVerbatim}

\begin{sphinxVerbatim}[commandchars=\\\{\}]
\PYG{n}{columns} \PYG{o}{=} \PYG{p}{[}\PYG{l+s+s1}{\PYGZsq{}}\PYG{l+s+s1}{Margin}\PYG{l+s+s1}{\PYGZsq{}}\PYG{p}{]}

\PYG{n}{margin150df} \PYG{o}{=} \PYG{n}{pd}\PYG{o}{.}\PYG{n}{DataFrame}\PYG{p}{(}\PYG{n}{margin150}\PYG{p}{,} \PYG{n}{columns}\PYG{o}{=}\PYG{n}{columns}\PYG{p}{)}
\PYG{n}{margin150df}
\PYG{n}{ax} \PYG{o}{=} \PYG{n}{margin150df}\PYG{o}{.}\PYG{n}{plot}\PYG{o}{.}\PYG{n}{kde}\PYG{p}{(}\PYG{n}{title}\PYG{o}{=}\PYG{l+s+s1}{\PYGZsq{}}\PYG{l+s+s1}{Density plot of Profit/Loss margin for 150 people experiments}\PYG{l+s+s1}{\PYGZsq{}}\PYG{p}{,} \PYG{n}{color}\PYG{o}{=}\PYG{l+s+s1}{\PYGZsq{}}\PYG{l+s+s1}{turquoise}\PYG{l+s+s1}{\PYGZsq{}}\PYG{p}{)}
\PYG{n}{ax}\PYG{o}{.}\PYG{n}{axvline}\PYG{p}{(}\PYG{n}{x}\PYG{o}{=}\PYG{l+m+mi}{0}\PYG{p}{,} \PYG{n}{linestyle}\PYG{o}{=}\PYG{l+s+s1}{\PYGZsq{}}\PYG{l+s+s1}{\PYGZhy{}\PYGZhy{}}\PYG{l+s+s1}{\PYGZsq{}}\PYG{p}{,} \PYG{n}{color}\PYG{o}{=}\PYG{l+s+s1}{\PYGZsq{}}\PYG{l+s+s1}{red}\PYG{l+s+s1}{\PYGZsq{}}\PYG{p}{)}
\end{sphinxVerbatim}

\begin{sphinxVerbatim}[commandchars=\\\{\}]
\PYGZlt{}matplotlib.lines.Line2D at 0x1c613363d60\PYGZgt{}
\end{sphinxVerbatim}

\noindent\sphinxincludegraphics{{data-analysis_23_1}.png}

\begin{sphinxVerbatim}[commandchars=\\\{\}]
\PYG{n+nb}{sum}\PYG{p}{(}\PYG{n}{margin150df}\PYG{o}{.}\PYG{n}{select\PYGZus{}dtypes}\PYG{p}{(}\PYG{n}{np}\PYG{o}{.}\PYG{n}{number}\PYG{p}{)}\PYG{o}{.}\PYG{n}{gt}\PYG{p}{(}\PYG{l+m+mi}{0}\PYG{p}{)}\PYG{o}{.}\PYG{n}{sum}\PYG{p}{(}\PYG{n}{axis}\PYG{o}{=}\PYG{l+m+mi}{1}\PYG{p}{)}\PYG{p}{)}
\end{sphinxVerbatim}

\begin{sphinxVerbatim}[commandchars=\\\{\}]
62
\end{sphinxVerbatim}

\sphinxAtStartPar
There is a much higher \% of people making profit in the 150 trial experiment (63\%). There is far less people taking part in this experiment than the 100 trial experiment but it may be worth drilling down further into the data to check the differences between the different groups undertaking the experiments here. It might be interesting to see if there is any trends surrounding the age profiles surveyed. We will now try to combine the study groups with our “margin” results.


\section{Adding in Study undertakers to participant data}
\label{\detokenize{data-analysis:adding-in-study-undertakers-to-participant-data}}
\begin{sphinxVerbatim}[commandchars=\\\{\}]
\PYG{n}{margin95df}\PYG{p}{[}\PYG{l+s+s1}{\PYGZsq{}}\PYG{l+s+s1}{Study}\PYG{l+s+s1}{\PYGZsq{}}\PYG{p}{]} \PYG{o}{=} \PYG{n}{index95}\PYG{p}{[}\PYG{l+s+s1}{\PYGZsq{}}\PYG{l+s+s1}{Study}\PYG{l+s+s1}{\PYGZsq{}}\PYG{p}{]}\PYG{o}{.}\PYG{n}{values}
\end{sphinxVerbatim}

\begin{sphinxVerbatim}[commandchars=\\\{\}]
\PYG{n}{margin150df}\PYG{p}{[}\PYG{l+s+s1}{\PYGZsq{}}\PYG{l+s+s1}{Study}\PYG{l+s+s1}{\PYGZsq{}}\PYG{p}{]} \PYG{o}{=} \PYG{n}{index150}\PYG{p}{[}\PYG{l+s+s1}{\PYGZsq{}}\PYG{l+s+s1}{Study}\PYG{l+s+s1}{\PYGZsq{}}\PYG{p}{]}\PYG{o}{.}\PYG{n}{values}
\end{sphinxVerbatim}

\begin{sphinxVerbatim}[commandchars=\\\{\}]
\PYG{n}{margin100df}\PYG{p}{[}\PYG{l+s+s1}{\PYGZsq{}}\PYG{l+s+s1}{Study}\PYG{l+s+s1}{\PYGZsq{}}\PYG{p}{]} \PYG{o}{=} \PYG{n}{index100}\PYG{p}{[}\PYG{l+s+s1}{\PYGZsq{}}\PYG{l+s+s1}{Study}\PYG{l+s+s1}{\PYGZsq{}}\PYG{p}{]}\PYG{o}{.}\PYG{n}{values}
\end{sphinxVerbatim}

\sphinxAtStartPar
Here we will investigate the Study groups in the 150 person experiment. We will try to see does one group achieve better results in the task than the other.

\begin{sphinxVerbatim}[commandchars=\\\{\}]
\PYG{n}{margin150df}\PYG{p}{[}\PYG{l+s+s1}{\PYGZsq{}}\PYG{l+s+s1}{Study}\PYG{l+s+s1}{\PYGZsq{}}\PYG{p}{]}\PYG{o}{.}\PYG{n}{value\PYGZus{}counts}\PYG{p}{(}\PYG{p}{)}
\end{sphinxVerbatim}

\begin{sphinxVerbatim}[commandchars=\\\{\}]
Steingroever2011    57
Wetzels             41
Name: Study, dtype: int64
\end{sphinxVerbatim}

\begin{sphinxVerbatim}[commandchars=\\\{\}]
\PYG{n+nb}{print}\PYG{p}{(}\PYG{l+s+s2}{\PYGZdq{}}\PYG{l+s+s2}{Profit for the 150 trial Wetzels study}\PYG{l+s+s2}{\PYGZdq{}}\PYG{p}{)}
\PYG{n}{margin150df}\PYG{o}{.}\PYG{n}{loc}\PYG{p}{[}\PYG{n}{margin150df}\PYG{p}{[}\PYG{l+s+s1}{\PYGZsq{}}\PYG{l+s+s1}{Study}\PYG{l+s+s1}{\PYGZsq{}}\PYG{p}{]} \PYG{o}{==} \PYG{l+s+s1}{\PYGZsq{}}\PYG{l+s+s1}{Wetzels}\PYG{l+s+s1}{\PYGZsq{}}\PYG{p}{,} \PYG{l+s+s1}{\PYGZsq{}}\PYG{l+s+s1}{Margin}\PYG{l+s+s1}{\PYGZsq{}}\PYG{p}{]}\PYG{o}{.}\PYG{n}{sum}\PYG{p}{(}\PYG{p}{)}
\end{sphinxVerbatim}

\begin{sphinxVerbatim}[commandchars=\\\{\}]
Profit for the 150 trial Wetzels study
\end{sphinxVerbatim}

\begin{sphinxVerbatim}[commandchars=\\\{\}]
24000
\end{sphinxVerbatim}

\begin{sphinxVerbatim}[commandchars=\\\{\}]
\PYG{n+nb}{print}\PYG{p}{(}\PYG{l+s+s2}{\PYGZdq{}}\PYG{l+s+s2}{Profit for the 150 trial Steingroever2011 study}\PYG{l+s+s2}{\PYGZdq{}}\PYG{p}{)}
\PYG{n}{margin150df}\PYG{o}{.}\PYG{n}{loc}\PYG{p}{[}\PYG{n}{margin150df}\PYG{p}{[}\PYG{l+s+s1}{\PYGZsq{}}\PYG{l+s+s1}{Study}\PYG{l+s+s1}{\PYGZsq{}}\PYG{p}{]} \PYG{o}{==} \PYG{l+s+s1}{\PYGZsq{}}\PYG{l+s+s1}{Steingroever2011}\PYG{l+s+s1}{\PYGZsq{}}\PYG{p}{,} \PYG{l+s+s1}{\PYGZsq{}}\PYG{l+s+s1}{Margin}\PYG{l+s+s1}{\PYGZsq{}}\PYG{p}{]}\PYG{o}{.}\PYG{n}{sum}\PYG{p}{(}\PYG{p}{)}
\end{sphinxVerbatim}

\begin{sphinxVerbatim}[commandchars=\\\{\}]
Profit for the 150 trial Steingroever2011 study
\end{sphinxVerbatim}

\begin{sphinxVerbatim}[commandchars=\\\{\}]
12550
\end{sphinxVerbatim}


\section{Assessment of 150 margin results for each study}
\label{\detokenize{data-analysis:assessment-of-150-margin-results-for-each-study}}
\sphinxAtStartPar
This is interesting to note these values. Both studies make a large cumulative profit over the course of the 150 trials undertaken. There is a net profit between the two studies of 36,550 dollars, almost an average profit of 375 dollars per participant. It is interesting to note that the 41 students in “Wetzels” study were exclusively students, while in Steingroever2011’s study there was a young average age (19.9) but no specific mention of if the participants were students or not. Although they made money it was almost half of the other group studied. We will measure this against the other datasets to see if age is a factor between the decision making of the groups.

\begin{sphinxVerbatim}[commandchars=\\\{\}]
\PYG{n}{margin100df}\PYG{p}{[}\PYG{l+s+s1}{\PYGZsq{}}\PYG{l+s+s1}{Study}\PYG{l+s+s1}{\PYGZsq{}}\PYG{p}{]}\PYG{o}{.}\PYG{n}{value\PYGZus{}counts}\PYG{p}{(}\PYG{p}{)}
\end{sphinxVerbatim}

\begin{sphinxVerbatim}[commandchars=\\\{\}]
Horstmann            162
Wood                 153
SteingroverInPrep     70
Maia                  40
Worthy                35
Premkumar             25
Kjome                 19
Name: Study, dtype: int64
\end{sphinxVerbatim}

\begin{sphinxVerbatim}[commandchars=\\\{\}]
\PYG{n+nb}{print}\PYG{p}{(}\PYG{l+s+s2}{\PYGZdq{}}\PYG{l+s+s2}{Loss for the 100 trial Horstmann study}\PYG{l+s+s2}{\PYGZdq{}}\PYG{p}{)}
\PYG{n}{margin100df}\PYG{o}{.}\PYG{n}{loc}\PYG{p}{[}\PYG{n}{margin100df}\PYG{p}{[}\PYG{l+s+s1}{\PYGZsq{}}\PYG{l+s+s1}{Study}\PYG{l+s+s1}{\PYGZsq{}}\PYG{p}{]} \PYG{o}{==} \PYG{l+s+s1}{\PYGZsq{}}\PYG{l+s+s1}{Horstmann}\PYG{l+s+s1}{\PYGZsq{}}\PYG{p}{,} \PYG{l+s+s1}{\PYGZsq{}}\PYG{l+s+s1}{Margin}\PYG{l+s+s1}{\PYGZsq{}}\PYG{p}{]}\PYG{o}{.}\PYG{n}{sum}\PYG{p}{(}\PYG{p}{)}
\end{sphinxVerbatim}

\begin{sphinxVerbatim}[commandchars=\\\{\}]
Loss for the 100 trial Horstmann study
\end{sphinxVerbatim}

\begin{sphinxVerbatim}[commandchars=\\\{\}]
\PYGZhy{}6200
\end{sphinxVerbatim}

\begin{sphinxVerbatim}[commandchars=\\\{\}]
\PYG{n+nb}{print}\PYG{p}{(}\PYG{l+s+s2}{\PYGZdq{}}\PYG{l+s+s2}{Loss for the 100 trial Wood study}\PYG{l+s+s2}{\PYGZdq{}}\PYG{p}{)}
\PYG{n}{margin100df}\PYG{o}{.}\PYG{n}{loc}\PYG{p}{[}\PYG{n}{margin100df}\PYG{p}{[}\PYG{l+s+s1}{\PYGZsq{}}\PYG{l+s+s1}{Study}\PYG{l+s+s1}{\PYGZsq{}}\PYG{p}{]} \PYG{o}{==} \PYG{l+s+s1}{\PYGZsq{}}\PYG{l+s+s1}{Wood}\PYG{l+s+s1}{\PYGZsq{}}\PYG{p}{,} \PYG{l+s+s1}{\PYGZsq{}}\PYG{l+s+s1}{Margin}\PYG{l+s+s1}{\PYGZsq{}}\PYG{p}{]}\PYG{o}{.}\PYG{n}{sum}\PYG{p}{(}\PYG{p}{)}
\end{sphinxVerbatim}

\begin{sphinxVerbatim}[commandchars=\\\{\}]
Loss for the 100 trial Wood study
\end{sphinxVerbatim}

\begin{sphinxVerbatim}[commandchars=\\\{\}]
\PYGZhy{}119410
\end{sphinxVerbatim}

\begin{sphinxVerbatim}[commandchars=\\\{\}]
\PYG{n+nb}{print}\PYG{p}{(}\PYG{l+s+s2}{\PYGZdq{}}\PYG{l+s+s2}{Loss for the 100 trial SteingroverInPrep study}\PYG{l+s+s2}{\PYGZdq{}}\PYG{p}{)}
\PYG{n}{margin100df}\PYG{o}{.}\PYG{n}{loc}\PYG{p}{[}\PYG{n}{margin100df}\PYG{p}{[}\PYG{l+s+s1}{\PYGZsq{}}\PYG{l+s+s1}{Study}\PYG{l+s+s1}{\PYGZsq{}}\PYG{p}{]} \PYG{o}{==} \PYG{l+s+s1}{\PYGZsq{}}\PYG{l+s+s1}{SteingroverInPrep}\PYG{l+s+s1}{\PYGZsq{}}\PYG{p}{,} \PYG{l+s+s1}{\PYGZsq{}}\PYG{l+s+s1}{Margin}\PYG{l+s+s1}{\PYGZsq{}}\PYG{p}{]}\PYG{o}{.}\PYG{n}{sum}\PYG{p}{(}\PYG{p}{)}
\end{sphinxVerbatim}

\begin{sphinxVerbatim}[commandchars=\\\{\}]
Loss for the 100 trial SteingroverInPrep study
\end{sphinxVerbatim}

\begin{sphinxVerbatim}[commandchars=\\\{\}]
\PYGZhy{}4700
\end{sphinxVerbatim}

\begin{sphinxVerbatim}[commandchars=\\\{\}]
\PYG{n+nb}{print}\PYG{p}{(}\PYG{l+s+s2}{\PYGZdq{}}\PYG{l+s+s2}{Profit for the 100 trial Maia study}\PYG{l+s+s2}{\PYGZdq{}}\PYG{p}{)}
\PYG{n}{margin100df}\PYG{o}{.}\PYG{n}{loc}\PYG{p}{[}\PYG{n}{margin100df}\PYG{p}{[}\PYG{l+s+s1}{\PYGZsq{}}\PYG{l+s+s1}{Study}\PYG{l+s+s1}{\PYGZsq{}}\PYG{p}{]} \PYG{o}{==} \PYG{l+s+s1}{\PYGZsq{}}\PYG{l+s+s1}{Maia}\PYG{l+s+s1}{\PYGZsq{}}\PYG{p}{,} \PYG{l+s+s1}{\PYGZsq{}}\PYG{l+s+s1}{Margin}\PYG{l+s+s1}{\PYGZsq{}}\PYG{p}{]}\PYG{o}{.}\PYG{n}{sum}\PYG{p}{(}\PYG{p}{)}
\end{sphinxVerbatim}

\begin{sphinxVerbatim}[commandchars=\\\{\}]
Profit for the 100 trial Maia study
\end{sphinxVerbatim}

\begin{sphinxVerbatim}[commandchars=\\\{\}]
13600
\end{sphinxVerbatim}

\begin{sphinxVerbatim}[commandchars=\\\{\}]
\PYG{n+nb}{print}\PYG{p}{(}\PYG{l+s+s2}{\PYGZdq{}}\PYG{l+s+s2}{Loss for the 100 trial Worthy study}\PYG{l+s+s2}{\PYGZdq{}}\PYG{p}{)}
\PYG{n}{margin100df}\PYG{o}{.}\PYG{n}{loc}\PYG{p}{[}\PYG{n}{margin100df}\PYG{p}{[}\PYG{l+s+s1}{\PYGZsq{}}\PYG{l+s+s1}{Study}\PYG{l+s+s1}{\PYGZsq{}}\PYG{p}{]} \PYG{o}{==} \PYG{l+s+s1}{\PYGZsq{}}\PYG{l+s+s1}{Worthy}\PYG{l+s+s1}{\PYGZsq{}}\PYG{p}{,} \PYG{l+s+s1}{\PYGZsq{}}\PYG{l+s+s1}{Margin}\PYG{l+s+s1}{\PYGZsq{}}\PYG{p}{]}\PYG{o}{.}\PYG{n}{sum}\PYG{p}{(}\PYG{p}{)}
\end{sphinxVerbatim}

\begin{sphinxVerbatim}[commandchars=\\\{\}]
Loss for the 100 trial Worthy study
\end{sphinxVerbatim}

\begin{sphinxVerbatim}[commandchars=\\\{\}]
\PYGZhy{}15100
\end{sphinxVerbatim}

\begin{sphinxVerbatim}[commandchars=\\\{\}]
\PYG{n+nb}{print}\PYG{p}{(}\PYG{l+s+s2}{\PYGZdq{}}\PYG{l+s+s2}{Profit for the 100 trial Premkumar study}\PYG{l+s+s2}{\PYGZdq{}}\PYG{p}{)}
\PYG{n}{margin100df}\PYG{o}{.}\PYG{n}{loc}\PYG{p}{[}\PYG{n}{margin100df}\PYG{p}{[}\PYG{l+s+s1}{\PYGZsq{}}\PYG{l+s+s1}{Study}\PYG{l+s+s1}{\PYGZsq{}}\PYG{p}{]} \PYG{o}{==} \PYG{l+s+s1}{\PYGZsq{}}\PYG{l+s+s1}{Premkumar}\PYG{l+s+s1}{\PYGZsq{}}\PYG{p}{,} \PYG{l+s+s1}{\PYGZsq{}}\PYG{l+s+s1}{Margin}\PYG{l+s+s1}{\PYGZsq{}}\PYG{p}{]}\PYG{o}{.}\PYG{n}{sum}\PYG{p}{(}\PYG{p}{)}
\end{sphinxVerbatim}

\begin{sphinxVerbatim}[commandchars=\\\{\}]
Profit for the 100 trial Premkumar study
\end{sphinxVerbatim}

\begin{sphinxVerbatim}[commandchars=\\\{\}]
5995
\end{sphinxVerbatim}

\begin{sphinxVerbatim}[commandchars=\\\{\}]
\PYG{n+nb}{print}\PYG{p}{(}\PYG{l+s+s2}{\PYGZdq{}}\PYG{l+s+s2}{Loss for the 100 trial Kjome study}\PYG{l+s+s2}{\PYGZdq{}}\PYG{p}{)}
\PYG{n}{margin100df}\PYG{o}{.}\PYG{n}{loc}\PYG{p}{[}\PYG{n}{margin100df}\PYG{p}{[}\PYG{l+s+s1}{\PYGZsq{}}\PYG{l+s+s1}{Study}\PYG{l+s+s1}{\PYGZsq{}}\PYG{p}{]} \PYG{o}{==} \PYG{l+s+s1}{\PYGZsq{}}\PYG{l+s+s1}{Kjome}\PYG{l+s+s1}{\PYGZsq{}}\PYG{p}{,} \PYG{l+s+s1}{\PYGZsq{}}\PYG{l+s+s1}{Margin}\PYG{l+s+s1}{\PYGZsq{}}\PYG{p}{]}\PYG{o}{.}\PYG{n}{sum}\PYG{p}{(}\PYG{p}{)}
\end{sphinxVerbatim}

\begin{sphinxVerbatim}[commandchars=\\\{\}]
Loss for the 100 trial Kjome study
\end{sphinxVerbatim}

\begin{sphinxVerbatim}[commandchars=\\\{\}]
\PYGZhy{}8750
\end{sphinxVerbatim}


\section{Assessment of margin for 100 trial experiments}
\label{\detokenize{data-analysis:assessment-of-margin-for-100-trial-experiments}}
\sphinxAtStartPar
The results here are in stark contrast to the 150 trial experiments. Despite less trials there is some significant losses accumalated by participants. Although the study conducted by Wood has a large number of participants in 153, the loss of 119,410 is certainly a major outlier. This equates to an average loss of roughly 780 dollars per person. It is particularly interesting to note \sphinxhref{https://openpsychologydata.metajnl.com/articles/10.5334/jopd.ak/}{here} in table 1, we see this group has the oldest average age of any group in the study by some distance. The next oldest average age specified actually makes a profit (Premkumar). Again we see students with strong results in the Maia study as undergraduate students here make a strong profit, similar to the groups in the 150 trial experiments. We now check the 95 trial study as our last part of our margin analysis.

\begin{sphinxVerbatim}[commandchars=\\\{\}]
\PYG{n}{margin95df}\PYG{o}{.}\PYG{n}{head}\PYG{p}{(}\PYG{p}{)}
\end{sphinxVerbatim}

\begin{sphinxVerbatim}[commandchars=\\\{\}]
        Margin     Study
Subj\PYGZus{}1    1150  Fridberg
Subj\PYGZus{}2    \PYGZhy{}675  Fridberg
Subj\PYGZus{}3    \PYGZhy{}750  Fridberg
Subj\PYGZus{}4    \PYGZhy{}525  Fridberg
Subj\PYGZus{}5     100  Fridberg
\end{sphinxVerbatim}

\begin{sphinxVerbatim}[commandchars=\\\{\}]
\PYG{n+nb}{print}\PYG{p}{(}\PYG{l+s+s2}{\PYGZdq{}}\PYG{l+s+s2}{Profit for the 95 trial Fridberg study}\PYG{l+s+s2}{\PYGZdq{}}\PYG{p}{)}
\PYG{n}{margin95df}\PYG{o}{.}\PYG{n}{loc}\PYG{p}{[}\PYG{n}{margin95df}\PYG{p}{[}\PYG{l+s+s1}{\PYGZsq{}}\PYG{l+s+s1}{Study}\PYG{l+s+s1}{\PYGZsq{}}\PYG{p}{]} \PYG{o}{==} \PYG{l+s+s1}{\PYGZsq{}}\PYG{l+s+s1}{Fridberg}\PYG{l+s+s1}{\PYGZsq{}}\PYG{p}{,} \PYG{l+s+s1}{\PYGZsq{}}\PYG{l+s+s1}{Margin}\PYG{l+s+s1}{\PYGZsq{}}\PYG{p}{]}\PYG{o}{.}\PYG{n}{sum}\PYG{p}{(}\PYG{p}{)}
\end{sphinxVerbatim}

\begin{sphinxVerbatim}[commandchars=\\\{\}]
Profit for the 95 trial Fridberg study
\end{sphinxVerbatim}

\begin{sphinxVerbatim}[commandchars=\\\{\}]
1250
\end{sphinxVerbatim}


\section{Assessment of margin for the 95 trial study}
\label{\detokenize{data-analysis:assessment-of-margin-for-the-95-trial-study}}
\sphinxAtStartPar
In this trial as mentioned previously less than half of the participants made profit, but the 15 participant group made 1,250 over the course of the task. This obviously points to some bigger wins mitigating a collection of smaller losses in the group. The group who took part in this study were slightly older than some of the groups who made bigger profits (mean age of 29.6 years old). One thing consistent through this analysis of profit/loss margins has been student groups making more money than older groups. Age appears to be a factor but it may be interesting to look at the flow of each study too. By this, we mean looking at how participants profit/loss fluctuated on each turn and see did a series of wins lead them to change strategy and go for broke for example? Or did a series of losses at the start of the game potentially set the tone for some participants? To do this we will need to combine win and loss dataframes together and graph our results accordingly.


\section{Analysis of participants selection flow}
\label{\detokenize{data-analysis:analysis-of-participants-selection-flow}}
\sphinxAtStartPar
We will start by looking at the participants in the 95 trial experiment. To merge our dataframes together we will need to change the column names so that they share common names.

\begin{sphinxVerbatim}[commandchars=\\\{\}]
\PYG{n}{columnnames95} \PYG{o}{=} \PYG{p}{[}\PYG{l+s+sa}{f}\PYG{l+s+s1}{\PYGZsq{}}\PYG{l+s+s1}{Trial}\PYG{l+s+si}{\PYGZob{}}\PYG{n}{num}\PYG{l+s+si}{\PYGZcb{}}\PYG{l+s+s1}{\PYGZsq{}} \PYG{k}{for} \PYG{n}{num} \PYG{o+ow}{in} \PYG{n+nb}{range}\PYG{p}{(}\PYG{l+m+mi}{1}\PYG{p}{,}\PYG{l+m+mi}{96}\PYG{p}{)}\PYG{p}{]}
\PYG{n}{wins95test} \PYG{o}{=} \PYG{n}{win95}
\PYG{n}{wins95test} \PYG{o}{=} \PYG{n}{wins95test}\PYG{o}{.}\PYG{n}{set\PYGZus{}axis}\PYG{p}{(}\PYG{n}{columnnames95}\PYG{p}{,} \PYG{n}{axis}\PYG{o}{=}\PYG{l+m+mi}{1}\PYG{p}{)}
\end{sphinxVerbatim}

\begin{sphinxVerbatim}[commandchars=\\\{\}]
\PYG{n}{loss95test} \PYG{o}{=} \PYG{n}{loss95}
\PYG{n}{loss95test} \PYG{o}{=} \PYG{n}{loss95test}\PYG{o}{.}\PYG{n}{set\PYGZus{}axis}\PYG{p}{(}\PYG{n}{columnnames95}\PYG{p}{,} \PYG{n}{axis}\PYG{o}{=}\PYG{l+m+mi}{1}\PYG{p}{)}
\end{sphinxVerbatim}

\begin{sphinxVerbatim}[commandchars=\\\{\}]
\PYG{n}{df95\PYGZus{}added} \PYG{o}{=} \PYG{n}{wins95test}\PYG{o}{.}\PYG{n}{add}\PYG{p}{(}\PYG{n}{loss95test}\PYG{p}{,} \PYG{n}{fill\PYGZus{}value}\PYG{o}{=}\PYG{l+m+mi}{0}\PYG{p}{)}
\end{sphinxVerbatim}

\sphinxAtStartPar
This study was all part of the Fridberg study so we don’t need to worry about comparing other studies here.

\begin{sphinxVerbatim}[commandchars=\\\{\}]
\PYG{n}{per\PYGZus{}trial\PYGZus{}95} \PYG{o}{=} \PYG{n}{df95\PYGZus{}added}\PYG{o}{.}\PYG{n}{sum}\PYG{p}{(}\PYG{n}{axis}\PYG{o}{=}\PYG{l+m+mi}{0}\PYG{p}{)}
\end{sphinxVerbatim}

\begin{sphinxVerbatim}[commandchars=\\\{\}]
\PYG{n}{per\PYGZus{}trial\PYGZus{}95}\PYG{o}{.}\PYG{n}{plot}\PYG{p}{(}\PYG{n}{title}\PYG{o}{=}\PYG{l+s+s1}{\PYGZsq{}}\PYG{l+s+s1}{Fridberg Study Win/Loss per round}\PYG{l+s+s1}{\PYGZsq{}}\PYG{p}{,} \PYG{n}{color}\PYG{o}{=}\PYG{l+s+s1}{\PYGZsq{}}\PYG{l+s+s1}{green}\PYG{l+s+s1}{\PYGZsq{}}\PYG{p}{)}
\end{sphinxVerbatim}

\begin{sphinxVerbatim}[commandchars=\\\{\}]
\PYGZlt{}AxesSubplot:title=\PYGZob{}\PYGZsq{}center\PYGZsq{}:\PYGZsq{}Fridberg Study Win/Loss per round\PYGZsq{}\PYGZcb{}\PYGZgt{}
\end{sphinxVerbatim}

\noindent\sphinxincludegraphics{{data-analysis_54_1}.png}


\subsection{Next we move onto the 150 trial experiments and see how these studies flowed over the course of their trials}
\label{\detokenize{data-analysis:next-we-move-onto-the-150-trial-experiments-and-see-how-these-studies-flowed-over-the-course-of-their-trials}}
\begin{sphinxVerbatim}[commandchars=\\\{\}]
\PYG{n}{columnnames150} \PYG{o}{=} \PYG{p}{[}\PYG{l+s+sa}{f}\PYG{l+s+s1}{\PYGZsq{}}\PYG{l+s+s1}{Trial}\PYG{l+s+si}{\PYGZob{}}\PYG{n}{num}\PYG{l+s+si}{\PYGZcb{}}\PYG{l+s+s1}{\PYGZsq{}} \PYG{k}{for} \PYG{n}{num} \PYG{o+ow}{in} \PYG{n+nb}{range}\PYG{p}{(}\PYG{l+m+mi}{1}\PYG{p}{,}\PYG{l+m+mi}{151}\PYG{p}{)}\PYG{p}{]}
\PYG{c+c1}{\PYGZsh{}columnnames95}
\PYG{n}{wins150test} \PYG{o}{=} \PYG{n}{win150}
\PYG{c+c1}{\PYGZsh{}wins95test.head()}
\PYG{n}{wins150test} \PYG{o}{=} \PYG{n}{wins150test}\PYG{o}{.}\PYG{n}{set\PYGZus{}axis}\PYG{p}{(}\PYG{n}{columnnames150}\PYG{p}{,} \PYG{n}{axis}\PYG{o}{=}\PYG{l+m+mi}{1}\PYG{p}{)}
\end{sphinxVerbatim}

\begin{sphinxVerbatim}[commandchars=\\\{\}]
\PYG{n}{loss150test} \PYG{o}{=} \PYG{n}{loss150}
\PYG{n}{loss150test} \PYG{o}{=} \PYG{n}{loss150test}\PYG{o}{.}\PYG{n}{set\PYGZus{}axis}\PYG{p}{(}\PYG{n}{columnnames150}\PYG{p}{,} \PYG{n}{axis}\PYG{o}{=}\PYG{l+m+mi}{1}\PYG{p}{)}
\end{sphinxVerbatim}

\begin{sphinxVerbatim}[commandchars=\\\{\}]
\PYG{n}{loss150test}\PYG{o}{.}\PYG{n}{head}\PYG{p}{(}\PYG{p}{)}
\end{sphinxVerbatim}

\begin{sphinxVerbatim}[commandchars=\\\{\}]
        Trial1  Trial2  Trial3  Trial4  Trial5  Trial6  Trial7  Trial8  \PYGZbs{}
Subj\PYGZus{}1    \PYGZhy{}250       0    \PYGZhy{}350       0       0    \PYGZhy{}200       0       0   
Subj\PYGZus{}2    \PYGZhy{}250    \PYGZhy{}350       0       0       0       0       0       0   
Subj\PYGZus{}3       0       0       0       0    \PYGZhy{}150   \PYGZhy{}1250       0     \PYGZhy{}50   
Subj\PYGZus{}4       0       0       0       0    \PYGZhy{}150       0       0     \PYGZhy{}50   
Subj\PYGZus{}5       0       0       0       0       0       0    \PYGZhy{}250       0   

        Trial9  Trial10  ...  Trial141  Trial142  Trial143  Trial144  \PYGZbs{}
Subj\PYGZus{}1       0        0  ...         0      \PYGZhy{}250         0     \PYGZhy{}1250   
Subj\PYGZus{}2       0        0  ...         0         0         0     \PYGZhy{}1250   
Subj\PYGZus{}3    \PYGZhy{}350        0  ...         0         0         0         0   
Subj\PYGZus{}4    \PYGZhy{}250        0  ...         0         0         0         0   
Subj\PYGZus{}5       0        0  ...         0         0         0         0   

        Trial145  Trial146  Trial147  Trial148  Trial149  Trial150  
Subj\PYGZus{}1         0         0       \PYGZhy{}50         0         0      \PYGZhy{}150  
Subj\PYGZus{}2         0      \PYGZhy{}250      \PYGZhy{}300         0       \PYGZhy{}50         0  
Subj\PYGZus{}3         0         0         0         0         0         0  
Subj\PYGZus{}4      \PYGZhy{}250         0         0         0         0         0  
Subj\PYGZus{}5         0         0         0         0         0      \PYGZhy{}250  

[5 rows x 150 columns]
\end{sphinxVerbatim}

\begin{sphinxVerbatim}[commandchars=\\\{\}]
\PYG{n}{df150\PYGZus{}added} \PYG{o}{=} \PYG{n}{wins150test}\PYG{o}{.}\PYG{n}{add}\PYG{p}{(}\PYG{n}{loss150test}\PYG{p}{,} \PYG{n}{fill\PYGZus{}value}\PYG{o}{=}\PYG{l+m+mi}{0}\PYG{p}{)}
\end{sphinxVerbatim}

\begin{sphinxVerbatim}[commandchars=\\\{\}]
\PYG{n}{df150\PYGZus{}added}\PYG{p}{[}\PYG{l+s+s1}{\PYGZsq{}}\PYG{l+s+s1}{Study}\PYG{l+s+s1}{\PYGZsq{}}\PYG{p}{]} \PYG{o}{=} \PYG{n}{index150}\PYG{p}{[}\PYG{l+s+s1}{\PYGZsq{}}\PYG{l+s+s1}{Study}\PYG{l+s+s1}{\PYGZsq{}}\PYG{p}{]}\PYG{o}{.}\PYG{n}{values}
\end{sphinxVerbatim}

\begin{sphinxVerbatim}[commandchars=\\\{\}]
\PYG{n}{teststein} \PYG{o}{=} \PYG{n}{df150\PYGZus{}added}\PYG{o}{.}\PYG{n}{loc}\PYG{p}{[}\PYG{n}{df150\PYGZus{}added}\PYG{p}{[}\PYG{l+s+s1}{\PYGZsq{}}\PYG{l+s+s1}{Study}\PYG{l+s+s1}{\PYGZsq{}}\PYG{p}{]} \PYG{o}{==} \PYG{l+s+s1}{\PYGZsq{}}\PYG{l+s+s1}{Steingroever2011}\PYG{l+s+s1}{\PYGZsq{}}\PYG{p}{]}
\PYG{k}{del} \PYG{n}{teststein}\PYG{p}{[}\PYG{l+s+s1}{\PYGZsq{}}\PYG{l+s+s1}{Study}\PYG{l+s+s1}{\PYGZsq{}}\PYG{p}{]}
\PYG{n}{per\PYGZus{}trial\PYGZus{}150\PYGZus{}stein} \PYG{o}{=} \PYG{n}{teststein}\PYG{o}{.}\PYG{n}{sum}\PYG{p}{(}\PYG{n}{axis}\PYG{o}{=}\PYG{l+m+mi}{0}\PYG{p}{)}
\end{sphinxVerbatim}

\begin{sphinxVerbatim}[commandchars=\\\{\}]
\PYG{n}{per\PYGZus{}trial\PYGZus{}150\PYGZus{}stein}\PYG{o}{.}\PYG{n}{plot}\PYG{p}{(}\PYG{n}{title}\PYG{o}{=}\PYG{l+s+s2}{\PYGZdq{}}\PYG{l+s+s2}{Winnings per round for Steingroever2011}\PYG{l+s+s2}{\PYGZsq{}}\PYG{l+s+s2}{s study}\PYG{l+s+s2}{\PYGZdq{}}\PYG{p}{,} \PYG{n}{color}\PYG{o}{=}\PYG{l+s+s2}{\PYGZdq{}}\PYG{l+s+s2}{green}\PYG{l+s+s2}{\PYGZdq{}}\PYG{p}{)}
\end{sphinxVerbatim}

\begin{sphinxVerbatim}[commandchars=\\\{\}]
\PYGZlt{}AxesSubplot:title=\PYGZob{}\PYGZsq{}center\PYGZsq{}:\PYGZdq{}Winnings per round for Steingroever2011\PYGZsq{}s study\PYGZdq{}\PYGZcb{}\PYGZgt{}
\end{sphinxVerbatim}

\noindent\sphinxincludegraphics{{data-analysis_62_1}.png}

\begin{sphinxVerbatim}[commandchars=\\\{\}]
\PYG{n}{testWetzels} \PYG{o}{=} \PYG{n}{df150\PYGZus{}added}\PYG{o}{.}\PYG{n}{loc}\PYG{p}{[}\PYG{n}{df150\PYGZus{}added}\PYG{p}{[}\PYG{l+s+s1}{\PYGZsq{}}\PYG{l+s+s1}{Study}\PYG{l+s+s1}{\PYGZsq{}}\PYG{p}{]} \PYG{o}{==} \PYG{l+s+s1}{\PYGZsq{}}\PYG{l+s+s1}{Wetzels}\PYG{l+s+s1}{\PYGZsq{}}\PYG{p}{]}
\PYG{k}{del} \PYG{n}{testWetzels}\PYG{p}{[}\PYG{l+s+s1}{\PYGZsq{}}\PYG{l+s+s1}{Study}\PYG{l+s+s1}{\PYGZsq{}}\PYG{p}{]}
\PYG{n}{per\PYGZus{}trial\PYGZus{}150\PYGZus{}wetzels} \PYG{o}{=} \PYG{n}{testWetzels}\PYG{o}{.}\PYG{n}{sum}\PYG{p}{(}\PYG{n}{axis}\PYG{o}{=}\PYG{l+m+mi}{0}\PYG{p}{)}
\end{sphinxVerbatim}

\begin{sphinxVerbatim}[commandchars=\\\{\}]
\PYG{n}{per\PYGZus{}trial\PYGZus{}150\PYGZus{}wetzels}\PYG{o}{.}\PYG{n}{plot}\PYG{p}{(}\PYG{n}{title}\PYG{o}{=}\PYG{l+s+s2}{\PYGZdq{}}\PYG{l+s+s2}{Winnings per round for Wetzels study}\PYG{l+s+s2}{\PYGZdq{}}\PYG{p}{,} \PYG{n}{color}\PYG{o}{=}\PYG{l+s+s2}{\PYGZdq{}}\PYG{l+s+s2}{green}\PYG{l+s+s2}{\PYGZdq{}}\PYG{p}{)}
\end{sphinxVerbatim}

\begin{sphinxVerbatim}[commandchars=\\\{\}]
\PYGZlt{}AxesSubplot:title=\PYGZob{}\PYGZsq{}center\PYGZsq{}:\PYGZsq{}Winnings per round for Wetzels study\PYGZsq{}\PYGZcb{}\PYGZgt{}
\end{sphinxVerbatim}

\noindent\sphinxincludegraphics{{data-analysis_64_1}.png}

\sphinxAtStartPar
After assessing the participants winnings in each study and how their respective win / losses flowed per trial I decided to delve deeper into why these studies appeared more profitable than others. It is easy to point to the varying amounts of cards that pay out on each study and the number of respective trials each study undertook or the number of participants each study had. I felt it would be interesting to build on the win / loss flow per trials mentioned just above and try to pick out patterns accordingly. This would be something like if studies lost a lot of money is this down to a lower average choice of deck or constant changing of choosen cards and see if we could link this to research and studies on the IGT. This study and research could be related to gender based decision making or age demographies and see if these related studies findings hold true to our sample data.


\chapter{2. Data preparation for Clustering}
\label{\detokenize{data-processing:data-preparation-for-clustering}}\label{\detokenize{data-processing::doc}}
\sphinxAtStartPar
For our clustering analysis we need to prepare the data accordingly. Following on from our data analysis we want to try to cluster on the profit margin of participants against the number of times the subjects picked their most common deck choice or their average choice. We will then combine this with a scatter plot showing the study each subject was a part of and see what information we can gather from this. We will be looking at age demographies more so but also look to combine this with the amount of cards that pay out in each study and gender breakdowns also. To do this we need to create appropriate CSV files that we can then use for clustering.

\begin{sphinxVerbatim}[commandchars=\\\{\}]
\PYG{k+kn}{import} \PYG{n+nn}{pandas} \PYG{k}{as} \PYG{n+nn}{pd}
\PYG{k+kn}{import} \PYG{n+nn}{seaborn} \PYG{k}{as} \PYG{n+nn}{sn}
\PYG{k+kn}{import} \PYG{n+nn}{numpy} \PYG{k}{as} \PYG{n+nn}{np}
\PYG{k+kn}{import} \PYG{n+nn}{matplotlib}\PYG{n+nn}{.}\PYG{n+nn}{pyplot} \PYG{k}{as} \PYG{n+nn}{plt}
\PYG{k+kn}{from} \PYG{n+nn}{sklearn}\PYG{n+nn}{.}\PYG{n+nn}{cluster} \PYG{k+kn}{import} \PYG{n}{KMeans}\PYG{p}{,} \PYG{n}{AgglomerativeClustering}
\PYG{k+kn}{from} \PYG{n+nn}{sklearn}\PYG{n+nn}{.}\PYG{n+nn}{metrics} \PYG{k+kn}{import} \PYG{n}{silhouette\PYGZus{}score}
\PYG{k+kn}{from} \PYG{n+nn}{sklearn}\PYG{n+nn}{.}\PYG{n+nn}{preprocessing} \PYG{k+kn}{import} \PYG{n}{OneHotEncoder}
\PYG{k+kn}{from} \PYG{n+nn}{sklearn}\PYG{n+nn}{.}\PYG{n+nn}{preprocessing} \PYG{k+kn}{import} \PYG{n}{LabelEncoder}
\PYG{k+kn}{from} \PYG{n+nn}{sklearn} \PYG{k+kn}{import} \PYG{n}{preprocessing}
\end{sphinxVerbatim}

\begin{sphinxVerbatim}[commandchars=\\\{\}]
\PYG{n}{index95} \PYG{o}{=} \PYG{n}{pd}\PYG{o}{.}\PYG{n}{read\PYGZus{}csv}\PYG{p}{(}\PYG{l+s+s1}{\PYGZsq{}}\PYG{l+s+s1}{data/index\PYGZus{}95.csv}\PYG{l+s+s1}{\PYGZsq{}}\PYG{p}{)}
\PYG{n}{index100} \PYG{o}{=} \PYG{n}{pd}\PYG{o}{.}\PYG{n}{read\PYGZus{}csv}\PYG{p}{(}\PYG{l+s+s1}{\PYGZsq{}}\PYG{l+s+s1}{data/index\PYGZus{}100.csv}\PYG{l+s+s1}{\PYGZsq{}}\PYG{p}{)}
\PYG{n}{index150} \PYG{o}{=} \PYG{n}{pd}\PYG{o}{.}\PYG{n}{read\PYGZus{}csv}\PYG{p}{(}\PYG{l+s+s1}{\PYGZsq{}}\PYG{l+s+s1}{data/index\PYGZus{}150.csv}\PYG{l+s+s1}{\PYGZsq{}}\PYG{p}{)}
\PYG{n}{win95} \PYG{o}{=} \PYG{n}{pd}\PYG{o}{.}\PYG{n}{read\PYGZus{}csv}\PYG{p}{(}\PYG{l+s+s1}{\PYGZsq{}}\PYG{l+s+s1}{data/wi\PYGZus{}95.csv}\PYG{l+s+s1}{\PYGZsq{}}\PYG{p}{)}
\PYG{n}{win100} \PYG{o}{=} \PYG{n}{pd}\PYG{o}{.}\PYG{n}{read\PYGZus{}csv}\PYG{p}{(}\PYG{l+s+s1}{\PYGZsq{}}\PYG{l+s+s1}{data/wi\PYGZus{}100.csv}\PYG{l+s+s1}{\PYGZsq{}}\PYG{p}{)}
\PYG{n}{win150} \PYG{o}{=} \PYG{n}{pd}\PYG{o}{.}\PYG{n}{read\PYGZus{}csv}\PYG{p}{(}\PYG{l+s+s1}{\PYGZsq{}}\PYG{l+s+s1}{data/wi\PYGZus{}150.csv}\PYG{l+s+s1}{\PYGZsq{}}\PYG{p}{)}
\PYG{n}{loss95} \PYG{o}{=} \PYG{n}{pd}\PYG{o}{.}\PYG{n}{read\PYGZus{}csv}\PYG{p}{(}\PYG{l+s+s1}{\PYGZsq{}}\PYG{l+s+s1}{data/lo\PYGZus{}95.csv}\PYG{l+s+s1}{\PYGZsq{}}\PYG{p}{)}
\PYG{n}{loss100} \PYG{o}{=} \PYG{n}{pd}\PYG{o}{.}\PYG{n}{read\PYGZus{}csv}\PYG{p}{(}\PYG{l+s+s1}{\PYGZsq{}}\PYG{l+s+s1}{data/lo\PYGZus{}100.csv}\PYG{l+s+s1}{\PYGZsq{}}\PYG{p}{)}
\PYG{n}{loss150} \PYG{o}{=} \PYG{n}{pd}\PYG{o}{.}\PYG{n}{read\PYGZus{}csv}\PYG{p}{(}\PYG{l+s+s1}{\PYGZsq{}}\PYG{l+s+s1}{data/lo\PYGZus{}150.csv}\PYG{l+s+s1}{\PYGZsq{}}\PYG{p}{)}
\PYG{n}{choice95} \PYG{o}{=} \PYG{n}{pd}\PYG{o}{.}\PYG{n}{read\PYGZus{}csv}\PYG{p}{(}\PYG{l+s+s1}{\PYGZsq{}}\PYG{l+s+s1}{data/choice\PYGZus{}95.csv}\PYG{l+s+s1}{\PYGZsq{}}\PYG{p}{)}
\PYG{n}{choice100} \PYG{o}{=} \PYG{n}{pd}\PYG{o}{.}\PYG{n}{read\PYGZus{}csv}\PYG{p}{(}\PYG{l+s+s1}{\PYGZsq{}}\PYG{l+s+s1}{data/choice\PYGZus{}100.csv}\PYG{l+s+s1}{\PYGZsq{}}\PYG{p}{)}
\PYG{n}{choice150} \PYG{o}{=} \PYG{n}{pd}\PYG{o}{.}\PYG{n}{read\PYGZus{}csv}\PYG{p}{(}\PYG{l+s+s1}{\PYGZsq{}}\PYG{l+s+s1}{data/choice\PYGZus{}150.csv}\PYG{l+s+s1}{\PYGZsq{}}\PYG{p}{)}
\end{sphinxVerbatim}


\section{Creating margin csv files}
\label{\detokenize{data-processing:creating-margin-csv-files}}
\begin{sphinxVerbatim}[commandchars=\\\{\}]
\PYG{n}{columnnames95} \PYG{o}{=} \PYG{p}{[}\PYG{l+s+sa}{f}\PYG{l+s+s1}{\PYGZsq{}}\PYG{l+s+s1}{Trial}\PYG{l+s+si}{\PYGZob{}}\PYG{n}{num}\PYG{l+s+si}{\PYGZcb{}}\PYG{l+s+s1}{\PYGZsq{}} \PYG{k}{for} \PYG{n}{num} \PYG{o+ow}{in} \PYG{n+nb}{range}\PYG{p}{(}\PYG{l+m+mi}{1}\PYG{p}{,}\PYG{l+m+mi}{96}\PYG{p}{)}\PYG{p}{]}
\PYG{n}{wins95} \PYG{o}{=} \PYG{n}{win95}
\PYG{n}{wins95} \PYG{o}{=} \PYG{n}{wins95}\PYG{o}{.}\PYG{n}{set\PYGZus{}axis}\PYG{p}{(}\PYG{n}{columnnames95}\PYG{p}{,} \PYG{n}{axis}\PYG{o}{=}\PYG{l+m+mi}{1}\PYG{p}{)}
\PYG{n}{wins95}\PYG{o}{.}\PYG{n}{head}\PYG{p}{(}\PYG{p}{)}
\end{sphinxVerbatim}

\begin{sphinxVerbatim}[commandchars=\\\{\}]
        Trial1  Trial2  Trial3  Trial4  Trial5  Trial6  Trial7  Trial8  \PYGZbs{}
Subj\PYGZus{}1     100     100     100     100     100     100     100     100   
Subj\PYGZus{}2     100     100      50     100     100     100     100     100   
Subj\PYGZus{}3      50      50      50     100     100     100     100     100   
Subj\PYGZus{}4      50      50     100     100     100     100     100      50   
Subj\PYGZus{}5     100     100      50      50      50     100     100     100   

        Trial9  Trial10  ...  Trial86  Trial87  Trial88  Trial89  Trial90  \PYGZbs{}
Subj\PYGZus{}1     100      100  ...       50       50       50       50       50   
Subj\PYGZus{}2     100      100  ...       50      100      100      100      100   
Subj\PYGZus{}3     100      100  ...      100      100      100       50       50   
Subj\PYGZus{}4     100      100  ...      100       50       50       50       50   
Subj\PYGZus{}5     100      100  ...       50       50       50       50       50   

        Trial91  Trial92  Trial93  Trial94  Trial95  
Subj\PYGZus{}1       50       50       50       50       50  
Subj\PYGZus{}2      100       50       50       50       50  
Subj\PYGZus{}3       50       50       50       50       50  
Subj\PYGZus{}4       50       50       50       50       50  
Subj\PYGZus{}5       50       50       50       50       50  

[5 rows x 95 columns]
\end{sphinxVerbatim}

\begin{sphinxVerbatim}[commandchars=\\\{\}]
\PYG{n}{losses95} \PYG{o}{=} \PYG{n}{loss95}
\PYG{n}{losses95} \PYG{o}{=} \PYG{n}{losses95}\PYG{o}{.}\PYG{n}{set\PYGZus{}axis}\PYG{p}{(}\PYG{n}{columnnames95}\PYG{p}{,} \PYG{n}{axis}\PYG{o}{=}\PYG{l+m+mi}{1}\PYG{p}{)}
\PYG{n}{losses95}\PYG{o}{.}\PYG{n}{head}\PYG{p}{(}\PYG{p}{)}
\end{sphinxVerbatim}

\begin{sphinxVerbatim}[commandchars=\\\{\}]
        Trial1  Trial2  Trial3  Trial4  Trial5  Trial6  Trial7  Trial8  \PYGZbs{}
Subj\PYGZus{}1       0       0       0       0       0       0       0       0   
Subj\PYGZus{}2       0       0       0       0       0       0       0       0   
Subj\PYGZus{}3       0       0       0       0       0       0       0    \PYGZhy{}150   
Subj\PYGZus{}4       0       0       0       0    \PYGZhy{}150       0       0       0   
Subj\PYGZus{}5       0       0       0       0       0       0    \PYGZhy{}150       0   

        Trial9  Trial10  ...  Trial86  Trial87  Trial88  Trial89  Trial90  \PYGZbs{}
Subj\PYGZus{}1   \PYGZhy{}1250        0  ...        0        0        0        0        0   
Subj\PYGZus{}2       0        0  ...      \PYGZhy{}50     \PYGZhy{}300        0     \PYGZhy{}350        0   
Subj\PYGZus{}3       0        0  ...        0        0        0        0        0   
Subj\PYGZus{}4       0        0  ...        0      \PYGZhy{}50        0      \PYGZhy{}50      \PYGZhy{}50   
Subj\PYGZus{}5       0        0  ...      \PYGZhy{}75        0        0        0        0   

        Trial91  Trial92  Trial93  Trial94  Trial95  
Subj\PYGZus{}1        0        0     \PYGZhy{}250        0        0  
Subj\PYGZus{}2        0        0        0        0      \PYGZhy{}25  
Subj\PYGZus{}3        0     \PYGZhy{}250        0        0        0  
Subj\PYGZus{}4        0      \PYGZhy{}25        0        0        0  
Subj\PYGZus{}5        0        0        0        0        0  

[5 rows x 95 columns]
\end{sphinxVerbatim}

\begin{sphinxVerbatim}[commandchars=\\\{\}]
\PYG{n}{df95\PYGZus{}sum} \PYG{o}{=} \PYG{n}{wins95}\PYG{o}{.}\PYG{n}{add}\PYG{p}{(}\PYG{n}{losses95}\PYG{p}{,} \PYG{n}{fill\PYGZus{}value}\PYG{o}{=}\PYG{l+m+mi}{0}\PYG{p}{)}
\PYG{n}{df95\PYGZus{}sum}\PYG{o}{.}\PYG{n}{head}\PYG{p}{(}\PYG{p}{)}
\end{sphinxVerbatim}

\begin{sphinxVerbatim}[commandchars=\\\{\}]
        Trial1  Trial2  Trial3  Trial4  Trial5  Trial6  Trial7  Trial8  \PYGZbs{}
Subj\PYGZus{}1     100     100     100     100     100     100     100     100   
Subj\PYGZus{}2     100     100      50     100     100     100     100     100   
Subj\PYGZus{}3      50      50      50     100     100     100     100     \PYGZhy{}50   
Subj\PYGZus{}4      50      50     100     100     \PYGZhy{}50     100     100      50   
Subj\PYGZus{}5     100     100      50      50      50     100     \PYGZhy{}50     100   

        Trial9  Trial10  ...  Trial86  Trial87  Trial88  Trial89  Trial90  \PYGZbs{}
Subj\PYGZus{}1   \PYGZhy{}1150      100  ...       50       50       50       50       50   
Subj\PYGZus{}2     100      100  ...        0     \PYGZhy{}200      100     \PYGZhy{}250      100   
Subj\PYGZus{}3     100      100  ...      100      100      100       50       50   
Subj\PYGZus{}4     100      100  ...      100        0       50        0        0   
Subj\PYGZus{}5     100      100  ...      \PYGZhy{}25       50       50       50       50   

        Trial91  Trial92  Trial93  Trial94  Trial95  
Subj\PYGZus{}1       50       50     \PYGZhy{}200       50       50  
Subj\PYGZus{}2      100       50       50       50       25  
Subj\PYGZus{}3       50     \PYGZhy{}200       50       50       50  
Subj\PYGZus{}4       50       25       50       50       50  
Subj\PYGZus{}5       50       50       50       50       50  

[5 rows x 95 columns]
\end{sphinxVerbatim}

\begin{sphinxVerbatim}[commandchars=\\\{\}]
\PYG{n}{profit95} \PYG{o}{=} \PYG{n}{df95\PYGZus{}sum}\PYG{o}{.}\PYG{n}{sum}\PYG{p}{(}\PYG{n}{axis}\PYG{o}{=}\PYG{l+m+mi}{1}\PYG{p}{)}
\PYG{n}{profit95df} \PYG{o}{=} \PYG{n}{pd}\PYG{o}{.}\PYG{n}{DataFrame}\PYG{p}{(}\PYG{n}{data}\PYG{o}{=}\PYG{n}{profit95}\PYG{p}{)}
\PYG{n}{profit95df}\PYG{o}{.}\PYG{n}{rename}\PYG{p}{(}\PYG{n}{columns}\PYG{o}{=}\PYG{p}{\PYGZob{}}\PYG{l+m+mi}{0}\PYG{p}{:} \PYG{l+s+s1}{\PYGZsq{}}\PYG{l+s+s1}{Margin}\PYG{l+s+s1}{\PYGZsq{}}\PYG{p}{\PYGZcb{}}\PYG{p}{,} \PYG{n}{inplace}\PYG{o}{=}\PYG{k+kc}{True}\PYG{p}{)}
\PYG{n}{profit95df}\PYG{o}{.}\PYG{n}{head}\PYG{p}{(}\PYG{p}{)}
\end{sphinxVerbatim}

\begin{sphinxVerbatim}[commandchars=\\\{\}]
        Margin
Subj\PYGZus{}1    1150
Subj\PYGZus{}2    \PYGZhy{}675
Subj\PYGZus{}3    \PYGZhy{}750
Subj\PYGZus{}4    \PYGZhy{}525
Subj\PYGZus{}5     100
\end{sphinxVerbatim}

\begin{sphinxVerbatim}[commandchars=\\\{\}]
\PYG{n}{choice95}\PYG{o}{.}\PYG{n}{head}\PYG{p}{(}\PYG{p}{)}
\end{sphinxVerbatim}

\begin{sphinxVerbatim}[commandchars=\\\{\}]
        Choice\PYGZus{}1  Choice\PYGZus{}2  Choice\PYGZus{}3  Choice\PYGZus{}4  Choice\PYGZus{}5  Choice\PYGZus{}6  Choice\PYGZus{}7  \PYGZbs{}
Subj\PYGZus{}1         2         2         2         2         2         2         2   
Subj\PYGZus{}2         1         2         3         2         2         2         2   
Subj\PYGZus{}3         3         4         3         2         2         1         1   
Subj\PYGZus{}4         4         3         1         1         1         2         2   
Subj\PYGZus{}5         1         2         3         4         3         1         1   

        Choice\PYGZus{}8  Choice\PYGZus{}9  Choice\PYGZus{}10  ...  Choice\PYGZus{}86  Choice\PYGZus{}87  Choice\PYGZus{}88  \PYGZbs{}
Subj\PYGZus{}1         2         2          1  ...          4          4          4   
Subj\PYGZus{}2         2         2          2  ...          3          1          1   
Subj\PYGZus{}3         1         1          2  ...          2          2          2   
Subj\PYGZus{}4         3         2          2  ...          2          3          3   
Subj\PYGZus{}5         2         2          2  ...          3          3          4   

        Choice\PYGZus{}89  Choice\PYGZus{}90  Choice\PYGZus{}91  Choice\PYGZus{}92  Choice\PYGZus{}93  Choice\PYGZus{}94  \PYGZbs{}
Subj\PYGZus{}1          4          4          4          4          4          4   
Subj\PYGZus{}2          1          2          2          3          4          4   
Subj\PYGZus{}3          4          4          4          4          4          4   
Subj\PYGZus{}4          3          3          3          3          4          4   
Subj\PYGZus{}5          4          3          4          4          4          4   

        Choice\PYGZus{}95  
Subj\PYGZus{}1          4  
Subj\PYGZus{}2          3  
Subj\PYGZus{}3          4  
Subj\PYGZus{}4          4  
Subj\PYGZus{}5          4  

[5 rows x 95 columns]
\end{sphinxVerbatim}

\begin{sphinxVerbatim}[commandchars=\\\{\}]
\PYG{n}{most\PYGZus{}common\PYGZus{}count} \PYG{o}{=} \PYG{n}{choice95}\PYG{o}{.}\PYG{n}{apply}\PYG{p}{(}\PYG{n}{pd}\PYG{o}{.}\PYG{n}{Series}\PYG{o}{.}\PYG{n}{value\PYGZus{}counts}\PYG{p}{,} \PYG{n}{axis}\PYG{o}{=}\PYG{l+m+mi}{1}\PYG{p}{)}
\PYG{n}{most\PYGZus{}common\PYGZus{}count} \PYG{o}{=} \PYG{n}{most\PYGZus{}common\PYGZus{}count}\PYG{o}{.}\PYG{n}{max}\PYG{p}{(}\PYG{n}{axis}\PYG{o}{=}\PYG{l+m+mi}{1}\PYG{p}{)}
\PYG{n}{profit95df}\PYG{p}{[}\PYG{l+s+s1}{\PYGZsq{}}\PYG{l+s+s1}{Most Common Choice Picked}\PYG{l+s+s1}{\PYGZsq{}}\PYG{p}{]} \PYG{o}{=} \PYG{n}{most\PYGZus{}common\PYGZus{}count}
\PYG{n}{profit95df}\PYG{o}{.}\PYG{n}{head}\PYG{p}{(}\PYG{p}{)}
\end{sphinxVerbatim}

\begin{sphinxVerbatim}[commandchars=\\\{\}]
        Margin  Most Common Choice Picked
Subj\PYGZus{}1    1150                         71
Subj\PYGZus{}2    \PYGZhy{}675                         33
Subj\PYGZus{}3    \PYGZhy{}750                         38
Subj\PYGZus{}4    \PYGZhy{}525                         38
Subj\PYGZus{}5     100                         46
\end{sphinxVerbatim}

\begin{sphinxVerbatim}[commandchars=\\\{\}]
\PYG{n}{mode95} \PYG{o}{=} \PYG{n}{choice95}\PYG{o}{.}\PYG{n}{mode}\PYG{p}{(}\PYG{n}{axis}\PYG{o}{=}\PYG{l+m+mi}{1}\PYG{p}{)}
\PYG{n}{mode95}\PYG{o}{.}\PYG{n}{rename}\PYG{p}{(}\PYG{n}{columns}\PYG{o}{=}\PYG{p}{\PYGZob{}}\PYG{l+m+mi}{0}\PYG{p}{:} \PYG{l+s+s1}{\PYGZsq{}}\PYG{l+s+s1}{Most Common Choice}\PYG{l+s+s1}{\PYGZsq{}}\PYG{p}{\PYGZcb{}}\PYG{p}{,} \PYG{n}{inplace}\PYG{o}{=}\PYG{k+kc}{True}\PYG{p}{)}
\end{sphinxVerbatim}

\begin{sphinxVerbatim}[commandchars=\\\{\}]
\PYG{n}{profit95df}\PYG{p}{[}\PYG{l+s+s1}{\PYGZsq{}}\PYG{l+s+s1}{Most Common Choice}\PYG{l+s+s1}{\PYGZsq{}}\PYG{p}{]} \PYG{o}{=} \PYG{n}{mode95}\PYG{p}{[}\PYG{l+s+s1}{\PYGZsq{}}\PYG{l+s+s1}{Most Common Choice}\PYG{l+s+s1}{\PYGZsq{}}\PYG{p}{]}\PYG{o}{.}\PYG{n}{values}
\end{sphinxVerbatim}

\begin{sphinxVerbatim}[commandchars=\\\{\}]
\PYG{n}{profit95df}\PYG{p}{[}\PYG{l+s+s1}{\PYGZsq{}}\PYG{l+s+s1}{Study}\PYG{l+s+s1}{\PYGZsq{}}\PYG{p}{]} \PYG{o}{=} \PYG{n}{index95}\PYG{p}{[}\PYG{l+s+s1}{\PYGZsq{}}\PYG{l+s+s1}{Study}\PYG{l+s+s1}{\PYGZsq{}}\PYG{p}{]}\PYG{o}{.}\PYG{n}{values}
\PYG{n}{profit95df}\PYG{o}{.}\PYG{n}{head}\PYG{p}{(}\PYG{p}{)}
\end{sphinxVerbatim}

\begin{sphinxVerbatim}[commandchars=\\\{\}]
        Margin  Most Common Choice Picked  Most Common Choice     Study
Subj\PYGZus{}1    1150                         71                   4  Fridberg
Subj\PYGZus{}2    \PYGZhy{}675                         33                   4  Fridberg
Subj\PYGZus{}3    \PYGZhy{}750                         38                   4  Fridberg
Subj\PYGZus{}4    \PYGZhy{}525                         38                   4  Fridberg
Subj\PYGZus{}5     100                         46                   4  Fridberg
\end{sphinxVerbatim}

\begin{sphinxVerbatim}[commandchars=\\\{\}]
\PYG{n}{mean95} \PYG{o}{=} \PYG{n}{choice95}\PYG{o}{.}\PYG{n}{mean}\PYG{p}{(}\PYG{n}{axis}\PYG{o}{=}\PYG{l+m+mi}{1}\PYG{p}{)}
\PYG{n}{mean95df} \PYG{o}{=} \PYG{n}{pd}\PYG{o}{.}\PYG{n}{DataFrame}\PYG{p}{(}\PYG{n}{data}\PYG{o}{=}\PYG{n}{mean95}\PYG{p}{)}
\PYG{n}{mean95df}\PYG{o}{.}\PYG{n}{rename}\PYG{p}{(}\PYG{n}{columns}\PYG{o}{=}\PYG{p}{\PYGZob{}}\PYG{l+m+mi}{0}\PYG{p}{:} \PYG{l+s+s1}{\PYGZsq{}}\PYG{l+s+s1}{Average Choice}\PYG{l+s+s1}{\PYGZsq{}}\PYG{p}{\PYGZcb{}}\PYG{p}{,} \PYG{n}{inplace}\PYG{o}{=}\PYG{k+kc}{True}\PYG{p}{)}
\PYG{n}{profit95df}\PYG{p}{[}\PYG{l+s+s1}{\PYGZsq{}}\PYG{l+s+s1}{Average Choice}\PYG{l+s+s1}{\PYGZsq{}}\PYG{p}{]} \PYG{o}{=} \PYG{n}{mean95df}\PYG{p}{[}\PYG{l+s+s1}{\PYGZsq{}}\PYG{l+s+s1}{Average Choice}\PYG{l+s+s1}{\PYGZsq{}}\PYG{p}{]}\PYG{o}{.}\PYG{n}{values}
\PYG{n}{profit95df}\PYG{o}{.}\PYG{n}{head}\PYG{p}{(}\PYG{p}{)}
\end{sphinxVerbatim}

\begin{sphinxVerbatim}[commandchars=\\\{\}]
        Margin  Most Common Choice Picked  Most Common Choice     Study  \PYGZbs{}
Subj\PYGZus{}1    1150                         71                   4  Fridberg   
Subj\PYGZus{}2    \PYGZhy{}675                         33                   4  Fridberg   
Subj\PYGZus{}3    \PYGZhy{}750                         38                   4  Fridberg   
Subj\PYGZus{}4    \PYGZhy{}525                         38                   4  Fridberg   
Subj\PYGZus{}5     100                         46                   4  Fridberg   

        Average Choice  
Subj\PYGZus{}1        3.400000  
Subj\PYGZus{}2        2.568421  
Subj\PYGZus{}3        2.778947  
Subj\PYGZus{}4        2.810526  
Subj\PYGZus{}5        3.021053  
\end{sphinxVerbatim}

\begin{sphinxVerbatim}[commandchars=\\\{\}]
\PYG{n}{profit95df}\PYG{o}{.}\PYG{n}{to\PYGZus{}csv}\PYG{p}{(}\PYG{l+s+s1}{\PYGZsq{}}\PYG{l+s+s1}{Data/cleaned95.csv}\PYG{l+s+s1}{\PYGZsq{}}\PYG{p}{)}
\end{sphinxVerbatim}


\section{We now do this for the 100 trial and 150 trial experiments}
\label{\detokenize{data-processing:we-now-do-this-for-the-100-trial-and-150-trial-experiments}}
\begin{sphinxVerbatim}[commandchars=\\\{\}]
\PYG{n}{columnnames100} \PYG{o}{=} \PYG{p}{[}\PYG{l+s+sa}{f}\PYG{l+s+s1}{\PYGZsq{}}\PYG{l+s+s1}{Trial}\PYG{l+s+si}{\PYGZob{}}\PYG{n}{num}\PYG{l+s+si}{\PYGZcb{}}\PYG{l+s+s1}{\PYGZsq{}} \PYG{k}{for} \PYG{n}{num} \PYG{o+ow}{in} \PYG{n+nb}{range}\PYG{p}{(}\PYG{l+m+mi}{1}\PYG{p}{,}\PYG{l+m+mi}{101}\PYG{p}{)}\PYG{p}{]}
\PYG{n}{wins100} \PYG{o}{=} \PYG{n}{win100}
\PYG{n}{wins100} \PYG{o}{=} \PYG{n}{wins100}\PYG{o}{.}\PYG{n}{set\PYGZus{}axis}\PYG{p}{(}\PYG{n}{columnnames100}\PYG{p}{,} \PYG{n}{axis}\PYG{o}{=}\PYG{l+m+mi}{1}\PYG{p}{)}
\end{sphinxVerbatim}

\begin{sphinxVerbatim}[commandchars=\\\{\}]
\PYG{n}{losses100} \PYG{o}{=} \PYG{n}{loss100}
\PYG{n}{losses100} \PYG{o}{=} \PYG{n}{losses100}\PYG{o}{.}\PYG{n}{set\PYGZus{}axis}\PYG{p}{(}\PYG{n}{columnnames100}\PYG{p}{,} \PYG{n}{axis}\PYG{o}{=}\PYG{l+m+mi}{1}\PYG{p}{)}
\end{sphinxVerbatim}

\begin{sphinxVerbatim}[commandchars=\\\{\}]
\PYG{n}{df100\PYGZus{}sum} \PYG{o}{=} \PYG{n}{wins100}\PYG{o}{.}\PYG{n}{add}\PYG{p}{(}\PYG{n}{losses100}\PYG{p}{,} \PYG{n}{fill\PYGZus{}value}\PYG{o}{=}\PYG{l+m+mi}{0}\PYG{p}{)}
\end{sphinxVerbatim}

\begin{sphinxVerbatim}[commandchars=\\\{\}]
\PYG{n}{profit100} \PYG{o}{=} \PYG{n}{df100\PYGZus{}sum}\PYG{o}{.}\PYG{n}{sum}\PYG{p}{(}\PYG{n}{axis}\PYG{o}{=}\PYG{l+m+mi}{1}\PYG{p}{)}
\PYG{n}{profit100df} \PYG{o}{=} \PYG{n}{pd}\PYG{o}{.}\PYG{n}{DataFrame}\PYG{p}{(}\PYG{n}{data}\PYG{o}{=}\PYG{n}{profit100}\PYG{p}{)}
\PYG{n}{profit100df}\PYG{o}{.}\PYG{n}{rename}\PYG{p}{(}\PYG{n}{columns}\PYG{o}{=}\PYG{p}{\PYGZob{}}\PYG{l+m+mi}{0}\PYG{p}{:} \PYG{l+s+s1}{\PYGZsq{}}\PYG{l+s+s1}{Margin}\PYG{l+s+s1}{\PYGZsq{}}\PYG{p}{\PYGZcb{}}\PYG{p}{,} \PYG{n}{inplace}\PYG{o}{=}\PYG{k+kc}{True}\PYG{p}{)}
\end{sphinxVerbatim}

\begin{sphinxVerbatim}[commandchars=\\\{\}]
\PYG{n}{profit100df}\PYG{p}{[}\PYG{l+s+s1}{\PYGZsq{}}\PYG{l+s+s1}{Study}\PYG{l+s+s1}{\PYGZsq{}}\PYG{p}{]} \PYG{o}{=} \PYG{n}{index100}\PYG{p}{[}\PYG{l+s+s1}{\PYGZsq{}}\PYG{l+s+s1}{Study}\PYG{l+s+s1}{\PYGZsq{}}\PYG{p}{]}\PYG{o}{.}\PYG{n}{values}
\end{sphinxVerbatim}

\begin{sphinxVerbatim}[commandchars=\\\{\}]
\PYG{n}{mode100} \PYG{o}{=} \PYG{n}{choice100}\PYG{o}{.}\PYG{n}{mode}\PYG{p}{(}\PYG{n}{axis}\PYG{o}{=}\PYG{l+m+mi}{1}\PYG{p}{)}
\PYG{n}{mode100}\PYG{o}{.}\PYG{n}{rename}\PYG{p}{(}\PYG{n}{columns}\PYG{o}{=}\PYG{p}{\PYGZob{}}\PYG{l+m+mi}{0}\PYG{p}{:} \PYG{l+s+s1}{\PYGZsq{}}\PYG{l+s+s1}{Most Common Choice}\PYG{l+s+s1}{\PYGZsq{}}\PYG{p}{\PYGZcb{}}\PYG{p}{,} \PYG{n}{inplace}\PYG{o}{=}\PYG{k+kc}{True}\PYG{p}{)}
\PYG{n}{profit100df}\PYG{p}{[}\PYG{l+s+s1}{\PYGZsq{}}\PYG{l+s+s1}{Most Common Choice}\PYG{l+s+s1}{\PYGZsq{}}\PYG{p}{]} \PYG{o}{=} \PYG{n}{mode100}\PYG{p}{[}\PYG{l+s+s1}{\PYGZsq{}}\PYG{l+s+s1}{Most Common Choice}\PYG{l+s+s1}{\PYGZsq{}}\PYG{p}{]}\PYG{o}{.}\PYG{n}{values}
\end{sphinxVerbatim}

\begin{sphinxVerbatim}[commandchars=\\\{\}]
\PYG{n}{profit100df}\PYG{p}{[}\PYG{l+s+s1}{\PYGZsq{}}\PYG{l+s+s1}{Most Common Choice}\PYG{l+s+s1}{\PYGZsq{}}\PYG{p}{]}\PYG{o}{.}\PYG{n}{value\PYGZus{}counts}\PYG{p}{(}\PYG{p}{)}
\end{sphinxVerbatim}

\begin{sphinxVerbatim}[commandchars=\\\{\}]
2.0    221
4.0    171
3.0     98
1.0     14
Name: Most Common Choice, dtype: int64
\end{sphinxVerbatim}

\begin{sphinxVerbatim}[commandchars=\\\{\}]
\PYG{n}{profit100df}\PYG{p}{[}\PYG{l+s+s1}{\PYGZsq{}}\PYG{l+s+s1}{Most Common Choice}\PYG{l+s+s1}{\PYGZsq{}}\PYG{p}{]} \PYG{o}{=} \PYG{n}{profit100df}\PYG{p}{[}\PYG{l+s+s1}{\PYGZsq{}}\PYG{l+s+s1}{Most Common Choice}\PYG{l+s+s1}{\PYGZsq{}}\PYG{p}{]}\PYG{o}{.}\PYG{n}{astype}\PYG{p}{(}\PYG{l+s+s1}{\PYGZsq{}}\PYG{l+s+s1}{int64}\PYG{l+s+s1}{\PYGZsq{}}\PYG{p}{)}
\end{sphinxVerbatim}

\begin{sphinxVerbatim}[commandchars=\\\{\}]
\PYG{n}{most\PYGZus{}common\PYGZus{}count100} \PYG{o}{=} \PYG{n}{choice100}\PYG{o}{.}\PYG{n}{apply}\PYG{p}{(}\PYG{n}{pd}\PYG{o}{.}\PYG{n}{Series}\PYG{o}{.}\PYG{n}{value\PYGZus{}counts}\PYG{p}{,} \PYG{n}{axis}\PYG{o}{=}\PYG{l+m+mi}{1}\PYG{p}{)}
\PYG{n}{most\PYGZus{}common\PYGZus{}count100} \PYG{o}{=} \PYG{n}{most\PYGZus{}common\PYGZus{}count100}\PYG{o}{.}\PYG{n}{max}\PYG{p}{(}\PYG{n}{axis}\PYG{o}{=}\PYG{l+m+mi}{1}\PYG{p}{)}
\PYG{n}{profit100df}\PYG{p}{[}\PYG{l+s+s1}{\PYGZsq{}}\PYG{l+s+s1}{Most Common Choice Picked}\PYG{l+s+s1}{\PYGZsq{}}\PYG{p}{]} \PYG{o}{=} \PYG{n}{most\PYGZus{}common\PYGZus{}count100}
\PYG{n}{profit100df}\PYG{o}{.}\PYG{n}{head}\PYG{p}{(}\PYG{p}{)}
\end{sphinxVerbatim}

\begin{sphinxVerbatim}[commandchars=\\\{\}]
        Margin      Study  Most Common Choice  Most Common Choice Picked
Subj\PYGZus{}1   \PYGZhy{}1800  Horstmann                   2                         42
Subj\PYGZus{}2    \PYGZhy{}800  Horstmann                   2                         35
Subj\PYGZus{}3    \PYGZhy{}450  Horstmann                   2                         42
Subj\PYGZus{}4    1200  Horstmann                   4                         35
Subj\PYGZus{}5   \PYGZhy{}1300  Horstmann                   2                         31
\end{sphinxVerbatim}

\begin{sphinxVerbatim}[commandchars=\\\{\}]
\PYG{n}{mean100} \PYG{o}{=} \PYG{n}{choice100}\PYG{o}{.}\PYG{n}{mean}\PYG{p}{(}\PYG{n}{axis}\PYG{o}{=}\PYG{l+m+mi}{1}\PYG{p}{)}
\PYG{n}{mean100df} \PYG{o}{=} \PYG{n}{pd}\PYG{o}{.}\PYG{n}{DataFrame}\PYG{p}{(}\PYG{n}{data}\PYG{o}{=}\PYG{n}{mean100}\PYG{p}{)}
\PYG{n}{mean100df}\PYG{o}{.}\PYG{n}{rename}\PYG{p}{(}\PYG{n}{columns}\PYG{o}{=}\PYG{p}{\PYGZob{}}\PYG{l+m+mi}{0}\PYG{p}{:} \PYG{l+s+s1}{\PYGZsq{}}\PYG{l+s+s1}{Average Choice}\PYG{l+s+s1}{\PYGZsq{}}\PYG{p}{\PYGZcb{}}\PYG{p}{,} \PYG{n}{inplace}\PYG{o}{=}\PYG{k+kc}{True}\PYG{p}{)}
\PYG{n}{profit100df}\PYG{p}{[}\PYG{l+s+s1}{\PYGZsq{}}\PYG{l+s+s1}{Average Choice}\PYG{l+s+s1}{\PYGZsq{}}\PYG{p}{]} \PYG{o}{=} \PYG{n}{mean100df}\PYG{p}{[}\PYG{l+s+s1}{\PYGZsq{}}\PYG{l+s+s1}{Average Choice}\PYG{l+s+s1}{\PYGZsq{}}\PYG{p}{]}\PYG{o}{.}\PYG{n}{values}
\end{sphinxVerbatim}

\begin{sphinxVerbatim}[commandchars=\\\{\}]
\PYG{n}{profit100df}\PYG{o}{.}\PYG{n}{to\PYGZus{}csv}\PYG{p}{(}\PYG{l+s+s1}{\PYGZsq{}}\PYG{l+s+s1}{Data/cleaned100.csv}\PYG{l+s+s1}{\PYGZsq{}}\PYG{p}{)}
\end{sphinxVerbatim}

\sphinxAtStartPar
Lastly, we take the 150 trial data.

\begin{sphinxVerbatim}[commandchars=\\\{\}]
\PYG{n}{columnnames150} \PYG{o}{=} \PYG{p}{[}\PYG{l+s+sa}{f}\PYG{l+s+s1}{\PYGZsq{}}\PYG{l+s+s1}{Trial}\PYG{l+s+si}{\PYGZob{}}\PYG{n}{num}\PYG{l+s+si}{\PYGZcb{}}\PYG{l+s+s1}{\PYGZsq{}} \PYG{k}{for} \PYG{n}{num} \PYG{o+ow}{in} \PYG{n+nb}{range}\PYG{p}{(}\PYG{l+m+mi}{1}\PYG{p}{,}\PYG{l+m+mi}{151}\PYG{p}{)}\PYG{p}{]}
\PYG{n}{wins150} \PYG{o}{=} \PYG{n}{win150}
\PYG{n}{wins150} \PYG{o}{=} \PYG{n}{wins150}\PYG{o}{.}\PYG{n}{set\PYGZus{}axis}\PYG{p}{(}\PYG{n}{columnnames150}\PYG{p}{,} \PYG{n}{axis}\PYG{o}{=}\PYG{l+m+mi}{1}\PYG{p}{)}
\end{sphinxVerbatim}

\begin{sphinxVerbatim}[commandchars=\\\{\}]
\PYG{n}{losses150} \PYG{o}{=} \PYG{n}{loss150}
\PYG{n}{losses150} \PYG{o}{=} \PYG{n}{losses150}\PYG{o}{.}\PYG{n}{set\PYGZus{}axis}\PYG{p}{(}\PYG{n}{columnnames150}\PYG{p}{,} \PYG{n}{axis}\PYG{o}{=}\PYG{l+m+mi}{1}\PYG{p}{)}
\end{sphinxVerbatim}

\begin{sphinxVerbatim}[commandchars=\\\{\}]
\PYG{n}{df150\PYGZus{}sum} \PYG{o}{=} \PYG{n}{wins150}\PYG{o}{.}\PYG{n}{add}\PYG{p}{(}\PYG{n}{losses150}\PYG{p}{,} \PYG{n}{fill\PYGZus{}value}\PYG{o}{=}\PYG{l+m+mi}{0}\PYG{p}{)}
\end{sphinxVerbatim}

\begin{sphinxVerbatim}[commandchars=\\\{\}]
\PYG{n}{profit150} \PYG{o}{=} \PYG{n}{df150\PYGZus{}sum}\PYG{o}{.}\PYG{n}{sum}\PYG{p}{(}\PYG{n}{axis}\PYG{o}{=}\PYG{l+m+mi}{1}\PYG{p}{)}
\PYG{n}{profit150df} \PYG{o}{=} \PYG{n}{pd}\PYG{o}{.}\PYG{n}{DataFrame}\PYG{p}{(}\PYG{n}{data}\PYG{o}{=}\PYG{n}{profit150}\PYG{p}{)}
\PYG{n}{profit150df}\PYG{o}{.}\PYG{n}{rename}\PYG{p}{(}\PYG{n}{columns}\PYG{o}{=}\PYG{p}{\PYGZob{}}\PYG{l+m+mi}{0}\PYG{p}{:} \PYG{l+s+s1}{\PYGZsq{}}\PYG{l+s+s1}{Margin}\PYG{l+s+s1}{\PYGZsq{}}\PYG{p}{\PYGZcb{}}\PYG{p}{,} \PYG{n}{inplace}\PYG{o}{=}\PYG{k+kc}{True}\PYG{p}{)}
\end{sphinxVerbatim}

\begin{sphinxVerbatim}[commandchars=\\\{\}]
\PYG{n}{profit150df}\PYG{p}{[}\PYG{l+s+s1}{\PYGZsq{}}\PYG{l+s+s1}{Study}\PYG{l+s+s1}{\PYGZsq{}}\PYG{p}{]} \PYG{o}{=} \PYG{n}{index150}\PYG{p}{[}\PYG{l+s+s1}{\PYGZsq{}}\PYG{l+s+s1}{Study}\PYG{l+s+s1}{\PYGZsq{}}\PYG{p}{]}\PYG{o}{.}\PYG{n}{values}
\end{sphinxVerbatim}

\begin{sphinxVerbatim}[commandchars=\\\{\}]
\PYG{n}{mode150} \PYG{o}{=} \PYG{n}{choice150}\PYG{o}{.}\PYG{n}{mode}\PYG{p}{(}\PYG{n}{axis}\PYG{o}{=}\PYG{l+m+mi}{1}\PYG{p}{)}
\PYG{n}{mode150}\PYG{o}{.}\PYG{n}{rename}\PYG{p}{(}\PYG{n}{columns}\PYG{o}{=}\PYG{p}{\PYGZob{}}\PYG{l+m+mi}{0}\PYG{p}{:} \PYG{l+s+s1}{\PYGZsq{}}\PYG{l+s+s1}{Most Common Choice}\PYG{l+s+s1}{\PYGZsq{}}\PYG{p}{\PYGZcb{}}\PYG{p}{,} \PYG{n}{inplace}\PYG{o}{=}\PYG{k+kc}{True}\PYG{p}{)}
\PYG{n}{profit150df}\PYG{p}{[}\PYG{l+s+s1}{\PYGZsq{}}\PYG{l+s+s1}{Most Common Choice}\PYG{l+s+s1}{\PYGZsq{}}\PYG{p}{]} \PYG{o}{=} \PYG{n}{mode150}\PYG{p}{[}\PYG{l+s+s1}{\PYGZsq{}}\PYG{l+s+s1}{Most Common Choice}\PYG{l+s+s1}{\PYGZsq{}}\PYG{p}{]}\PYG{o}{.}\PYG{n}{values}
\end{sphinxVerbatim}

\begin{sphinxVerbatim}[commandchars=\\\{\}]
\PYG{n}{most\PYGZus{}common\PYGZus{}count150} \PYG{o}{=} \PYG{n}{choice150}\PYG{o}{.}\PYG{n}{apply}\PYG{p}{(}\PYG{n}{pd}\PYG{o}{.}\PYG{n}{Series}\PYG{o}{.}\PYG{n}{value\PYGZus{}counts}\PYG{p}{,} \PYG{n}{axis}\PYG{o}{=}\PYG{l+m+mi}{1}\PYG{p}{)}
\PYG{n}{most\PYGZus{}common\PYGZus{}count150} \PYG{o}{=} \PYG{n}{most\PYGZus{}common\PYGZus{}count150}\PYG{o}{.}\PYG{n}{max}\PYG{p}{(}\PYG{n}{axis}\PYG{o}{=}\PYG{l+m+mi}{1}\PYG{p}{)}
\PYG{n}{most\PYGZus{}common\PYGZus{}count150} \PYG{o}{=} \PYG{n}{most\PYGZus{}common\PYGZus{}count150}\PYG{o}{.}\PYG{n}{astype}\PYG{p}{(}\PYG{l+s+s1}{\PYGZsq{}}\PYG{l+s+s1}{int64}\PYG{l+s+s1}{\PYGZsq{}}\PYG{p}{)}
\PYG{n}{profit150df}\PYG{p}{[}\PYG{l+s+s1}{\PYGZsq{}}\PYG{l+s+s1}{Most Common Choice Picked}\PYG{l+s+s1}{\PYGZsq{}}\PYG{p}{]} \PYG{o}{=} \PYG{n}{most\PYGZus{}common\PYGZus{}count150}
\PYG{n}{profit150df}\PYG{o}{.}\PYG{n}{head}\PYG{p}{(}\PYG{p}{)}
\end{sphinxVerbatim}

\begin{sphinxVerbatim}[commandchars=\\\{\}]
        Margin             Study  Most Common Choice  \PYGZbs{}
Subj\PYGZus{}1    \PYGZhy{}550  Steingroever2011                 1.0   
Subj\PYGZus{}2   \PYGZhy{}1600  Steingroever2011                 2.0   
Subj\PYGZus{}3     900  Steingroever2011                 4.0   
Subj\PYGZus{}4    2200  Steingroever2011                 4.0   
Subj\PYGZus{}5    1900  Steingroever2011                 4.0   

        Most Common Choice Picked  
Subj\PYGZus{}1                         46  
Subj\PYGZus{}2                         57  
Subj\PYGZus{}3                         88  
Subj\PYGZus{}4                        111  
Subj\PYGZus{}5                        135  
\end{sphinxVerbatim}

\begin{sphinxVerbatim}[commandchars=\\\{\}]
\PYG{n}{mean150} \PYG{o}{=} \PYG{n}{choice150}\PYG{o}{.}\PYG{n}{mean}\PYG{p}{(}\PYG{n}{axis}\PYG{o}{=}\PYG{l+m+mi}{1}\PYG{p}{)}
\PYG{n}{mean150df} \PYG{o}{=} \PYG{n}{pd}\PYG{o}{.}\PYG{n}{DataFrame}\PYG{p}{(}\PYG{n}{data}\PYG{o}{=}\PYG{n}{mean150}\PYG{p}{)}
\PYG{n}{mean150df}\PYG{o}{.}\PYG{n}{rename}\PYG{p}{(}\PYG{n}{columns}\PYG{o}{=}\PYG{p}{\PYGZob{}}\PYG{l+m+mi}{0}\PYG{p}{:} \PYG{l+s+s1}{\PYGZsq{}}\PYG{l+s+s1}{Average Choice}\PYG{l+s+s1}{\PYGZsq{}}\PYG{p}{\PYGZcb{}}\PYG{p}{,} \PYG{n}{inplace}\PYG{o}{=}\PYG{k+kc}{True}\PYG{p}{)}
\PYG{n}{profit150df}\PYG{p}{[}\PYG{l+s+s1}{\PYGZsq{}}\PYG{l+s+s1}{Average Choice}\PYG{l+s+s1}{\PYGZsq{}}\PYG{p}{]} \PYG{o}{=} \PYG{n}{mean150df}\PYG{p}{[}\PYG{l+s+s1}{\PYGZsq{}}\PYG{l+s+s1}{Average Choice}\PYG{l+s+s1}{\PYGZsq{}}\PYG{p}{]}\PYG{o}{.}\PYG{n}{values}
\end{sphinxVerbatim}

\begin{sphinxVerbatim}[commandchars=\\\{\}]
\PYG{n}{profit150df}\PYG{o}{.}\PYG{n}{to\PYGZus{}csv}\PYG{p}{(}\PYG{l+s+s1}{\PYGZsq{}}\PYG{l+s+s1}{Data/cleaned150.csv}\PYG{l+s+s1}{\PYGZsq{}}\PYG{p}{)}
\end{sphinxVerbatim}

\begin{sphinxVerbatim}[commandchars=\\\{\}]
\PYG{n}{merged95\PYGZus{}150} \PYG{o}{=} \PYG{n}{pd}\PYG{o}{.}\PYG{n}{concat}\PYG{p}{(}\PYG{p}{[}\PYG{n}{profit95df}\PYG{p}{,} \PYG{n}{profit150df}\PYG{p}{]}\PYG{p}{)}
\end{sphinxVerbatim}

\begin{sphinxVerbatim}[commandchars=\\\{\}]
\PYG{n}{mergedall} \PYG{o}{=} \PYG{n}{pd}\PYG{o}{.}\PYG{n}{concat}\PYG{p}{(}\PYG{p}{[}\PYG{n}{merged95\PYGZus{}150}\PYG{p}{,} \PYG{n}{profit100df}\PYG{p}{]}\PYG{p}{)}
\PYG{n}{mergedall}\PYG{p}{[}\PYG{l+s+s1}{\PYGZsq{}}\PYG{l+s+s1}{Most Common Choice}\PYG{l+s+s1}{\PYGZsq{}}\PYG{p}{]} \PYG{o}{=} \PYG{n}{mergedall}\PYG{p}{[}\PYG{l+s+s1}{\PYGZsq{}}\PYG{l+s+s1}{Most Common Choice}\PYG{l+s+s1}{\PYGZsq{}}\PYG{p}{]}\PYG{o}{.}\PYG{n}{astype}\PYG{p}{(}\PYG{l+s+s1}{\PYGZsq{}}\PYG{l+s+s1}{int64}\PYG{l+s+s1}{\PYGZsq{}}\PYG{p}{)}
\PYG{n}{mergedall}
\end{sphinxVerbatim}

\begin{sphinxVerbatim}[commandchars=\\\{\}]
          Margin  Most Common Choice Picked  Most Common Choice     Study  \PYGZbs{}
Subj\PYGZus{}1      1150                         71                   4  Fridberg   
Subj\PYGZus{}2      \PYGZhy{}675                         33                   4  Fridberg   
Subj\PYGZus{}3      \PYGZhy{}750                         38                   4  Fridberg   
Subj\PYGZus{}4      \PYGZhy{}525                         38                   4  Fridberg   
Subj\PYGZus{}5       100                         46                   4  Fridberg   
...          ...                        ...                 ...       ...   
Subj\PYGZus{}500      75                         29                   2    Worthy   
Subj\PYGZus{}501     600                         44                   3    Worthy   
Subj\PYGZus{}502   \PYGZhy{}1525                         32                   2    Worthy   
Subj\PYGZus{}503    \PYGZhy{}750                         27                   1    Worthy   
Subj\PYGZus{}504     175                         54                   4    Worthy   

          Average Choice  
Subj\PYGZus{}1          3.400000  
Subj\PYGZus{}2          2.568421  
Subj\PYGZus{}3          2.778947  
Subj\PYGZus{}4          2.810526  
Subj\PYGZus{}5          3.021053  
...                  ...  
Subj\PYGZus{}500        2.630000  
Subj\PYGZus{}501        2.840000  
Subj\PYGZus{}502        2.380000  
Subj\PYGZus{}503        2.460000  
Subj\PYGZus{}504        3.100000  

[617 rows x 5 columns]
\end{sphinxVerbatim}

\sphinxAtStartPar
For some of our comparisons we may want to draw on in our k\sphinxhyphen{}means clusters we need to change our study values from strings to integers. After plotting our k\sphinxhyphen{}means algorithm this will allow us to plot a scatter plot comprising of the different studies which will be colour coded based on their numbers here.

\begin{sphinxVerbatim}[commandchars=\\\{\}]
\PYG{n}{replacements\PYGZus{}study} \PYG{o}{=} \PYG{p}{\PYGZob{}}
  \PYG{l+s+sa}{r}\PYG{l+s+s1}{\PYGZsq{}}\PYG{l+s+s1}{Fridberg}\PYG{l+s+s1}{\PYGZsq{}}\PYG{p}{:} \PYG{l+m+mi}{0}\PYG{p}{,}  
  \PYG{l+s+sa}{r}\PYG{l+s+s1}{\PYGZsq{}}\PYG{l+s+s1}{Horstmann}\PYG{l+s+s1}{\PYGZsq{}}\PYG{p}{:} \PYG{l+m+mi}{1}\PYG{p}{,}
  \PYG{l+s+sa}{r}\PYG{l+s+s1}{\PYGZsq{}}\PYG{l+s+s1}{Kjome}\PYG{l+s+s1}{\PYGZsq{}}\PYG{p}{:} \PYG{l+m+mi}{2}\PYG{p}{,}
  \PYG{l+s+sa}{r}\PYG{l+s+s1}{\PYGZsq{}}\PYG{l+s+s1}{Maia}\PYG{l+s+s1}{\PYGZsq{}}\PYG{p}{:} \PYG{l+m+mi}{3}\PYG{p}{,}
  \PYG{l+s+sa}{r}\PYG{l+s+s1}{\PYGZsq{}}\PYG{l+s+s1}{SteingroverInPrep}\PYG{l+s+s1}{\PYGZsq{}}\PYG{p}{:} \PYG{l+m+mi}{4}\PYG{p}{,}
  \PYG{l+s+sa}{r}\PYG{l+s+s1}{\PYGZsq{}}\PYG{l+s+s1}{Premkumar}\PYG{l+s+s1}{\PYGZsq{}}\PYG{p}{:} \PYG{l+m+mi}{5}\PYG{p}{,}
  \PYG{l+s+sa}{r}\PYG{l+s+s1}{\PYGZsq{}}\PYG{l+s+s1}{Wood}\PYG{l+s+s1}{\PYGZsq{}}\PYG{p}{:} \PYG{l+m+mi}{6}\PYG{p}{,}
  \PYG{l+s+sa}{r}\PYG{l+s+s1}{\PYGZsq{}}\PYG{l+s+s1}{Worthy}\PYG{l+s+s1}{\PYGZsq{}}\PYG{p}{:} \PYG{l+m+mi}{7}\PYG{p}{,}
  \PYG{l+s+sa}{r}\PYG{l+s+s1}{\PYGZsq{}}\PYG{l+s+s1}{Steingroever2011}\PYG{l+s+s1}{\PYGZsq{}}\PYG{p}{:} \PYG{l+m+mi}{8}\PYG{p}{,}
  \PYG{l+s+sa}{r}\PYG{l+s+s1}{\PYGZsq{}}\PYG{l+s+s1}{Wetzels}\PYG{l+s+s1}{\PYGZsq{}}\PYG{p}{:} \PYG{l+m+mi}{9}\PYG{p}{,}  
\PYG{p}{\PYGZcb{}}

\PYG{n}{mergedall}\PYG{p}{[}\PYG{l+s+s1}{\PYGZsq{}}\PYG{l+s+s1}{StudyNumber}\PYG{l+s+s1}{\PYGZsq{}}\PYG{p}{]} \PYG{o}{=} \PYG{n}{mergedall}\PYG{o}{.}\PYG{n}{Study}\PYG{o}{.}\PYG{n}{replace}\PYG{p}{(}\PYG{n}{replacements\PYGZus{}study}\PYG{p}{,} \PYG{n}{regex}\PYG{o}{=}\PYG{k+kc}{True}\PYG{p}{)}
\PYG{n}{mergedall} \PYG{o}{=} \PYG{n}{mergedall}\PYG{o}{.}\PYG{n}{drop}\PYG{p}{(}\PYG{n}{columns}\PYG{o}{=}\PYG{p}{[}\PYG{l+s+s1}{\PYGZsq{}}\PYG{l+s+s1}{Study}\PYG{l+s+s1}{\PYGZsq{}}\PYG{p}{]}\PYG{p}{)}
\PYG{n}{mergedall}
\end{sphinxVerbatim}

\begin{sphinxVerbatim}[commandchars=\\\{\}]
          Margin  Most Common Choice Picked  Most Common Choice  \PYGZbs{}
Subj\PYGZus{}1      1150                         71                   4   
Subj\PYGZus{}2      \PYGZhy{}675                         33                   4   
Subj\PYGZus{}3      \PYGZhy{}750                         38                   4   
Subj\PYGZus{}4      \PYGZhy{}525                         38                   4   
Subj\PYGZus{}5       100                         46                   4   
...          ...                        ...                 ...   
Subj\PYGZus{}500      75                         29                   2   
Subj\PYGZus{}501     600                         44                   3   
Subj\PYGZus{}502   \PYGZhy{}1525                         32                   2   
Subj\PYGZus{}503    \PYGZhy{}750                         27                   1   
Subj\PYGZus{}504     175                         54                   4   

          Average Choice  StudyNumber  
Subj\PYGZus{}1          3.400000            0  
Subj\PYGZus{}2          2.568421            0  
Subj\PYGZus{}3          2.778947            0  
Subj\PYGZus{}4          2.810526            0  
Subj\PYGZus{}5          3.021053            0  
...                  ...          ...  
Subj\PYGZus{}500        2.630000            7  
Subj\PYGZus{}501        2.840000            7  
Subj\PYGZus{}502        2.380000            7  
Subj\PYGZus{}503        2.460000            7  
Subj\PYGZus{}504        3.100000            7  

[617 rows x 5 columns]
\end{sphinxVerbatim}

\begin{sphinxVerbatim}[commandchars=\\\{\}]
\PYG{n}{mergedall}\PYG{o}{.}\PYG{n}{to\PYGZus{}csv}\PYG{p}{(}\PYG{l+s+s1}{\PYGZsq{}}\PYG{l+s+s1}{Data/cleaned\PYGZus{}all.csv}\PYG{l+s+s1}{\PYGZsq{}}\PYG{p}{)}
\end{sphinxVerbatim}


\section{Standardize our Data}
\label{\detokenize{data-processing:standardize-our-data}}
\sphinxAtStartPar
To work best with our k\sphinxhyphen{}means algorithm we choose to standardize our values in our joined dataset. This is because the k\sphinxhyphen{}means algorithm is a distance based algorithm, calculating the similarity between points based on distance. This gives the data a mean of 0 and standard deviation of 1 and gives common ground between features which would use different values such as our margin and average choice columns.

\begin{sphinxVerbatim}[commandchars=\\\{\}]
\PYG{n}{scaler} \PYG{o}{=} \PYG{n}{preprocessing}\PYG{o}{.}\PYG{n}{StandardScaler}\PYG{p}{(}\PYG{p}{)}\PYG{o}{.}\PYG{n}{fit}\PYG{p}{(}\PYG{n}{mergedall}\PYG{p}{)}
\PYG{n}{X\PYGZus{}scaled} \PYG{o}{=} \PYG{n}{scaler}\PYG{o}{.}\PYG{n}{transform}\PYG{p}{(}\PYG{n}{mergedall}\PYG{p}{)}
\PYG{n}{X\PYGZus{}scaled}\PYG{o}{.}\PYG{n}{std}\PYG{p}{(}\PYG{n}{axis}\PYG{o}{=}\PYG{l+m+mi}{0}\PYG{p}{)}
\end{sphinxVerbatim}

\begin{sphinxVerbatim}[commandchars=\\\{\}]
array([1., 1., 1., 1., 1.])
\end{sphinxVerbatim}

\begin{sphinxVerbatim}[commandchars=\\\{\}]
\PYG{n}{standard\PYGZus{}all} \PYG{o}{=} \PYG{n}{pd}\PYG{o}{.}\PYG{n}{DataFrame}\PYG{p}{(}\PYG{n}{X\PYGZus{}scaled}\PYG{p}{)}
\end{sphinxVerbatim}

\begin{sphinxVerbatim}[commandchars=\\\{\}]
\PYG{n}{standard\PYGZus{}all} \PYG{o}{=} \PYG{n}{standard\PYGZus{}all}\PYG{o}{.}\PYG{n}{rename}\PYG{p}{(}\PYG{n}{columns}\PYG{o}{=}\PYG{p}{\PYGZob{}}\PYG{l+m+mi}{0}\PYG{p}{:}\PYG{l+s+s1}{\PYGZsq{}}\PYG{l+s+s1}{Margin}\PYG{l+s+s1}{\PYGZsq{}}\PYG{p}{,} \PYG{l+m+mi}{1}\PYG{p}{:} \PYG{l+s+s1}{\PYGZsq{}}\PYG{l+s+s1}{Most Common Choice Picked}\PYG{l+s+s1}{\PYGZsq{}}\PYG{p}{,} \PYG{l+m+mi}{2}\PYG{p}{:} \PYG{l+s+s1}{\PYGZsq{}}\PYG{l+s+s1}{Most Common Choice}\PYG{l+s+s1}{\PYGZsq{}}\PYG{p}{,} \PYG{l+m+mi}{3}\PYG{p}{:} \PYG{l+s+s1}{\PYGZsq{}}\PYG{l+s+s1}{Average Choice}\PYG{l+s+s1}{\PYGZsq{}}\PYG{p}{,}
                                           \PYG{l+m+mi}{4}\PYG{p}{:} \PYG{l+s+s1}{\PYGZsq{}}\PYG{l+s+s1}{StudyNumber}\PYG{l+s+s1}{\PYGZsq{}}\PYG{p}{\PYGZcb{}}\PYG{p}{)}
\end{sphinxVerbatim}

\begin{sphinxVerbatim}[commandchars=\\\{\}]
\PYG{n}{standard\PYGZus{}all}
\end{sphinxVerbatim}

\begin{sphinxVerbatim}[commandchars=\\\{\}]
       Margin  Most Common Choice Picked  Most Common Choice  Average Choice  \PYGZbs{}
0    1.044988                   1.062672            1.234405        2.271920   
1   \PYGZhy{}0.414346                  \PYGZhy{}0.804770            1.234405       \PYGZhy{}0.410317   
2   \PYGZhy{}0.474318                  \PYGZhy{}0.559054            1.234405        0.268730   
3   \PYGZhy{}0.294400                  \PYGZhy{}0.559054            1.234405        0.370587   
4    0.205371                  \PYGZhy{}0.165908            1.234405        1.049635   
..        ...                        ...                 ...             ...   
612  0.185381                  \PYGZhy{}1.001343           \PYGZhy{}0.923181       \PYGZhy{}0.211696   
613  0.605189                  \PYGZhy{}0.264195            0.155612        0.465654   
614 \PYGZhy{}1.094035                  \PYGZhy{}0.853913           \PYGZhy{}0.923181       \PYGZhy{}1.018065   
615 \PYGZhy{}0.474318                  \PYGZhy{}1.099629           \PYGZhy{}2.001975       \PYGZhy{}0.760027   
616  0.265344                   0.227238            1.234405        1.304277   

     StudyNumber  
0      \PYGZhy{}1.609380  
1      \PYGZhy{}1.609380  
2      \PYGZhy{}1.609380  
3      \PYGZhy{}1.609380  
4      \PYGZhy{}1.609380  
..           ...  
612     0.952696  
613     0.952696  
614     0.952696  
615     0.952696  
616     0.952696  

[617 rows x 5 columns]
\end{sphinxVerbatim}

\begin{sphinxVerbatim}[commandchars=\\\{\}]
\PYG{n}{standard\PYGZus{}all}\PYG{o}{.}\PYG{n}{to\PYGZus{}csv}\PYG{p}{(}\PYG{l+s+s1}{\PYGZsq{}}\PYG{l+s+s1}{data/standardized\PYGZus{}all.csv}\PYG{l+s+s1}{\PYGZsq{}}\PYG{p}{)}
\end{sphinxVerbatim}


\chapter{3. Clustering}
\label{\detokenize{clustering:clustering}}\label{\detokenize{clustering::doc}}

\section{After our analysis of the data we must now cluster the data accordingly}
\label{\detokenize{clustering:after-our-analysis-of-the-data-we-must-now-cluster-the-data-accordingly}}
\begin{sphinxVerbatim}[commandchars=\\\{\}]
\PYG{k+kn}{import} \PYG{n+nn}{pandas} \PYG{k}{as} \PYG{n+nn}{pd}
\PYG{k+kn}{import} \PYG{n+nn}{seaborn} \PYG{k}{as} \PYG{n+nn}{sn}
\PYG{k+kn}{import} \PYG{n+nn}{numpy} \PYG{k}{as} \PYG{n+nn}{np}
\PYG{k+kn}{import} \PYG{n+nn}{matplotlib}\PYG{n+nn}{.}\PYG{n+nn}{pyplot} \PYG{k}{as} \PYG{n+nn}{plt}
\PYG{k+kn}{from} \PYG{n+nn}{sklearn}\PYG{n+nn}{.}\PYG{n+nn}{cluster} \PYG{k+kn}{import} \PYG{n}{KMeans}\PYG{p}{,} \PYG{n}{AgglomerativeClustering}
\PYG{k+kn}{from} \PYG{n+nn}{sklearn}\PYG{n+nn}{.}\PYG{n+nn}{metrics} \PYG{k+kn}{import} \PYG{n}{silhouette\PYGZus{}score}
\PYG{k+kn}{import} \PYG{n+nn}{seaborn} \PYG{k}{as} \PYG{n+nn}{sns}
\PYG{k+kn}{import} \PYG{n+nn}{warnings}
\PYG{n}{warnings}\PYG{o}{.}\PYG{n}{filterwarnings}\PYG{p}{(}\PYG{l+s+s2}{\PYGZdq{}}\PYG{l+s+s2}{ignore}\PYG{l+s+s2}{\PYGZdq{}}\PYG{p}{)}
\end{sphinxVerbatim}


\subsection{First, we read in our data}
\label{\detokenize{clustering:first-we-read-in-our-data}}
\begin{sphinxVerbatim}[commandchars=\\\{\}]
\PYG{n}{index95} \PYG{o}{=} \PYG{n}{pd}\PYG{o}{.}\PYG{n}{read\PYGZus{}csv}\PYG{p}{(}\PYG{l+s+s1}{\PYGZsq{}}\PYG{l+s+s1}{data/index\PYGZus{}95.csv}\PYG{l+s+s1}{\PYGZsq{}}\PYG{p}{)}
\PYG{n}{index100} \PYG{o}{=} \PYG{n}{pd}\PYG{o}{.}\PYG{n}{read\PYGZus{}csv}\PYG{p}{(}\PYG{l+s+s1}{\PYGZsq{}}\PYG{l+s+s1}{data/index\PYGZus{}100.csv}\PYG{l+s+s1}{\PYGZsq{}}\PYG{p}{)}
\PYG{n}{index150} \PYG{o}{=} \PYG{n}{pd}\PYG{o}{.}\PYG{n}{read\PYGZus{}csv}\PYG{p}{(}\PYG{l+s+s1}{\PYGZsq{}}\PYG{l+s+s1}{data/index\PYGZus{}150.csv}\PYG{l+s+s1}{\PYGZsq{}}\PYG{p}{)}
\PYG{n}{win95} \PYG{o}{=} \PYG{n}{pd}\PYG{o}{.}\PYG{n}{read\PYGZus{}csv}\PYG{p}{(}\PYG{l+s+s1}{\PYGZsq{}}\PYG{l+s+s1}{data/wi\PYGZus{}95.csv}\PYG{l+s+s1}{\PYGZsq{}}\PYG{p}{)}
\PYG{n}{win100} \PYG{o}{=} \PYG{n}{pd}\PYG{o}{.}\PYG{n}{read\PYGZus{}csv}\PYG{p}{(}\PYG{l+s+s1}{\PYGZsq{}}\PYG{l+s+s1}{data/wi\PYGZus{}100.csv}\PYG{l+s+s1}{\PYGZsq{}}\PYG{p}{)}
\PYG{n}{win150} \PYG{o}{=} \PYG{n}{pd}\PYG{o}{.}\PYG{n}{read\PYGZus{}csv}\PYG{p}{(}\PYG{l+s+s1}{\PYGZsq{}}\PYG{l+s+s1}{data/wi\PYGZus{}150.csv}\PYG{l+s+s1}{\PYGZsq{}}\PYG{p}{)}
\PYG{n}{loss95} \PYG{o}{=} \PYG{n}{pd}\PYG{o}{.}\PYG{n}{read\PYGZus{}csv}\PYG{p}{(}\PYG{l+s+s1}{\PYGZsq{}}\PYG{l+s+s1}{data/lo\PYGZus{}95.csv}\PYG{l+s+s1}{\PYGZsq{}}\PYG{p}{)}
\PYG{n}{loss100} \PYG{o}{=} \PYG{n}{pd}\PYG{o}{.}\PYG{n}{read\PYGZus{}csv}\PYG{p}{(}\PYG{l+s+s1}{\PYGZsq{}}\PYG{l+s+s1}{data/lo\PYGZus{}100.csv}\PYG{l+s+s1}{\PYGZsq{}}\PYG{p}{)}
\PYG{n}{loss150} \PYG{o}{=} \PYG{n}{pd}\PYG{o}{.}\PYG{n}{read\PYGZus{}csv}\PYG{p}{(}\PYG{l+s+s1}{\PYGZsq{}}\PYG{l+s+s1}{data/lo\PYGZus{}150.csv}\PYG{l+s+s1}{\PYGZsq{}}\PYG{p}{)}
\PYG{n}{choice95} \PYG{o}{=} \PYG{n}{pd}\PYG{o}{.}\PYG{n}{read\PYGZus{}csv}\PYG{p}{(}\PYG{l+s+s1}{\PYGZsq{}}\PYG{l+s+s1}{data/choice\PYGZus{}95.csv}\PYG{l+s+s1}{\PYGZsq{}}\PYG{p}{)}
\PYG{n}{choice100} \PYG{o}{=} \PYG{n}{pd}\PYG{o}{.}\PYG{n}{read\PYGZus{}csv}\PYG{p}{(}\PYG{l+s+s1}{\PYGZsq{}}\PYG{l+s+s1}{data/choice\PYGZus{}100.csv}\PYG{l+s+s1}{\PYGZsq{}}\PYG{p}{)}
\PYG{n}{choice150} \PYG{o}{=} \PYG{n}{pd}\PYG{o}{.}\PYG{n}{read\PYGZus{}csv}\PYG{p}{(}\PYG{l+s+s1}{\PYGZsq{}}\PYG{l+s+s1}{data/choice\PYGZus{}150.csv}\PYG{l+s+s1}{\PYGZsq{}}\PYG{p}{)}
\end{sphinxVerbatim}

\sphinxAtStartPar
Cleaned data from processing

\begin{sphinxVerbatim}[commandchars=\\\{\}]
\PYG{n}{cleaned95} \PYG{o}{=} \PYG{n}{pd}\PYG{o}{.}\PYG{n}{read\PYGZus{}csv}\PYG{p}{(}\PYG{l+s+s1}{\PYGZsq{}}\PYG{l+s+s1}{data/cleaned95.csv}\PYG{l+s+s1}{\PYGZsq{}}\PYG{p}{,} \PYG{n}{index\PYGZus{}col}\PYG{o}{=}\PYG{l+s+s1}{\PYGZsq{}}\PYG{l+s+s1}{Unnamed: 0}\PYG{l+s+s1}{\PYGZsq{}}\PYG{p}{)}
\PYG{n}{cleaned100} \PYG{o}{=} \PYG{n}{pd}\PYG{o}{.}\PYG{n}{read\PYGZus{}csv}\PYG{p}{(}\PYG{l+s+s1}{\PYGZsq{}}\PYG{l+s+s1}{data/cleaned100.csv}\PYG{l+s+s1}{\PYGZsq{}}\PYG{p}{,} \PYG{n}{index\PYGZus{}col}\PYG{o}{=}\PYG{l+s+s1}{\PYGZsq{}}\PYG{l+s+s1}{Unnamed: 0}\PYG{l+s+s1}{\PYGZsq{}}\PYG{p}{)}
\PYG{n}{cleaned150} \PYG{o}{=} \PYG{n}{pd}\PYG{o}{.}\PYG{n}{read\PYGZus{}csv}\PYG{p}{(}\PYG{l+s+s1}{\PYGZsq{}}\PYG{l+s+s1}{data/cleaned150.csv}\PYG{l+s+s1}{\PYGZsq{}}\PYG{p}{,} \PYG{n}{index\PYGZus{}col}\PYG{o}{=}\PYG{l+s+s1}{\PYGZsq{}}\PYG{l+s+s1}{Unnamed: 0}\PYG{l+s+s1}{\PYGZsq{}}\PYG{p}{)}
\end{sphinxVerbatim}

\sphinxAtStartPar
Initially, I decided to cluster based on the profit/loss margin for each subject and their most common choice. However, the most common choice would limit the clusters greatly I felt. There would only be 4 possible values (1,2,3 or 4) and this would limit what we could learn from the approriate cluster analysis. I looked into clustering on the number of times each deck was selected but this would involve multiple clusters, one for each choice against the profit margin but I decided against it. This lead me to going back to my data processing and creating the average choice column to add to my data. This was the sum of all the subjects selection divided by the number of trials and I felt this would provide better cluster analysis as a result as there would be far more variety in the range of values. I also decided to look into how many times subjects picked their most common choice. I felt this would give us a good overview of the respective studies and how they played the game, did they play safe and stick to what they know would win or would they attempt riskier decks in the hope of winning more? We are going to use our standardized data from our data processing as standardized data tends to work better with the k\sphinxhyphen{}means algorithm.


\chapter{K\sphinxhyphen{}Means analysis}
\label{\detokenize{clustering:k-means-analysis}}

\section{Finding our value for k}
\label{\detokenize{clustering:finding-our-value-for-k}}

\subsection{Silhouette Scores for our  dataset}
\label{\detokenize{clustering:silhouette-scores-for-our-dataset}}
\sphinxAtStartPar
We now use another metric to test for the optimal number of clusters in our datasets. We try the silhouette coefficient which calculates the robustness of a clustering technique. This metric measures the degree of seperation between clusters. The score scales from \sphinxhyphen{}1 to 1. 1 means the clusters are very distinguished and perfectly easy to identify, 0 means the clusters are indifferent or hard to identify and \sphinxhyphen{}1 means the clusters are assigned in the wrong way. We will try to use this with our earlier elbow coefficient to confirm the optimal number of clusters for our datasets. The formula for the silhouette score is defined as follows:
\begin{equation*}
\begin{split}
 silhouette - score = {b_{i} - a_{i} \over max(bi, a_{i})} \\
\end{split}
\end{equation*}
\sphinxAtStartPar
In the formula above from \sphinxhref{https://towardsdatascience.com/unsupervised-learning-techniques-using-python-k-means-and-silhouette-score-for-clustering-d6dd1f30b660}{here} bi represents the shortest mean distance between a point to all points in any other cluster of which i is not a part whereas ai is the mean distance of i and all data points from the same cluster.

\begin{sphinxVerbatim}[commandchars=\\\{\}]
\PYG{n}{standard} \PYG{o}{=} \PYG{n}{pd}\PYG{o}{.}\PYG{n}{read\PYGZus{}csv}\PYG{p}{(}\PYG{l+s+s1}{\PYGZsq{}}\PYG{l+s+s1}{data/standardized\PYGZus{}all.csv}\PYG{l+s+s1}{\PYGZsq{}}\PYG{p}{,} \PYG{n}{index\PYGZus{}col}\PYG{o}{=}\PYG{l+s+s1}{\PYGZsq{}}\PYG{l+s+s1}{Unnamed: 0}\PYG{l+s+s1}{\PYGZsq{}}\PYG{p}{)}
\PYG{n}{standard}\PYG{o}{.}\PYG{n}{head}\PYG{p}{(}\PYG{p}{)}
\end{sphinxVerbatim}

\begin{sphinxVerbatim}[commandchars=\\\{\}]
     Margin  Most Common Choice Picked  Most Common Choice  Average Choice  \PYGZbs{}
0  1.044988                   1.062672            1.234405        2.271920   
1 \PYGZhy{}0.414346                  \PYGZhy{}0.804770            1.234405       \PYGZhy{}0.410317   
2 \PYGZhy{}0.474318                  \PYGZhy{}0.559054            1.234405        0.268730   
3 \PYGZhy{}0.294400                  \PYGZhy{}0.559054            1.234405        0.370587   
4  0.205371                  \PYGZhy{}0.165908            1.234405        1.049635   

   StudyNumber  
0     \PYGZhy{}1.60938  
1     \PYGZhy{}1.60938  
2     \PYGZhy{}1.60938  
3     \PYGZhy{}1.60938  
4     \PYGZhy{}1.60938  
\end{sphinxVerbatim}

\begin{sphinxVerbatim}[commandchars=\\\{\}]
\PYG{k}{for} \PYG{n}{n} \PYG{o+ow}{in} \PYG{n+nb}{range}\PYG{p}{(}\PYG{l+m+mi}{2}\PYG{p}{,} \PYG{l+m+mi}{11}\PYG{p}{)}\PYG{p}{:}
    \PYG{n}{km} \PYG{o}{=} \PYG{n}{KMeans}\PYG{p}{(}\PYG{n}{n\PYGZus{}clusters}\PYG{o}{=}\PYG{n}{n}\PYG{p}{)}
\PYG{c+c1}{\PYGZsh{}}
\PYG{c+c1}{\PYGZsh{} Fit the KMeans model}
\PYG{c+c1}{\PYGZsh{} Have to pick subset of columns as Study column is in string format}
    \PYG{n}{km}\PYG{o}{.}\PYG{n}{fit\PYGZus{}predict}\PYG{p}{(}\PYG{n}{standard}\PYG{p}{)}
\PYG{c+c1}{\PYGZsh{}}
\PYG{c+c1}{\PYGZsh{} Calculate Silhoutte Score}
\PYG{c+c1}{\PYGZsh{}}
    \PYG{n}{score} \PYG{o}{=} \PYG{n}{silhouette\PYGZus{}score}\PYG{p}{(}\PYG{n}{standard}\PYG{p}{,} \PYG{n}{km}\PYG{o}{.}\PYG{n}{labels\PYGZus{}}\PYG{p}{,} \PYG{n}{metric}\PYG{o}{=}\PYG{l+s+s1}{\PYGZsq{}}\PYG{l+s+s1}{euclidean}\PYG{l+s+s1}{\PYGZsq{}}\PYG{p}{)}
\PYG{c+c1}{\PYGZsh{}}
\PYG{c+c1}{\PYGZsh{} Print the score}
\PYG{c+c1}{\PYGZsh{}}
    \PYG{n+nb}{print}\PYG{p}{(}\PYG{l+s+s1}{\PYGZsq{}}\PYG{l+s+s1}{N = }\PYG{l+s+s1}{\PYGZsq{}} \PYG{o}{+} \PYG{n+nb}{str}\PYG{p}{(}\PYG{n}{n}\PYG{p}{)} \PYG{o}{+} \PYG{l+s+s1}{\PYGZsq{}}\PYG{l+s+s1}{ Silhouette Score: }\PYG{l+s+si}{\PYGZpc{}.3f}\PYG{l+s+s1}{\PYGZsq{}} \PYG{o}{\PYGZpc{}} \PYG{n}{score}\PYG{p}{)}
\end{sphinxVerbatim}

\begin{sphinxVerbatim}[commandchars=\\\{\}]
N = 2 Silhouette Score: 0.313
N = 3 Silhouette Score: 0.289
N = 4 Silhouette Score: 0.267
N = 5 Silhouette Score: 0.273
N = 6 Silhouette Score: 0.277
N = 7 Silhouette Score: 0.290
N = 8 Silhouette Score: 0.299
N = 9 Silhouette Score: 0.302
N = 10 Silhouette Score: 0.294
\end{sphinxVerbatim}


\subsection{Elbow Method}
\label{\detokenize{clustering:elbow-method}}
\begin{sphinxVerbatim}[commandchars=\\\{\}]
\PYG{n}{distortions\PYGZus{}joined\PYGZus{}st} \PYG{o}{=} \PYG{p}{[}\PYG{p}{]}
\PYG{n}{K} \PYG{o}{=} \PYG{n+nb}{range}\PYG{p}{(}\PYG{l+m+mi}{1}\PYG{p}{,} \PYG{l+m+mi}{10}\PYG{p}{)}
\PYG{k}{for} \PYG{n}{k} \PYG{o+ow}{in} \PYG{n}{K}\PYG{p}{:}
    \PYG{n}{kmeanModel} \PYG{o}{=} \PYG{n}{KMeans}\PYG{p}{(}\PYG{n}{n\PYGZus{}clusters}\PYG{o}{=}\PYG{n}{k}\PYG{p}{)}
    \PYG{n}{kmeanModel}\PYG{o}{.}\PYG{n}{fit}\PYG{p}{(}\PYG{n}{standard}\PYG{p}{)}
    \PYG{n}{distortions\PYGZus{}joined\PYGZus{}st}\PYG{o}{.}\PYG{n}{append}\PYG{p}{(}\PYG{n}{kmeanModel}\PYG{o}{.}\PYG{n}{inertia\PYGZus{}}\PYG{p}{)}
\end{sphinxVerbatim}

\begin{sphinxVerbatim}[commandchars=\\\{\}]
\PYG{n}{plt}\PYG{o}{.}\PYG{n}{figure}\PYG{p}{(}\PYG{n}{figsize}\PYG{o}{=}\PYG{p}{(}\PYG{l+m+mi}{16}\PYG{p}{,}\PYG{l+m+mi}{8}\PYG{p}{)}\PYG{p}{)}
\PYG{n}{plt}\PYG{o}{.}\PYG{n}{plot}\PYG{p}{(}\PYG{n}{K}\PYG{p}{,} \PYG{n}{distortions\PYGZus{}joined\PYGZus{}st}\PYG{p}{,} \PYG{l+s+s1}{\PYGZsq{}}\PYG{l+s+s1}{bx\PYGZhy{}}\PYG{l+s+s1}{\PYGZsq{}}\PYG{p}{)}
\PYG{n}{plt}\PYG{o}{.}\PYG{n}{xlabel}\PYG{p}{(}\PYG{l+s+s1}{\PYGZsq{}}\PYG{l+s+s1}{k}\PYG{l+s+s1}{\PYGZsq{}}\PYG{p}{)}
\PYG{n}{plt}\PYG{o}{.}\PYG{n}{ylabel}\PYG{p}{(}\PYG{l+s+s1}{\PYGZsq{}}\PYG{l+s+s1}{Distortion}\PYG{l+s+s1}{\PYGZsq{}}\PYG{p}{)}
\PYG{n}{plt}\PYG{o}{.}\PYG{n}{title}\PYG{p}{(}\PYG{l+s+s1}{\PYGZsq{}}\PYG{l+s+s1}{The Elbow Method showing the optimal k for standardized Dataset}\PYG{l+s+s1}{\PYGZsq{}}\PYG{p}{)}
\PYG{n}{plt}\PYG{o}{.}\PYG{n}{show}\PYG{p}{(}\PYG{p}{)}
\end{sphinxVerbatim}

\noindent\sphinxincludegraphics{{clustering_13_0}.png}

\sphinxAtStartPar
Looking at our elbow method and silhouette scores for the dataset as whole we can conclude using k=2 or k=3 is a safe value to use for the amount of clusters in our data.


\section{Now we look at what we will cluster on in our data}
\label{\detokenize{clustering:now-we-look-at-what-we-will-cluster-on-in-our-data}}
\begin{sphinxVerbatim}[commandchars=\\\{\}]
\PYG{n}{pd}\PYG{o}{.}\PYG{n}{plotting}\PYG{o}{.}\PYG{n}{scatter\PYGZus{}matrix}\PYG{p}{(}\PYG{n}{standard}\PYG{p}{,} \PYG{n}{figsize}\PYG{o}{=}\PYG{p}{(}\PYG{l+m+mi}{16}\PYG{p}{,}\PYG{l+m+mi}{12}\PYG{p}{)}\PYG{p}{,} \PYG{n}{hist\PYGZus{}kwds}\PYG{o}{=}\PYG{n+nb}{dict}\PYG{p}{(}\PYG{n}{bins}\PYG{o}{=}\PYG{l+m+mi}{50}\PYG{p}{)}\PYG{p}{,} \PYG{n}{cmap}\PYG{o}{=}\PYG{l+s+s2}{\PYGZdq{}}\PYG{l+s+s2}{Set1}\PYG{l+s+s2}{\PYGZdq{}}\PYG{p}{)}
\PYG{n}{plt}\PYG{o}{.}\PYG{n}{show}\PYG{p}{(}\PYG{p}{)}
\end{sphinxVerbatim}

\noindent\sphinxincludegraphics{{clustering_16_0}.png}

\sphinxAtStartPar
After looking at this scatterplot comparing the various columns against each other in a scatter plot I felt it might be worth looking into how regularly a participant picked their most common choice and the profit margins. We could further look into this by looking at what was their most common choice and the respective study they belonged to.

\begin{sphinxVerbatim}[commandchars=\\\{\}]
\PYG{n}{kmeans\PYGZus{}margin\PYGZus{}standard} \PYG{o}{=} \PYG{n}{KMeans}\PYG{p}{(}\PYG{n}{n\PYGZus{}clusters}\PYG{o}{=}\PYG{l+m+mi}{3}\PYG{p}{)}\PYG{o}{.}\PYG{n}{fit}\PYG{p}{(}\PYG{n}{standard}\PYG{p}{[}\PYG{p}{[}\PYG{l+s+s2}{\PYGZdq{}}\PYG{l+s+s2}{Margin}\PYG{l+s+s2}{\PYGZdq{}}\PYG{p}{,} \PYG{l+s+s2}{\PYGZdq{}}\PYG{l+s+s2}{Most Common Choice Picked}\PYG{l+s+s2}{\PYGZdq{}}\PYG{p}{]}\PYG{p}{]}\PYG{p}{)}
\PYG{n}{centroids\PYGZus{}betas\PYGZus{}standard} \PYG{o}{=} \PYG{n}{kmeans\PYGZus{}margin\PYGZus{}standard}\PYG{o}{.}\PYG{n}{cluster\PYGZus{}centers\PYGZus{}}
\end{sphinxVerbatim}

\begin{sphinxVerbatim}[commandchars=\\\{\}]
\PYG{n}{plt}\PYG{o}{.}\PYG{n}{figure}\PYG{p}{(}\PYG{n}{figsize}\PYG{o}{=}\PYG{p}{(}\PYG{l+m+mi}{16}\PYG{p}{,}\PYG{l+m+mi}{8}\PYG{p}{)}\PYG{p}{)}
\PYG{n}{plt}\PYG{o}{.}\PYG{n}{scatter}\PYG{p}{(}\PYG{n}{standard}\PYG{p}{[}\PYG{l+s+s1}{\PYGZsq{}}\PYG{l+s+s1}{Margin}\PYG{l+s+s1}{\PYGZsq{}}\PYG{p}{]}\PYG{p}{,} \PYG{n}{standard}\PYG{p}{[}\PYG{l+s+s1}{\PYGZsq{}}\PYG{l+s+s1}{Most Common Choice Picked}\PYG{l+s+s1}{\PYGZsq{}}\PYG{p}{]}\PYG{p}{,} \PYG{n}{c}\PYG{o}{=} \PYG{n}{kmeans\PYGZus{}margin\PYGZus{}standard}\PYG{o}{.}\PYG{n}{labels\PYGZus{}}\PYG{p}{,} \PYG{n}{cmap} \PYG{o}{=} \PYG{l+s+s2}{\PYGZdq{}}\PYG{l+s+s2}{Set1}\PYG{l+s+s2}{\PYGZdq{}}\PYG{p}{,} \PYG{n}{alpha}\PYG{o}{=}\PYG{l+m+mf}{0.5}\PYG{p}{)}
\PYG{n}{plt}\PYG{o}{.}\PYG{n}{scatter}\PYG{p}{(}\PYG{n}{centroids\PYGZus{}betas\PYGZus{}standard}\PYG{p}{[}\PYG{p}{:}\PYG{p}{,} \PYG{l+m+mi}{0}\PYG{p}{]}\PYG{p}{,} \PYG{n}{centroids\PYGZus{}betas\PYGZus{}standard}\PYG{p}{[}\PYG{p}{:}\PYG{p}{,} \PYG{l+m+mi}{1}\PYG{p}{]}\PYG{p}{,} \PYG{n}{c}\PYG{o}{=}\PYG{l+s+s1}{\PYGZsq{}}\PYG{l+s+s1}{blue}\PYG{l+s+s1}{\PYGZsq{}}\PYG{p}{,} \PYG{n}{marker}\PYG{o}{=}\PYG{l+s+s1}{\PYGZsq{}}\PYG{l+s+s1}{x}\PYG{l+s+s1}{\PYGZsq{}}\PYG{p}{)}
\PYG{n}{plt}\PYG{o}{.}\PYG{n}{title}\PYG{p}{(}\PYG{l+s+s1}{\PYGZsq{}}\PYG{l+s+s1}{K\PYGZhy{}Means cluster for all Subjects \PYGZhy{} Most Common Choice Picked}\PYG{l+s+s1}{\PYGZsq{}}\PYG{p}{)}
\PYG{n}{plt}\PYG{o}{.}\PYG{n}{xlabel}\PYG{p}{(}\PYG{l+s+s1}{\PYGZsq{}}\PYG{l+s+s1}{Margin}\PYG{l+s+s1}{\PYGZsq{}}\PYG{p}{)}
\PYG{n}{plt}\PYG{o}{.}\PYG{n}{ylabel}\PYG{p}{(}\PYG{l+s+s1}{\PYGZsq{}}\PYG{l+s+s1}{Times Most Common Choice Picked}\PYG{l+s+s1}{\PYGZsq{}}\PYG{p}{)}
\PYG{n}{plt}\PYG{o}{.}\PYG{n}{show}\PYG{p}{(}\PYG{p}{)}
\end{sphinxVerbatim}

\noindent\sphinxincludegraphics{{clustering_19_0}.png}

\begin{sphinxVerbatim}[commandchars=\\\{\}]
\PYG{n}{plt}\PYG{o}{.}\PYG{n}{figure}\PYG{p}{(}\PYG{n}{figsize}\PYG{o}{=}\PYG{p}{(}\PYG{l+m+mi}{16}\PYG{p}{,}\PYG{l+m+mi}{8}\PYG{p}{)}\PYG{p}{)}
\PYG{n}{plt}\PYG{o}{.}\PYG{n}{scatter}\PYG{p}{(}\PYG{n}{standard}\PYG{p}{[}\PYG{l+s+s1}{\PYGZsq{}}\PYG{l+s+s1}{Margin}\PYG{l+s+s1}{\PYGZsq{}}\PYG{p}{]}\PYG{p}{,} \PYG{n}{standard}\PYG{p}{[}\PYG{l+s+s1}{\PYGZsq{}}\PYG{l+s+s1}{Most Common Choice Picked}\PYG{l+s+s1}{\PYGZsq{}}\PYG{p}{]}\PYG{p}{,} \PYG{n}{c}\PYG{o}{=}\PYG{n}{standard}\PYG{p}{[}\PYG{l+s+s1}{\PYGZsq{}}\PYG{l+s+s1}{StudyNumber}\PYG{l+s+s1}{\PYGZsq{}}\PYG{p}{]}\PYG{p}{,} \PYG{n}{cmap}\PYG{o}{=}\PYG{l+s+s1}{\PYGZsq{}}\PYG{l+s+s1}{tab10}\PYG{l+s+s1}{\PYGZsq{}}\PYG{p}{)}
\PYG{n}{plt}\PYG{o}{.}\PYG{n}{title}\PYG{p}{(}\PYG{l+s+s2}{\PYGZdq{}}\PYG{l+s+s2}{Most common choice picked \PYGZhy{} Study groups}\PYG{l+s+s2}{\PYGZdq{}}\PYG{p}{)}
\PYG{n}{plt}\PYG{o}{.}\PYG{n}{xlabel}\PYG{p}{(}\PYG{l+s+s1}{\PYGZsq{}}\PYG{l+s+s1}{Margin}\PYG{l+s+s1}{\PYGZsq{}}\PYG{p}{)}
\PYG{n}{plt}\PYG{o}{.}\PYG{n}{ylabel}\PYG{p}{(}\PYG{l+s+s1}{\PYGZsq{}}\PYG{l+s+s1}{Times most common choice selected}\PYG{l+s+s1}{\PYGZsq{}}\PYG{p}{)}
\PYG{n}{plt}\PYG{o}{.}\PYG{n}{colorbar}\PYG{p}{(}\PYG{p}{)}
\end{sphinxVerbatim}

\begin{sphinxVerbatim}[commandchars=\\\{\}]
\PYGZlt{}matplotlib.colorbar.Colorbar at 0x20fd1c423a0\PYGZgt{}
\end{sphinxVerbatim}

\noindent\sphinxincludegraphics{{clustering_20_1}.png}

\sphinxAtStartPar
From earlier our studies are as follows: Fridberg: 0, Horstmann: 1, Kjome: 2, Maia: 3, SteingroverInPrep: 4, Premkumar: 5, Wood: 6, Worthy: 7, Steingroever2011: 8 and Wetzels: 9. Immediately we notice our neon scatters here on both the left and right side of the plot. There is a large amount of subjects from Steingroever2011’s study that regularly pick their favoured outcome, a lot of these can be seen to the right of the plot in the more profitable outcome while picking this selection 100 trials or more in the 150 trial experiment. The majority of this study picks their most common choice at least 80 times or more from our plot also. We can see a couple of outliers in this study also, picking their favoured outcome well over a hundred times with poor results. Again, this study is based on the youngest specifically mentioned mean of subjects (19.9 years old) and it certainly seems to play a part in their decision making. We can see from our scatterplot something similar with the Wetzels study, a student based study with a noticeable group of these subjects picking their favoured choice 80 times or more and winning money over the course of their trials. It is also interesting to note the clusters to the left and centre from our K\sphinxhyphen{}means plot containing a sizable proportion of subjects from the Wood study (in pink). This is a particularly interesting study as the first 90 participants were between the ages of 18\sphinxhyphen{}40 and the rest had a mean age of 76.98 years old. Despite it being a 100 trial study the vast majority of this study pick their preferred choice around 40\sphinxhyphen{}50 times, less than or equal to half of their trials. A lot of these participants also lose money over the course of their trials which raises interesting questions about how age can impair decision making. It is interesting to note when researching the Wood et al study that it was regarded as an outlier in that age over time impairs performance in the IGT {[}\hyperlink{cite.zbibliography:id2}{1}{]}. However, in general it is seen as something that can negatively impact decision making over time. It also suggests in {[}\hyperlink{cite.zbibliography:id2}{1}{]} in later adulthood that a low loss strategy is seen more predominately than the end profit. The Horstmann study follows a similar dispersion to the Woods study with a large amount of subjects centred around the middle cluster and also picking their preffered choice 40\sphinxhyphen{}50 out of 100 trials which could be seen as a relatively low numbers and maybe hints at an adaptive approach to the game, where after a period of maybe settling for preferred decks they adapt as their wins/losses dictate. With a similar number of participants in this trial to the Woods trial the one key difference is far more of these subjects lean to the right cluster (breaking even or making a marginal gain). Again, looking at the age demography this is a young adult group with a mean age of 25.6 years old. Something suggested in {[}\hyperlink{cite.zbibliography:id2}{1}{]} could hold true here in that adults past adolescence prefer experimental decision making instead of just frequency preference. This could also explain the cohort of the Horstmann study who followed a similar decision making process (40\sphinxhyphen{}50 most common choice) but lost money in the end. Another study which also has an interesting cluster is the Worthy study, which has 35 subjects the majority of which (22) is female. This study contains a lot of subjects who picked their most common choice less than 40 times and very few subjects are above 50 or so. It is something that ties in with some of the potential behavioural findings of {[}\hyperlink{cite.zbibliography:id3}{2}{]} in that women are more sensitive to losses in the long term profitable decks. This would explain the constant tinkering between the decks and the lower most common choice value among many subjects in this study. The aforementioned Horstmann study also has a large number of female participants in it (82/162) and as mentioned previously has similarly lower count of the times the most common choice was picked. It is something mentioned in {[}\hyperlink{cite.zbibliography:id4}{3}{]} that women pick more disadvantageous decks as they mix a policy of exploration and exploitation, whereas as men will after initial exploration focus on exploitation then. This could definitely explain why these studies with a lower common choice pick and specified amount of females taking part in the study have a more mixed approach to the task and why they float around breaking even. They obviously accept some losses in exploration but exploit the winning cards regularly enough to be around even. We see further information from {[}\hyperlink{cite.zbibliography:id5}{4}{]} in gender decision making from the task. Similar to our data there is 40 healthy males and females in this article and it suggests females are more sensitive to losses than males. This could explain the more varied approach females seem to take in our k\sphinxhyphen{}means plot.

\begin{sphinxVerbatim}[commandchars=\\\{\}]
\PYG{n}{joined} \PYG{o}{=} \PYG{n}{pd}\PYG{o}{.}\PYG{n}{read\PYGZus{}csv}\PYG{p}{(}\PYG{l+s+s1}{\PYGZsq{}}\PYG{l+s+s1}{data/cleaned\PYGZus{}all.csv}\PYG{l+s+s1}{\PYGZsq{}}\PYG{p}{,} \PYG{n}{index\PYGZus{}col}\PYG{o}{=}\PYG{l+s+s1}{\PYGZsq{}}\PYG{l+s+s1}{Unnamed: 0}\PYG{l+s+s1}{\PYGZsq{}}\PYG{p}{)}
\PYG{n}{joined}\PYG{p}{[}\PYG{l+s+s1}{\PYGZsq{}}\PYG{l+s+s1}{cluster}\PYG{l+s+s1}{\PYGZsq{}}\PYG{p}{]} \PYG{o}{=} \PYG{n}{kmeans\PYGZus{}margin\PYGZus{}standard}\PYG{o}{.}\PYG{n}{labels\PYGZus{}}\PYG{o}{.}\PYG{n}{tolist}\PYG{p}{(}\PYG{p}{)}
\end{sphinxVerbatim}

\begin{sphinxVerbatim}[commandchars=\\\{\}]
\PYG{n}{heatmap} \PYG{o}{=} \PYG{n}{joined}\PYG{p}{[}\PYG{p}{[}\PYG{l+s+s1}{\PYGZsq{}}\PYG{l+s+s1}{StudyNumber}\PYG{l+s+s1}{\PYGZsq{}}\PYG{p}{,} \PYG{l+s+s1}{\PYGZsq{}}\PYG{l+s+s1}{cluster}\PYG{l+s+s1}{\PYGZsq{}}\PYG{p}{]}\PYG{p}{]}
\PYG{n}{replacements} \PYG{o}{=} \PYG{p}{\PYGZob{}}
  \PYG{l+m+mi}{0}\PYG{p}{:} \PYG{l+s+sa}{r}\PYG{l+s+s1}{\PYGZsq{}}\PYG{l+s+s1}{Fridberg}\PYG{l+s+s1}{\PYGZsq{}}\PYG{p}{,}  
  \PYG{l+m+mi}{1}\PYG{p}{:} \PYG{l+s+sa}{r}\PYG{l+s+s1}{\PYGZsq{}}\PYG{l+s+s1}{Horstmann}\PYG{l+s+s1}{\PYGZsq{}}\PYG{p}{,}
  \PYG{l+m+mi}{2}\PYG{p}{:} \PYG{l+s+sa}{r}\PYG{l+s+s1}{\PYGZsq{}}\PYG{l+s+s1}{Kjome}\PYG{l+s+s1}{\PYGZsq{}}\PYG{p}{,}
  \PYG{l+m+mi}{3}\PYG{p}{:} \PYG{l+s+sa}{r}\PYG{l+s+s1}{\PYGZsq{}}\PYG{l+s+s1}{Maia}\PYG{l+s+s1}{\PYGZsq{}}\PYG{p}{,}
  \PYG{l+m+mi}{4}\PYG{p}{:} \PYG{l+s+sa}{r}\PYG{l+s+s1}{\PYGZsq{}}\PYG{l+s+s1}{SteingroverInPrep}\PYG{l+s+s1}{\PYGZsq{}}\PYG{p}{,}
  \PYG{l+m+mi}{5}\PYG{p}{:} \PYG{l+s+sa}{r}\PYG{l+s+s1}{\PYGZsq{}}\PYG{l+s+s1}{Premkumar}\PYG{l+s+s1}{\PYGZsq{}}\PYG{p}{,}
  \PYG{l+m+mi}{6}\PYG{p}{:} \PYG{l+s+sa}{r}\PYG{l+s+s1}{\PYGZsq{}}\PYG{l+s+s1}{Wood}\PYG{l+s+s1}{\PYGZsq{}}\PYG{p}{,}
  \PYG{l+m+mi}{7}\PYG{p}{:} \PYG{l+s+sa}{r}\PYG{l+s+s1}{\PYGZsq{}}\PYG{l+s+s1}{Worthy}\PYG{l+s+s1}{\PYGZsq{}}\PYG{p}{,}
  \PYG{l+m+mi}{8}\PYG{p}{:} \PYG{l+s+sa}{r}\PYG{l+s+s1}{\PYGZsq{}}\PYG{l+s+s1}{Steingroever2011}\PYG{l+s+s1}{\PYGZsq{}}\PYG{p}{,}
  \PYG{l+m+mi}{9}\PYG{p}{:} \PYG{l+s+sa}{r}\PYG{l+s+s1}{\PYGZsq{}}\PYG{l+s+s1}{Wetzels}\PYG{l+s+s1}{\PYGZsq{}}\PYG{p}{,}  
\PYG{p}{\PYGZcb{}}

\PYG{n}{heatmap}\PYG{p}{[}\PYG{l+s+s1}{\PYGZsq{}}\PYG{l+s+s1}{Study}\PYG{l+s+s1}{\PYGZsq{}}\PYG{p}{]} \PYG{o}{=} \PYG{n}{heatmap}\PYG{o}{.}\PYG{n}{StudyNumber}\PYG{o}{.}\PYG{n}{replace}\PYG{p}{(}\PYG{n}{replacements}\PYG{p}{,} \PYG{n}{regex}\PYG{o}{=}\PYG{k+kc}{True}\PYG{p}{)}
\PYG{n}{heatmap} \PYG{o}{=} \PYG{n}{heatmap}\PYG{o}{.}\PYG{n}{drop}\PYG{p}{(}\PYG{n}{columns}\PYG{o}{=}\PYG{p}{[}\PYG{l+s+s1}{\PYGZsq{}}\PYG{l+s+s1}{StudyNumber}\PYG{l+s+s1}{\PYGZsq{}}\PYG{p}{]}\PYG{p}{)}
\end{sphinxVerbatim}

\begin{sphinxVerbatim}[commandchars=\\\{\}]
\PYG{n}{counts} \PYG{o}{=} \PYG{n}{heatmap}\PYG{o}{.}\PYG{n}{groupby}\PYG{p}{(}\PYG{l+s+s1}{\PYGZsq{}}\PYG{l+s+s1}{Study}\PYG{l+s+s1}{\PYGZsq{}}\PYG{p}{)}\PYG{p}{[}\PYG{l+s+s1}{\PYGZsq{}}\PYG{l+s+s1}{cluster}\PYG{l+s+s1}{\PYGZsq{}}\PYG{p}{]}\PYG{o}{.}\PYG{n}{value\PYGZus{}counts}\PYG{p}{(}\PYG{p}{)}
\end{sphinxVerbatim}

\begin{sphinxVerbatim}[commandchars=\\\{\}]
\PYG{n}{histdf} \PYG{o}{=} \PYG{n}{pd}\PYG{o}{.}\PYG{n}{DataFrame}\PYG{p}{(}\PYG{n}{counts}\PYG{p}{)}
\PYG{n}{histdf} \PYG{o}{=} \PYG{n}{histdf}\PYG{o}{.}\PYG{n}{rename}\PYG{p}{(}\PYG{n}{columns}\PYG{o}{=}\PYG{p}{\PYGZob{}}\PYG{l+s+s1}{\PYGZsq{}}\PYG{l+s+s1}{cluster}\PYG{l+s+s1}{\PYGZsq{}}\PYG{p}{:} \PYG{l+s+s1}{\PYGZsq{}}\PYG{l+s+s1}{number}\PYG{l+s+s1}{\PYGZsq{}}\PYG{p}{\PYGZcb{}}\PYG{p}{)}
\PYG{n}{plt} \PYG{o}{=} \PYG{n}{histdf}\PYG{o}{.}\PYG{n}{plot}\PYG{o}{.}\PYG{n}{bar}\PYG{p}{(}\PYG{n}{figsize}\PYG{o}{=}\PYG{p}{(}\PYG{l+m+mi}{16}\PYG{p}{,} \PYG{l+m+mi}{8}\PYG{p}{)}\PYG{p}{,} \PYG{n}{ylabel}\PYG{o}{=}\PYG{l+s+s1}{\PYGZsq{}}\PYG{l+s+s1}{No. in Cluster}\PYG{l+s+s1}{\PYGZsq{}}\PYG{p}{,} \PYG{n}{title}\PYG{o}{=}\PYG{l+s+s1}{\PYGZsq{}}\PYG{l+s+s1}{How many per study was in each cluster respectively}\PYG{l+s+s1}{\PYGZsq{}}\PYG{p}{)}
\PYG{n}{plt}
\end{sphinxVerbatim}

\begin{sphinxVerbatim}[commandchars=\\\{\}]
\PYGZlt{}AxesSubplot:title=\PYGZob{}\PYGZsq{}center\PYGZsq{}:\PYGZsq{}How many per study was in each cluster respectively\PYGZsq{}\PYGZcb{}, xlabel=\PYGZsq{}Study,cluster\PYGZsq{}, ylabel=\PYGZsq{}No. in Cluster\PYGZsq{}\PYGZgt{}
\end{sphinxVerbatim}

\noindent\sphinxincludegraphics{{clustering_25_1}.png}

\begin{sphinxVerbatim}[commandchars=\\\{\}]
\PYG{n}{centroids\PYGZus{}betas\PYGZus{}standard}
\end{sphinxVerbatim}

\begin{sphinxVerbatim}[commandchars=\\\{\}]
array([[ 1.443018  ,  1.61018515],
       [\PYGZhy{}1.07289372, \PYGZhy{}0.09542955],
       [ 0.29751279, \PYGZhy{}0.35872379]])
\end{sphinxVerbatim}

\sphinxAtStartPar
We see our cluster centres above, the left most cluster is denoted as 1, the right most cluster denoted as 0 and the central cluster as 2.

\sphinxAtStartPar
Using our histogram above it confirms some of our previous statements from earlier. We do indeed see a large majority of Wood study subjects in the less profitable and more varied choice selection cluster to the left of our k\sphinxhyphen{}means scatter plot. We also see a large number of subjects from the Horstmann study in the central cluster which from our earlier analysis of profitable studies adds up also. We also see can confirm how unprofitable the Wood study was with very few subjects in the right most cluster.

\begin{sphinxVerbatim}[commandchars=\\\{\}]
\PYG{n}{commonchoice} \PYG{o}{=} \PYG{n}{joined}\PYG{p}{[}\PYG{p}{[}\PYG{l+s+s1}{\PYGZsq{}}\PYG{l+s+s1}{Most Common Choice}\PYG{l+s+s1}{\PYGZsq{}}\PYG{p}{,} \PYG{l+s+s1}{\PYGZsq{}}\PYG{l+s+s1}{StudyNumber}\PYG{l+s+s1}{\PYGZsq{}}\PYG{p}{]}\PYG{p}{]}
\PYG{n}{replacements} \PYG{o}{=} \PYG{p}{\PYGZob{}}
  \PYG{l+m+mi}{0}\PYG{p}{:} \PYG{l+s+sa}{r}\PYG{l+s+s1}{\PYGZsq{}}\PYG{l+s+s1}{Fridberg}\PYG{l+s+s1}{\PYGZsq{}}\PYG{p}{,}  
  \PYG{l+m+mi}{1}\PYG{p}{:} \PYG{l+s+sa}{r}\PYG{l+s+s1}{\PYGZsq{}}\PYG{l+s+s1}{Horstmann}\PYG{l+s+s1}{\PYGZsq{}}\PYG{p}{,}
  \PYG{l+m+mi}{2}\PYG{p}{:} \PYG{l+s+sa}{r}\PYG{l+s+s1}{\PYGZsq{}}\PYG{l+s+s1}{Kjome}\PYG{l+s+s1}{\PYGZsq{}}\PYG{p}{,}
  \PYG{l+m+mi}{3}\PYG{p}{:} \PYG{l+s+sa}{r}\PYG{l+s+s1}{\PYGZsq{}}\PYG{l+s+s1}{Maia}\PYG{l+s+s1}{\PYGZsq{}}\PYG{p}{,}
  \PYG{l+m+mi}{4}\PYG{p}{:} \PYG{l+s+sa}{r}\PYG{l+s+s1}{\PYGZsq{}}\PYG{l+s+s1}{SteingroverInPrep}\PYG{l+s+s1}{\PYGZsq{}}\PYG{p}{,}
  \PYG{l+m+mi}{5}\PYG{p}{:} \PYG{l+s+sa}{r}\PYG{l+s+s1}{\PYGZsq{}}\PYG{l+s+s1}{Premkumar}\PYG{l+s+s1}{\PYGZsq{}}\PYG{p}{,}
  \PYG{l+m+mi}{6}\PYG{p}{:} \PYG{l+s+sa}{r}\PYG{l+s+s1}{\PYGZsq{}}\PYG{l+s+s1}{Wood}\PYG{l+s+s1}{\PYGZsq{}}\PYG{p}{,}
  \PYG{l+m+mi}{7}\PYG{p}{:} \PYG{l+s+sa}{r}\PYG{l+s+s1}{\PYGZsq{}}\PYG{l+s+s1}{Worthy}\PYG{l+s+s1}{\PYGZsq{}}\PYG{p}{,}
  \PYG{l+m+mi}{8}\PYG{p}{:} \PYG{l+s+sa}{r}\PYG{l+s+s1}{\PYGZsq{}}\PYG{l+s+s1}{Steingroever2011}\PYG{l+s+s1}{\PYGZsq{}}\PYG{p}{,}
  \PYG{l+m+mi}{9}\PYG{p}{:} \PYG{l+s+sa}{r}\PYG{l+s+s1}{\PYGZsq{}}\PYG{l+s+s1}{Wetzels}\PYG{l+s+s1}{\PYGZsq{}}\PYG{p}{,}  
\PYG{p}{\PYGZcb{}}

\PYG{n}{commonchoice}\PYG{p}{[}\PYG{l+s+s1}{\PYGZsq{}}\PYG{l+s+s1}{Study}\PYG{l+s+s1}{\PYGZsq{}}\PYG{p}{]} \PYG{o}{=} \PYG{n}{commonchoice}\PYG{o}{.}\PYG{n}{StudyNumber}\PYG{o}{.}\PYG{n}{replace}\PYG{p}{(}\PYG{n}{replacements}\PYG{p}{,} \PYG{n}{regex}\PYG{o}{=}\PYG{k+kc}{True}\PYG{p}{)}
\PYG{n}{commonchoice} \PYG{o}{=} \PYG{n}{commonchoice}\PYG{o}{.}\PYG{n}{drop}\PYG{p}{(}\PYG{n}{columns}\PYG{o}{=}\PYG{p}{[}\PYG{l+s+s1}{\PYGZsq{}}\PYG{l+s+s1}{StudyNumber}\PYG{l+s+s1}{\PYGZsq{}}\PYG{p}{]}\PYG{p}{)}
\PYG{n}{common\PYGZus{}counts} \PYG{o}{=} \PYG{n}{commonchoice}\PYG{o}{.}\PYG{n}{groupby}\PYG{p}{(}\PYG{l+s+s1}{\PYGZsq{}}\PYG{l+s+s1}{Study}\PYG{l+s+s1}{\PYGZsq{}}\PYG{p}{)}\PYG{p}{[}\PYG{l+s+s1}{\PYGZsq{}}\PYG{l+s+s1}{Most Common Choice}\PYG{l+s+s1}{\PYGZsq{}}\PYG{p}{]}\PYG{o}{.}\PYG{n}{value\PYGZus{}counts}\PYG{p}{(}\PYG{p}{)}
\PYG{n}{plt2} \PYG{o}{=} \PYG{n}{common\PYGZus{}counts}\PYG{o}{.}\PYG{n}{plot}\PYG{o}{.}\PYG{n}{bar}\PYG{p}{(}\PYG{n}{figsize}\PYG{o}{=}\PYG{p}{(}\PYG{l+m+mi}{16}\PYG{p}{,} \PYG{l+m+mi}{8}\PYG{p}{)}\PYG{p}{,} \PYG{n}{color}\PYG{o}{=}\PYG{l+s+s1}{\PYGZsq{}}\PYG{l+s+s1}{green}\PYG{l+s+s1}{\PYGZsq{}}\PYG{p}{,} \PYG{n}{ylabel}\PYG{o}{=}\PYG{l+s+s1}{\PYGZsq{}}\PYG{l+s+s1}{Times Picked}\PYG{l+s+s1}{\PYGZsq{}}\PYG{p}{,} \PYG{n}{title}\PYG{o}{=}\PYG{l+s+s1}{\PYGZsq{}}\PYG{l+s+s1}{Breakdown of how many subjects per studies had a specific common choice}\PYG{l+s+s1}{\PYGZsq{}}\PYG{p}{)}
\PYG{n}{plt2}
\end{sphinxVerbatim}

\begin{sphinxVerbatim}[commandchars=\\\{\}]
\PYGZlt{}AxesSubplot:title=\PYGZob{}\PYGZsq{}center\PYGZsq{}:\PYGZsq{}Breakdown of how many subjects per studies had a specific common choice\PYGZsq{}\PYGZcb{}, xlabel=\PYGZsq{}Study,Most Common Choice\PYGZsq{}, ylabel=\PYGZsq{}Times Picked\PYGZsq{}\PYGZgt{}
\end{sphinxVerbatim}

\noindent\sphinxincludegraphics{{clustering_29_2}.png}

\sphinxAtStartPar
To further add on to our observations from our k\sphinxhyphen{}means analysis for most common choice picked and margin we look at the breakdown of each study and how many subjects favoured a specific deck. This graph here adds to the theory of what we discussed earlier with regards how subjects played the game with exploitation versus exploration. We mentioned earlier how females appeared to prefer a policy of exploration over end results. The studies we mentioned then which were the Horstmann and Wood studies and have large numbers which preffered deck 2 which was one of the least favourable decks. They also had large numbers picking deck 4 which could suggest this is the more exploitative section of subjects in the studies. We also see looking at our earlier age studies that two of the seemingly younger studies SteingroeverInPrep and Wetzels follow very consistent decision making patterns in each of the most common choices. This is something we alluded to earlier in that younger adults tend to follow what they know. It is also interesting to note in the SteingroeverInPrep and the Worthy study that have large specified numbers of female participants the variety of spread of subjects across the most common choices. They both have large numbers picking deck 2 and reasonably equal spread across decks 3 and 4 too. This definitely follows on from our previous findings of potential differences in task approaches along gender lines.

\begin{sphinxVerbatim}[commandchars=\\\{\}]
\PYG{n}{standard}\PYG{o}{.}\PYG{n}{plot}\PYG{o}{.}\PYG{n}{scatter}\PYG{p}{(}\PYG{n}{x}\PYG{o}{=}\PYG{l+s+s1}{\PYGZsq{}}\PYG{l+s+s1}{Margin}\PYG{l+s+s1}{\PYGZsq{}}\PYG{p}{,} \PYG{n}{y}\PYG{o}{=}\PYG{l+s+s1}{\PYGZsq{}}\PYG{l+s+s1}{Average Choice}\PYG{l+s+s1}{\PYGZsq{}}\PYG{p}{,} \PYG{n}{c}\PYG{o}{=}\PYG{l+s+s1}{\PYGZsq{}}\PYG{l+s+s1}{StudyNumber}\PYG{l+s+s1}{\PYGZsq{}}\PYG{p}{,} \PYG{n}{cmap}\PYG{o}{=}\PYG{l+s+s1}{\PYGZsq{}}\PYG{l+s+s1}{tab10}\PYG{l+s+s1}{\PYGZsq{}}\PYG{p}{,} \PYG{n}{figsize}\PYG{o}{=}\PYG{p}{(}\PYG{l+m+mi}{16}\PYG{p}{,} \PYG{l+m+mi}{8}\PYG{p}{)}\PYG{p}{)}
\end{sphinxVerbatim}

\begin{sphinxVerbatim}[commandchars=\\\{\}]
\PYGZlt{}AxesSubplot:xlabel=\PYGZsq{}Margin\PYGZsq{}, ylabel=\PYGZsq{}Average Choice\PYGZsq{}\PYGZgt{}
\end{sphinxVerbatim}

\noindent\sphinxincludegraphics{{clustering_31_1}.png}

\sphinxAtStartPar
We will look at the margin vs average choice plot and try to combine this with our earlier graphs too to further our knowledge on subjects decison making. Using our colour bar we can deduce the clusters from what study they are a part of. If we look at the cluster to the left in our k\sphinxhyphen{}means scatter plot we can see that a substantial amount of this cluster contains subjects from the Wood et al study. It is interesting to note this had the highest mean average age of any study in the datasets. It also had a large number of participants but looking at the scatter plot very few participants made money over the course of the trials. The majority had an average choice of below 3 and certainly fell into the category of average lower choice and lower financial gain. This study also features heavily in the second cluster (the one most central) and this cluster also contains subjects who struggled to break even. The Horstmann study also features heavily in this cluster as does the Worthy study in yellow. The Worthy study leans more towards the first cluster again in the lower choice average, lower money made category. It is interesting to note that this study does not explicitly state the age demography of the group studied but tells us it was a solely female, undergraduate student study, which hints at it being a younger age group. In the third cluster to the right, which is the higher average choice, higher profit group we can see a large mix of groups with comparatively less subjects in this cluster compared to the other two. We can see a significant amount of subjects from the Maia study and also quite a few from the previously mentioned Horstmann study. We also see even with a small sample size from the study there is a significant number of Premkumar participants in this profitable cluster. Two of these studies contain a very young mean age again. The Maia study is another that focuses on undergraduate students again, but with better results than previous.


\chapter{4. K\sphinxhyphen{}Means Variation}
\label{\detokenize{k-means_variation:k-means-variation}}\label{\detokenize{k-means_variation::doc}}
\begin{sphinxVerbatim}[commandchars=\\\{\}]
\PYG{k+kn}{import} \PYG{n+nn}{pandas} \PYG{k}{as} \PYG{n+nn}{pd}
\PYG{k+kn}{import} \PYG{n+nn}{seaborn} \PYG{k}{as} \PYG{n+nn}{sn}
\PYG{k+kn}{import} \PYG{n+nn}{numpy} \PYG{k}{as} \PYG{n+nn}{np}
\PYG{k+kn}{import} \PYG{n+nn}{matplotlib}\PYG{n+nn}{.}\PYG{n+nn}{pyplot} \PYG{k}{as} \PYG{n+nn}{plt}
\PYG{k+kn}{from} \PYG{n+nn}{sklearn}\PYG{n+nn}{.}\PYG{n+nn}{cluster} \PYG{k+kn}{import} \PYG{n}{KMeans}\PYG{p}{,} \PYG{n}{AgglomerativeClustering}
\PYG{k+kn}{from} \PYG{n+nn}{sklearn}\PYG{n+nn}{.}\PYG{n+nn}{metrics} \PYG{k+kn}{import} \PYG{n}{silhouette\PYGZus{}score}
\end{sphinxVerbatim}

\sphinxAtStartPar
Import our data

\begin{sphinxVerbatim}[commandchars=\\\{\}]
\PYG{n}{cleaned95} \PYG{o}{=} \PYG{n}{pd}\PYG{o}{.}\PYG{n}{read\PYGZus{}csv}\PYG{p}{(}\PYG{l+s+s1}{\PYGZsq{}}\PYG{l+s+s1}{data/cleaned95.csv}\PYG{l+s+s1}{\PYGZsq{}}\PYG{p}{,} \PYG{n}{index\PYGZus{}col}\PYG{o}{=}\PYG{l+s+s1}{\PYGZsq{}}\PYG{l+s+s1}{Unnamed: 0}\PYG{l+s+s1}{\PYGZsq{}}\PYG{p}{)}
\PYG{n}{cleaned100} \PYG{o}{=} \PYG{n}{pd}\PYG{o}{.}\PYG{n}{read\PYGZus{}csv}\PYG{p}{(}\PYG{l+s+s1}{\PYGZsq{}}\PYG{l+s+s1}{data/cleaned100.csv}\PYG{l+s+s1}{\PYGZsq{}}\PYG{p}{,} \PYG{n}{index\PYGZus{}col}\PYG{o}{=}\PYG{l+s+s1}{\PYGZsq{}}\PYG{l+s+s1}{Unnamed: 0}\PYG{l+s+s1}{\PYGZsq{}}\PYG{p}{)}
\PYG{n}{cleaned150} \PYG{o}{=} \PYG{n}{pd}\PYG{o}{.}\PYG{n}{read\PYGZus{}csv}\PYG{p}{(}\PYG{l+s+s1}{\PYGZsq{}}\PYG{l+s+s1}{data/cleaned150.csv}\PYG{l+s+s1}{\PYGZsq{}}\PYG{p}{,} \PYG{n}{index\PYGZus{}col}\PYG{o}{=}\PYG{l+s+s1}{\PYGZsq{}}\PYG{l+s+s1}{Unnamed: 0}\PYG{l+s+s1}{\PYGZsq{}}\PYG{p}{)}
\PYG{n}{joined} \PYG{o}{=} \PYG{n}{pd}\PYG{o}{.}\PYG{n}{read\PYGZus{}csv}\PYG{p}{(}\PYG{l+s+s1}{\PYGZsq{}}\PYG{l+s+s1}{data/cleaned\PYGZus{}all.csv}\PYG{l+s+s1}{\PYGZsq{}}\PYG{p}{,} \PYG{n}{index\PYGZus{}col}\PYG{o}{=}\PYG{l+s+s1}{\PYGZsq{}}\PYG{l+s+s1}{Unnamed: 0}\PYG{l+s+s1}{\PYGZsq{}}\PYG{p}{)}
\PYG{n}{standard} \PYG{o}{=} \PYG{n}{pd}\PYG{o}{.}\PYG{n}{read\PYGZus{}csv}\PYG{p}{(}\PYG{l+s+s1}{\PYGZsq{}}\PYG{l+s+s1}{data/standardized\PYGZus{}all.csv}\PYG{l+s+s1}{\PYGZsq{}}\PYG{p}{,} \PYG{n}{index\PYGZus{}col}\PYG{o}{=}\PYG{l+s+s1}{\PYGZsq{}}\PYG{l+s+s1}{Unnamed: 0}\PYG{l+s+s1}{\PYGZsq{}}\PYG{p}{)}
\end{sphinxVerbatim}

\begin{sphinxVerbatim}[commandchars=\\\{\}]
\PYG{n}{standard}\PYG{o}{.}\PYG{n}{head}\PYG{p}{(}\PYG{p}{)}
\end{sphinxVerbatim}

\begin{sphinxVerbatim}[commandchars=\\\{\}]
     Margin  Most Common Choice Picked  Most Common Choice  Average Choice  \PYGZbs{}
0  1.044988                   1.062672            1.234405        2.271920   
1 \PYGZhy{}0.414346                  \PYGZhy{}0.804770            1.234405       \PYGZhy{}0.410317   
2 \PYGZhy{}0.474318                  \PYGZhy{}0.559054            1.234405        0.268730   
3 \PYGZhy{}0.294400                  \PYGZhy{}0.559054            1.234405        0.370587   
4  0.205371                  \PYGZhy{}0.165908            1.234405        1.049635   

   StudyNumber  
0     \PYGZhy{}1.60938  
1     \PYGZhy{}1.60938  
2     \PYGZhy{}1.60938  
3     \PYGZhy{}1.60938  
4     \PYGZhy{}1.60938  
\end{sphinxVerbatim}


\section{Methodology}
\label{\detokenize{k-means_variation:methodology}}
\sphinxAtStartPar
We are going to attempt to follow the methods stated in {[}\hyperlink{cite.zbibliography:id6}{5}{]} in his attempt at constructing a privacy preserving clustering technique based on the k\sphinxhyphen{}means algorithm. This involves a 2 step process which is as follows:
\begin{enumerate}
\sphinxsetlistlabels{\arabic}{enumi}{enumii}{}{.}%
\item {} 
\sphinxAtStartPar
Data Protection Phase and

\item {} 
\sphinxAtStartPar
Data Recovery Phase

\end{enumerate}

\sphinxAtStartPar
The first phase involving the data protection phase involves 4 key steps. Firstly, we apply the K\sphinxhyphen{}means algorithm on our dataset and then we select one of the clusters from the result. In our cluster let’s say A, we select the furthest data away from the centroid of A. We generate the set of noises by the using the following equation:
\begin{equation*}
\begin{split}
 n_i^u = d^{u} + \alpha \times (distance(c,d)) \tag{1}
\end{split}
\end{equation*}
\sphinxAtStartPar
We then use the following equation:
\begin{equation*}
\begin{split}
 p_i = |D| \times Rand(s) \tag{2}
\end{split}
\end{equation*}
\sphinxAtStartPar
This is to obtain the position of the noise from eq(1) in dataset D. This leads us on to our data recovery phase. Our first step in the phase is to use eq(2) and obtain p\_i for position of noise in D’ and commence removals. Then we delete all the noises and the original dataset D can be recovered immediately. The end result should be a dataset that shares cluster information but protects the privacy of the individuals at hand.

\begin{sphinxVerbatim}[commandchars=\\\{\}]
\PYG{n}{kmeans\PYGZus{}margin\PYGZus{}joined} \PYG{o}{=} \PYG{n}{KMeans}\PYG{p}{(}\PYG{n}{n\PYGZus{}clusters}\PYG{o}{=}\PYG{l+m+mi}{3}\PYG{p}{)}\PYG{o}{.}\PYG{n}{fit}\PYG{p}{(}\PYG{n}{standard}\PYG{p}{[}\PYG{p}{[}\PYG{l+s+s2}{\PYGZdq{}}\PYG{l+s+s2}{Margin}\PYG{l+s+s2}{\PYGZdq{}}\PYG{p}{,} \PYG{l+s+s2}{\PYGZdq{}}\PYG{l+s+s2}{Most Common Choice Picked}\PYG{l+s+s2}{\PYGZdq{}}\PYG{p}{]}\PYG{p}{]}\PYG{p}{)}
\PYG{n}{centroids\PYGZus{}betas\PYGZus{}joined} \PYG{o}{=} \PYG{n}{kmeans\PYGZus{}margin\PYGZus{}joined}\PYG{o}{.}\PYG{n}{cluster\PYGZus{}centers\PYGZus{}}
\end{sphinxVerbatim}

\begin{sphinxVerbatim}[commandchars=\\\{\}]
\PYG{n}{plt}\PYG{o}{.}\PYG{n}{figure}\PYG{p}{(}\PYG{n}{figsize}\PYG{o}{=}\PYG{p}{(}\PYG{l+m+mi}{16}\PYG{p}{,}\PYG{l+m+mi}{8}\PYG{p}{)}\PYG{p}{)}
\PYG{n}{plt}\PYG{o}{.}\PYG{n}{scatter}\PYG{p}{(}\PYG{n}{standard}\PYG{p}{[}\PYG{l+s+s1}{\PYGZsq{}}\PYG{l+s+s1}{Margin}\PYG{l+s+s1}{\PYGZsq{}}\PYG{p}{]}\PYG{p}{,} \PYG{n}{standard}\PYG{p}{[}\PYG{l+s+s1}{\PYGZsq{}}\PYG{l+s+s1}{Most Common Choice Picked}\PYG{l+s+s1}{\PYGZsq{}}\PYG{p}{]}\PYG{p}{,} \PYG{n}{c}\PYG{o}{=} \PYG{n}{kmeans\PYGZus{}margin\PYGZus{}joined}\PYG{o}{.}\PYG{n}{labels\PYGZus{}}\PYG{p}{,} \PYG{n}{cmap} \PYG{o}{=} \PYG{l+s+s2}{\PYGZdq{}}\PYG{l+s+s2}{Set1}\PYG{l+s+s2}{\PYGZdq{}}\PYG{p}{,} \PYG{n}{alpha}\PYG{o}{=}\PYG{l+m+mf}{0.5}\PYG{p}{)}
\PYG{n}{plt}\PYG{o}{.}\PYG{n}{scatter}\PYG{p}{(}\PYG{n}{centroids\PYGZus{}betas\PYGZus{}joined}\PYG{p}{[}\PYG{p}{:}\PYG{p}{,} \PYG{l+m+mi}{0}\PYG{p}{]}\PYG{p}{,} \PYG{n}{centroids\PYGZus{}betas\PYGZus{}joined}\PYG{p}{[}\PYG{p}{:}\PYG{p}{,} \PYG{l+m+mi}{1}\PYG{p}{]}\PYG{p}{,} \PYG{n}{c}\PYG{o}{=}\PYG{l+s+s1}{\PYGZsq{}}\PYG{l+s+s1}{blue}\PYG{l+s+s1}{\PYGZsq{}}\PYG{p}{,} \PYG{n}{marker}\PYG{o}{=}\PYG{l+s+s1}{\PYGZsq{}}\PYG{l+s+s1}{x}\PYG{l+s+s1}{\PYGZsq{}}\PYG{p}{)}
\PYG{n}{plt}\PYG{o}{.}\PYG{n}{title}\PYG{p}{(}\PYG{l+s+s1}{\PYGZsq{}}\PYG{l+s+s1}{K\PYGZhy{}Means cluster for all Subjects \PYGZhy{} Most Common Choice Picked}\PYG{l+s+s1}{\PYGZsq{}}\PYG{p}{)}
\PYG{n}{plt}\PYG{o}{.}\PYG{n}{xlabel}\PYG{p}{(}\PYG{l+s+s1}{\PYGZsq{}}\PYG{l+s+s1}{Margin}\PYG{l+s+s1}{\PYGZsq{}}\PYG{p}{)}
\PYG{n}{plt}\PYG{o}{.}\PYG{n}{ylabel}\PYG{p}{(}\PYG{l+s+s1}{\PYGZsq{}}\PYG{l+s+s1}{Times Most Common Choice Picked}\PYG{l+s+s1}{\PYGZsq{}}\PYG{p}{)}
\PYG{n}{plt}\PYG{o}{.}\PYG{n}{show}\PYG{p}{(}\PYG{p}{)}
\end{sphinxVerbatim}

\noindent\sphinxincludegraphics{{k-means_variation_7_0}.png}

\begin{sphinxVerbatim}[commandchars=\\\{\}]
\PYG{n}{centroids\PYGZus{}betas\PYGZus{}joined}
\end{sphinxVerbatim}

\begin{sphinxVerbatim}[commandchars=\\\{\}]
array([[\PYGZhy{}1.10872288, \PYGZhy{}0.06074673],
       [ 1.41597414,  1.55028183],
       [ 0.25911929, \PYGZhy{}0.38263846]])
\end{sphinxVerbatim}

\sphinxAtStartPar
We can tell from our above cluster centres that cluster 0 is in red, cluster 1 is the right most cluster and cluster 2 is in grey.

\begin{sphinxVerbatim}[commandchars=\\\{\}]
\PYG{n}{standard}\PYG{p}{[}\PYG{l+s+s1}{\PYGZsq{}}\PYG{l+s+s1}{cluster}\PYG{l+s+s1}{\PYGZsq{}}\PYG{p}{]} \PYG{o}{=} \PYG{n}{kmeans\PYGZus{}margin\PYGZus{}joined}\PYG{o}{.}\PYG{n}{labels\PYGZus{}}\PYG{o}{.}\PYG{n}{tolist}\PYG{p}{(}\PYG{p}{)}
\PYG{n}{standard}\PYG{o}{.}\PYG{n}{head}\PYG{p}{(}\PYG{p}{)}
\end{sphinxVerbatim}

\begin{sphinxVerbatim}[commandchars=\\\{\}]
     Margin  Most Common Choice Picked  Most Common Choice  Average Choice  \PYGZbs{}
0  1.044988                   1.062672            1.234405        2.271920   
1 \PYGZhy{}0.414346                  \PYGZhy{}0.804770            1.234405       \PYGZhy{}0.410317   
2 \PYGZhy{}0.474318                  \PYGZhy{}0.559054            1.234405        0.268730   
3 \PYGZhy{}0.294400                  \PYGZhy{}0.559054            1.234405        0.370587   
4  0.205371                  \PYGZhy{}0.165908            1.234405        1.049635   

   StudyNumber  cluster  
0     \PYGZhy{}1.60938        1  
1     \PYGZhy{}1.60938        2  
2     \PYGZhy{}1.60938        2  
3     \PYGZhy{}1.60938        2  
4     \PYGZhy{}1.60938        2  
\end{sphinxVerbatim}

\begin{sphinxVerbatim}[commandchars=\\\{\}]
\PYG{n}{cluster0} \PYG{o}{=} \PYG{n}{standard}\PYG{p}{[}\PYG{n}{standard}\PYG{o}{.}\PYG{n}{cluster}\PYG{o}{==}\PYG{l+m+mi}{0}\PYG{p}{]}
\PYG{n}{cluster0}\PYG{o}{.}\PYG{n}{head}\PYG{p}{(}\PYG{p}{)}
\end{sphinxVerbatim}

\begin{sphinxVerbatim}[commandchars=\\\{\}]
      Margin  Most Common Choice Picked  Most Common Choice  Average Choice  \PYGZbs{}
16 \PYGZhy{}1.154008                   0.374667           \PYGZhy{}0.923181       \PYGZhy{}1.233097   
20 \PYGZhy{}1.074044                  \PYGZhy{}0.067622           \PYGZhy{}2.001975       \PYGZhy{}1.319109   
26 \PYGZhy{}0.594263                   1.357531           \PYGZhy{}0.923181       \PYGZhy{}0.889046   
28 \PYGZhy{}1.673771                   2.586111           \PYGZhy{}0.923181       \PYGZhy{}1.899695   
37 \PYGZhy{}1.193990                   2.094679           \PYGZhy{}0.923181       \PYGZhy{}0.631008   

    StudyNumber  cluster  
16     1.318707        0  
20     1.318707        0  
26     1.318707        0  
28     1.318707        0  
37     1.318707        0  
\end{sphinxVerbatim}

\sphinxAtStartPar
From looking at our graph and cluster 0, I feel data points with margin values less than \sphinxhyphen{}2 would be classified as noise.

\begin{sphinxVerbatim}[commandchars=\\\{\}]
\PYG{n}{noises} \PYG{o}{=} \PYG{n}{cluster0}\PYG{p}{[}\PYG{n}{cluster0}\PYG{o}{.}\PYG{n}{Margin} \PYG{o}{\PYGZlt{}}\PYG{o}{=} \PYG{o}{\PYGZhy{}}\PYG{l+m+mi}{2}\PYG{p}{]}
\PYG{n}{noises}
\end{sphinxVerbatim}

\begin{sphinxVerbatim}[commandchars=\\\{\}]
       Margin  Most Common Choice Picked  Most Common Choice  Average Choice  \PYGZbs{}
43  \PYGZhy{}2.593351                   4.600983           \PYGZhy{}0.923181       \PYGZhy{}2.114727   
49  \PYGZhy{}3.233059                   3.224973           \PYGZhy{}0.923181       \PYGZhy{}2.566293   
287 \PYGZhy{}2.501393                  \PYGZhy{}0.657340           \PYGZhy{}2.001975       \PYGZhy{}2.340510   
404 \PYGZhy{}2.829243                   0.522097           \PYGZhy{}0.923181       \PYGZhy{}1.630905   
450 \PYGZhy{}3.273040                  \PYGZhy{}0.067622           \PYGZhy{}0.923181       \PYGZhy{}2.405019   
501 \PYGZhy{}2.853232                   0.522097           \PYGZhy{}0.923181       \PYGZhy{}1.888943   
531 \PYGZhy{}2.453415                   0.522097           \PYGZhy{}0.923181       \PYGZhy{}0.953555   
547 \PYGZhy{}3.113113                   0.522097           \PYGZhy{}0.923181       \PYGZhy{}2.146981   
572 \PYGZhy{}2.505391                   0.522097           \PYGZhy{}0.923181       \PYGZhy{}1.308358   
573 \PYGZhy{}2.505391                   0.522097           \PYGZhy{}0.923181       \PYGZhy{}1.308358   

     StudyNumber  cluster  
43      1.318707        0  
49      1.318707        0  
287    \PYGZhy{}0.877358        0  
404     0.220674        0  
450     0.586685        0  
501     0.586685        0  
531     0.586685        0  
547     0.586685        0  
572     0.586685        0  
573     0.586685        0  
\end{sphinxVerbatim}

\begin{sphinxVerbatim}[commandchars=\\\{\}]
\PYG{n}{marginnoise} \PYG{o}{=} \PYG{n}{pd}\PYG{o}{.}\PYG{n}{DataFrame}\PYG{p}{(}\PYG{n}{np}\PYG{o}{.}\PYG{n}{random}\PYG{o}{.}\PYG{n}{uniform}\PYG{p}{(}\PYG{o}{\PYGZhy{}}\PYG{l+m+mf}{3.4}\PYG{p}{,}\PYG{o}{\PYGZhy{}}\PYG{l+m+mf}{2.5}\PYG{p}{,}\PYG{l+m+mi}{15}\PYG{p}{)}\PYG{p}{)}
\end{sphinxVerbatim}

\begin{sphinxVerbatim}[commandchars=\\\{\}]
\PYG{n}{marginnoise} \PYG{o}{=} \PYG{n}{marginnoise}\PYG{o}{.}\PYG{n}{rename}\PYG{p}{(}\PYG{n}{columns}\PYG{o}{=}\PYG{p}{\PYGZob{}}\PYG{l+m+mi}{0}\PYG{p}{:} \PYG{l+s+s1}{\PYGZsq{}}\PYG{l+s+s1}{Margin}\PYG{l+s+s1}{\PYGZsq{}}\PYG{p}{\PYGZcb{}}\PYG{p}{)}
\PYG{n}{marginnoise}\PYG{o}{.}\PYG{n}{head}\PYG{p}{(}\PYG{p}{)}
\end{sphinxVerbatim}

\begin{sphinxVerbatim}[commandchars=\\\{\}]
\PYG{n}{choicenoise} \PYG{o}{=} \PYG{n}{pd}\PYG{o}{.}\PYG{n}{DataFrame}\PYG{p}{(}\PYG{n}{np}\PYG{o}{.}\PYG{n}{random}\PYG{o}{.}\PYG{n}{uniform}\PYG{p}{(}\PYG{o}{\PYGZhy{}}\PYG{l+m+mf}{0.1}\PYG{p}{,}\PYG{l+m+mi}{4}\PYG{p}{,}\PYG{l+m+mi}{15}\PYG{p}{)}\PYG{p}{)}
\end{sphinxVerbatim}

\begin{sphinxVerbatim}[commandchars=\\\{\}]
\PYG{n}{choicenoise} \PYG{o}{=} \PYG{n}{choicenoise}\PYG{o}{.}\PYG{n}{rename}\PYG{p}{(}\PYG{n}{columns}\PYG{o}{=}\PYG{p}{\PYGZob{}}\PYG{l+m+mi}{0}\PYG{p}{:} \PYG{l+s+s1}{\PYGZsq{}}\PYG{l+s+s1}{Most Common Choice Picked}\PYG{l+s+s1}{\PYGZsq{}}\PYG{p}{\PYGZcb{}}\PYG{p}{)}
\PYG{n}{choicenoise}\PYG{o}{.}\PYG{n}{head}\PYG{p}{(}\PYG{p}{)}
\end{sphinxVerbatim}

\begin{sphinxVerbatim}[commandchars=\\\{\}]
   Most Common Choice Picked
0                   2.220171
1                   2.045314
2                   3.831233
3                   3.094479
4                   2.395261
\end{sphinxVerbatim}

\begin{sphinxVerbatim}[commandchars=\\\{\}]
\PYG{n}{noise} \PYG{o}{=} \PYG{n}{pd}\PYG{o}{.}\PYG{n}{concat}\PYG{p}{(}\PYG{p}{[}\PYG{n}{marginnoise}\PYG{p}{,} \PYG{n}{choicenoise}\PYG{p}{]}\PYG{p}{,} \PYG{n}{axis}\PYG{o}{=}\PYG{l+m+mi}{1}\PYG{p}{)}
\end{sphinxVerbatim}

\sphinxAtStartPar
Our noise data is now generated we add this back to the original dataset now.

\begin{sphinxVerbatim}[commandchars=\\\{\}]
\PYG{n}{noisesf} \PYG{o}{=} \PYG{n}{noises}\PYG{p}{[}\PYG{p}{[}\PYG{l+s+s1}{\PYGZsq{}}\PYG{l+s+s1}{Margin}\PYG{l+s+s1}{\PYGZsq{}}\PYG{p}{,} \PYG{l+s+s1}{\PYGZsq{}}\PYG{l+s+s1}{Most Common Choice Picked}\PYG{l+s+s1}{\PYGZsq{}}\PYG{p}{]}\PYG{p}{]}
\PYG{n}{noisesf} \PYG{o}{=} \PYG{n}{noisesf}\PYG{o}{.}\PYG{n}{to\PYGZus{}numpy}\PYG{p}{(}\PYG{p}{)}
\PYG{n}{noisesf}
\end{sphinxVerbatim}

\begin{sphinxVerbatim}[commandchars=\\\{\}]
\PYG{n}{centroids\PYGZus{}betas\PYGZus{}joined}\PYG{p}{[}\PYG{l+m+mi}{0}\PYG{p}{]}
\end{sphinxVerbatim}

\begin{sphinxVerbatim}[commandchars=\\\{\}]
array([\PYGZhy{}1.10872288, \PYGZhy{}0.06074673])
\end{sphinxVerbatim}

\begin{sphinxVerbatim}[commandchars=\\\{\}]
\PYG{k+kn}{from} \PYG{n+nn}{scipy}\PYG{n+nn}{.}\PYG{n+nn}{spatial}\PYG{n+nn}{.}\PYG{n+nn}{distance} \PYG{k+kn}{import} \PYG{n}{cdist}
\PYG{k+kn}{from} \PYG{n+nn}{scipy}\PYG{n+nn}{.}\PYG{n+nn}{spatial} \PYG{k+kn}{import} \PYG{n}{distance}

\PYG{n}{distances} \PYG{o}{=} \PYG{n}{distance}\PYG{o}{.}\PYG{n}{cdist}\PYG{p}{(}\PYG{n}{centroids\PYGZus{}betas\PYGZus{}joined}\PYG{p}{,} \PYG{n}{noisesf}\PYG{p}{,} \PYG{l+s+s1}{\PYGZsq{}}\PYG{l+s+s1}{euclidean}\PYG{l+s+s1}{\PYGZsq{}}\PYG{p}{)}
\PYG{n}{distances}\PYG{p}{[}\PYG{l+m+mi}{0}\PYG{p}{]}
\end{sphinxVerbatim}

\begin{sphinxVerbatim}[commandchars=\\\{\}]
array([4.89242699, 3.91264062, 1.51507518, 1.81656155, 2.16432834,
       1.8392984 , 1.46557233, 2.0874117 , 1.51340276, 1.51340276])
\end{sphinxVerbatim}

\sphinxAtStartPar
Our furtherest point away from the centroid of our choosen cluster 0 is the first point we see in the array here. This will be denoted as our data “d” the furtherest point from our centroid, C of cluster 0.

\sphinxAtStartPar
We now need to use this distance value obtained and combine it with our “noise offset ratio” denoted as α. From our previousy cited paper this value is typically in the range 0 \sphinxhyphen{} 0.05. We will use 0.05 for α here.

\begin{sphinxVerbatim}[commandchars=\\\{\}]
\PYG{n}{d} \PYG{o}{=} \PYG{n}{distances}\PYG{p}{[}\PYG{l+m+mi}{0}\PYG{p}{]}\PYG{p}{[}\PYG{l+m+mi}{0}\PYG{p}{]}
\PYG{n}{alpha} \PYG{o}{=} \PYG{l+m+mf}{0.05}
\end{sphinxVerbatim}

\begin{sphinxVerbatim}[commandchars=\\\{\}]
\PYG{n}{d} \PYG{o}{*} \PYG{n}{alpha}
\end{sphinxVerbatim}

\begin{sphinxVerbatim}[commandchars=\\\{\}]
0.2446213497147065
\end{sphinxVerbatim}

\sphinxAtStartPar
We will now add our noises to dataset we generated above. This is the last step as part of our data protection phase.

\begin{sphinxVerbatim}[commandchars=\\\{\}]
\PYG{n}{clusterframes} \PYG{o}{=} \PYG{n}{standard}\PYG{p}{[}\PYG{p}{[}\PYG{l+s+s1}{\PYGZsq{}}\PYG{l+s+s1}{Margin}\PYG{l+s+s1}{\PYGZsq{}}\PYG{p}{,} \PYG{l+s+s1}{\PYGZsq{}}\PYG{l+s+s1}{Most Common Choice Picked}\PYG{l+s+s1}{\PYGZsq{}}\PYG{p}{]}\PYG{p}{]}
\end{sphinxVerbatim}

\begin{sphinxVerbatim}[commandchars=\\\{\}]
\PYG{n}{clusterframes} \PYG{o}{=} \PYG{n}{clusterframes}\PYG{o}{.}\PYG{n}{append}\PYG{p}{(}\PYG{n}{noise}\PYG{p}{)}
\PYG{n}{clusterframes}
\end{sphinxVerbatim}

\sphinxAtStartPar
We now need to randomly shuffle our dataframe with noises added so it is harder to identify our noise.

\begin{sphinxVerbatim}[commandchars=\\\{\}]
\PYG{n}{sample} \PYG{o}{=} \PYG{n}{clusterframes}\PYG{o}{.}\PYG{n}{sample}\PYG{p}{(}\PYG{n}{frac}\PYG{o}{=}\PYG{l+m+mi}{1}\PYG{p}{)}\PYG{o}{.}\PYG{n}{reset\PYGZus{}index}\PYG{p}{(}\PYG{n}{drop}\PYG{o}{=}\PYG{k+kc}{True}\PYG{p}{)}
\PYG{n}{sample}
\end{sphinxVerbatim}

\begin{sphinxVerbatim}[commandchars=\\\{\}]
       Margin  Most Common Choice Picked
0    0.613185                   0.522097
1   \PYGZhy{}0.462324                  \PYGZhy{}0.018478
2   \PYGZhy{}0.354373                  \PYGZhy{}0.706483
3   \PYGZhy{}0.914117                  \PYGZhy{}0.313338
4   \PYGZhy{}1.321931                  \PYGZhy{}0.755627
..        ...                        ...
627  0.245353                  \PYGZhy{}0.952199
628 \PYGZhy{}0.194446                   0.079808
629  0.045444                  \PYGZhy{}0.411624
630 \PYGZhy{}3.313970                   0.733010
631 \PYGZhy{}0.698216                  \PYGZhy{}0.313338

[632 rows x 2 columns]
\end{sphinxVerbatim}

\sphinxAtStartPar
With our dataframe randomly assigned we now begin our data recovery phase.


\chapter{Conclusions}
\label{\detokenize{conclusion:conclusions}}\label{\detokenize{conclusion::doc}}

\section{Consistent Choice, Consistent Rewards}
\label{\detokenize{conclusion:consistent-choice-consistent-rewards}}
\sphinxAtStartPar
In our data analysis using the k\sphinxhyphen{}means algorithm we looked at how does profit / loss margins correlate with a subjects consistency of choice and how regularly they pick their favoured deck. From our K\sphinxhyphen{}means analysis we learned that a consistency of higher choice led to greater financial rewards at the end of the trials. Although there was some outliers to the left of our scatter plot that picked their favourite deck a high number of times, we could see from our margin vs average choice plot that these subjects most likely picked lower numbers a large percentage of the time.

\sphinxAtStartPar
\sphinxincludegraphics{{image-475x317}.png}


\section{Different strategies from males and females}
\label{\detokenize{conclusion:different-strategies-from-males-and-females}}
\sphinxAtStartPar
It was also interesting to note from our scatter plot that there appeared to be a pronounced difference in how males and females selected their decks. Studies that indicated the number of female particpiants in their work (and that had high female representation) appeared to mix up their choices regularly over the course of the trials even after an initial period of observation. Whereas studies that did not indicate the number of female particpiants or had lower numbers appeared to be more consistent with their choices, experimenting initially and then sticking to decks they feel are winners.


\chapter{Bibliography}
\label{\detokenize{zbibliography:bibliography}}\label{\detokenize{zbibliography::doc}}
\sphinxAtStartPar


\begin{sphinxthebibliography}{1}
\bibitem[1]{zbibliography:id2}
\sphinxAtStartPar
Kevin M. Beitz, Timothy A. Salthouse and Hasker P. Davis. Performance on the iowa gambling task: from 5 to 89 years of age. \sphinxstyleemphasis{Journal of experimental psychology: General}, pages 1677–1689, 2014.
\bibitem[2]{zbibliography:id3}
\sphinxAtStartPar
Leoniede Visser Ruud van den Bos, Judith Homberg. A critical review of sex differences in decision\sphinxhyphen{}making tasks: focus on the iowa gambling task. \sphinxstyleemphasis{Behavioural Brain Research}, 238:Pages 95–108, 2012.
\bibitem[3]{zbibliography:id4}
\sphinxAtStartPar
Suzanna Vlaar Ruud van den Bos, Esther den Heijer and Bart Houx. Exploring gender differences in decision\sphinxhyphen{}making using the iowa gambling task. \sphinxstyleemphasis{Psychology of Decision Making in Education, Behavior, and High Risk Situations}, pages Pages 1115–1134, 2007.
\bibitem[4]{zbibliography:id5}
\sphinxAtStartPar
Ruth Garrido\sphinxhyphen{}Chaves, Mario Perez\sphinxhyphen{}Alarcon, Vanesa Perez, Vanesa Hidalgo, Matias M. Pulopulos, and Alicia Salvador. Frn and p3 during the iowa gambling task: the importance of gender. \sphinxstyleemphasis{Pyschophysiology}, 2021.
\bibitem[5]{zbibliography:id6}
\sphinxAtStartPar
Chen\sphinxhyphen{}Yi Lin. A reversible privacy\sphinxhyphen{}preserving clustering technique based on k\sphinxhyphen{}means algorithm. \sphinxstyleemphasis{Applied Soft Computing}, 2020.
\end{sphinxthebibliography}







\renewcommand{\indexname}{Index}
\printindex
\end{document}